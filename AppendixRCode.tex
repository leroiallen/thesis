\documentclass[12pt,a4paper,oneside]{report}
\usepackage{fancyhdr}	
%\usepackage{fancyheadings}	
\usepackage{geometry}%[margin=2.5cm, a4paper][top=15mm, bottom=15mm, left=35mm, right=15mm]
\usepackage{amsmath}			% packages to give lots of maths stuff
\usepackage{amssymb}
\usepackage{color}
\usepackage[parfill]{parskip}    % Activate to begin paragraphs with an empty line rather than an inden
\usepackage{graphicx}	
\usepackage{fancyvrb}
\usepackage{array}
\usepackage{listings}
\usepackage{amsthm}
\usepackage{dsfont}

%\usepackage[latin1]{inputenc}
%%% make dejvu monospace default tt font
\usepackage[scaled=.8]{beramono}
\usepackage[T1]{fontenc}

\setlength{\textwidth}{150.0mm}
\setlength{\textheight}{220.0mm}


% Set up the naming conventions for equations etc
\renewcommand{\theequation}{\arabic{chapter}.\arabic{equation}}%.\arabic{section}
\renewcommand{\thefigure}{\arabic{chapter}.\arabic{figure}}%.\arabic{section}
\renewcommand{\thetable}{\arabic{chapter}.\arabic{table}}%.\arabic{section}


%% This is for the Verbatim environments using pygments
%% shortlist: paraiso-light, manni, autumn, xcode, default, trac, native, friendly

\makeatletter
\def\PY@reset{\let\PY@it=\relax \let\PY@bf=\relax%
    \let\PY@ul=\relax \let\PY@tc=\relax%
    \let\PY@bc=\relax \let\PY@ff=\relax}
\def\PY@tok#1{\csname PY@tok@#1\endcsname}
\def\PY@toks#1+{\ifx\relax#1\empty\else%
    \PY@tok{#1}\expandafter\PY@toks\fi}
\def\PY@do#1{\PY@bc{\PY@tc{\PY@ul{%
    \PY@it{\PY@bf{\PY@ff{#1}}}}}}}
\def\PY#1#2{\PY@reset\PY@toks#1+\relax+\PY@do{#2}}

\expandafter\def\csname PY@tok@gd\endcsname{\def\PY@bc##1{\setlength{\fboxsep}{0pt}\fcolorbox[rgb]{0.80,0.00,0.00}{1.00,0.80,0.80}{\strut ##1}}}
\expandafter\def\csname PY@tok@gu\endcsname{\let\PY@bf=\textbf\def\PY@tc##1{\textcolor[rgb]{0.00,0.20,0.00}{##1}}}
\expandafter\def\csname PY@tok@gt\endcsname{\def\PY@tc##1{\textcolor[rgb]{0.60,0.80,0.40}{##1}}}
\expandafter\def\csname PY@tok@gs\endcsname{\let\PY@bf=\textbf}
\expandafter\def\csname PY@tok@gr\endcsname{\def\PY@tc##1{\textcolor[rgb]{1.00,0.00,0.00}{##1}}}
\expandafter\def\csname PY@tok@cm\endcsname{\let\PY@it=\textit\def\PY@tc##1{\textcolor[rgb]{0.00,0.60,1.00}{##1}}}
\expandafter\def\csname PY@tok@vg\endcsname{\def\PY@tc##1{\textcolor[rgb]{0.00,0.20,0.20}{##1}}}
\expandafter\def\csname PY@tok@m\endcsname{\def\PY@tc##1{\textcolor[rgb]{0.9333333 ,0.1725490 ,0.1725490}{##1}}}   %%%% edited:  CONSTANT/NUMBER
\expandafter\def\csname PY@tok@mh\endcsname{\def\PY@tc##1{\textcolor[rgb]{1.00,0.40,0.00}{##1}}}
\expandafter\def\csname PY@tok@cs\endcsname{\let\PY@bf=\textbf\let\PY@it=\textit\def\PY@tc##1{\textcolor[rgb]{0.00,0.60,1.00}{##1}}}
\expandafter\def\csname PY@tok@ge\endcsname{\let\PY@it=\textit}
\expandafter\def\csname PY@tok@vc\endcsname{\def\PY@tc##1{\textcolor[rgb]{0.00,0.20,0.20}{##1}}}
\expandafter\def\csname PY@tok@il\endcsname{\def\PY@tc##1{\textcolor[rgb]{1.00,0.40,0.00}{##1}}}
\expandafter\def\csname PY@tok@go\endcsname{\def\PY@tc##1{\textcolor[rgb]{0.67,0.67,0.67}{##1}}}
\expandafter\def\csname PY@tok@cp\endcsname{\def\PY@tc##1{\textcolor[rgb]{1.00,0.40,0.0}{##1}}} %%%% "#include" statement in C code
\expandafter\def\csname PY@tok@gi\endcsname{\def\PY@bc##1{\setlength{\fboxsep}{0pt}\fcolorbox[rgb]{0.00,0.80,0.00}{0.80,1.00,0.80}{\strut ##1}}}
\expandafter\def\csname PY@tok@gh\endcsname{\let\PY@bf=\textbf\def\PY@tc##1{\textcolor[rgb]{0.00,0.20,0.00}{##1}}}
\expandafter\def\csname PY@tok@ni\endcsname{\let\PY@bf=\textbf\def\PY@tc##1{\textcolor[rgb]{0.60,0.60,0.60}{##1}}}
\expandafter\def\csname PY@tok@nl\endcsname{\def\PY@tc##1{\textcolor[rgb]{0.60,0.60,1.00}{##1}}}
\expandafter\def\csname PY@tok@nn\endcsname{\def\PY@tc##1{\textcolor[rgb]{0.8509804, 0.3921569, 0.2901961}{##1}}}  %edited %namespace
\expandafter\def\csname PY@tok@no\endcsname{\def\PY@tc##1{\textcolor[rgb]{0.20,0.40,0.00}{##1}}}
\expandafter\def\csname PY@tok@na\endcsname{\def\PY@tc##1{\textcolor[rgb]{0.20,0.00,0.60}{##1}}}
\expandafter\def\csname PY@tok@nb\endcsname{\def\PY@tc##1{\textcolor[rgb]{0.20,0.40,0.40}{##1}}}
\expandafter\def\csname PY@tok@nc\endcsname{\def\PY@tc##1{\textcolor[rgb]{0.8509804, 0.3921569, 0.2901961}{##1}}}  %edited %name class
\expandafter\def\csname PY@tok@nd\endcsname{\def\PY@tc##1{\textcolor[rgb]{0.60,0.60,1.00}{##1}}}
\expandafter\def\csname PY@tok@ne\endcsname{\let\PY@bf=\textbf\def\PY@tc##1{\textcolor[rgb]{0.80,0.00,0.00}{##1}}}
\expandafter\def\csname PY@tok@nf\endcsname{\def\PY@tc##1{\textcolor[rgb]{0.00,0.60,1.00}{##1}}} %%% edited: FUNCTION NAME (C)
\expandafter\def\csname PY@tok@si\endcsname{\def\PY@tc##1{\textcolor[rgb]{0.67,0.00,0.00}{##1}}}
\expandafter\def\csname PY@tok@s2\endcsname{\def\PY@tc##1{\textcolor[rgb]{0.80,0.20,0.00}{##1}}}
\expandafter\def\csname PY@tok@vi\endcsname{\def\PY@tc##1{\textcolor[rgb]{0.00,0.20,0.20}{##1}}}
\expandafter\def\csname PY@tok@nt\endcsname{\def\PY@tc##1{\textcolor[rgb]{0.8509804, 0.3921569, 0.2901961}{##1}}} %edited %name tag
\expandafter\def\csname PY@tok@nv\endcsname{\def\PY@tc##1{\textcolor[rgb]{0.00,0.20,0.20}{##1}}}
\expandafter\def\csname PY@tok@s1\endcsname{\def\PY@tc##1{\textcolor[rgb]{0.80,0.20,0.00}{##1}}}
\expandafter\def\csname PY@tok@gp\endcsname{\let\PY@bf=\textbf\def\PY@tc##1{\textcolor[rgb]{0.00,0.00,0.60}{##1}}}
\expandafter\def\csname PY@tok@sh\endcsname{\def\PY@tc##1{\textcolor[rgb]{0.80,0.20,0.00}{##1}}}
\expandafter\def\csname PY@tok@ow\endcsname{\let\PY@bf=\textbf\def\PY@tc##1{\textcolor[rgb]{0.00,0.00,0.00}{##1}}}
\expandafter\def\csname PY@tok@sx\endcsname{\def\PY@tc##1{\textcolor[rgb]{0.80,0.20,0.00}{##1}}}
\expandafter\def\csname PY@tok@bp\endcsname{\def\PY@tc##1{\textcolor[rgb]{0.20,0.40,0.40}{##1}}}
\expandafter\def\csname PY@tok@c1\endcsname{\let\PY@it=\textit\def\PY@tc##1{\textcolor[rgb]{ 0.6 ,0.6 ,0.6}{##1}}}  %%% edited: COMMENTS 
\expandafter\def\csname PY@tok@kc\endcsname{\def\PY@tc##1{\textcolor[rgb]{0.9333333 ,0.1725490 ,0.1725490}{##1}}} % edited:  CONSTANT/NUMBER                      eg NULL NA,TRUE,FALSE
\expandafter\def\csname PY@tok@c\endcsname{\let\PY@it=\textit\def\PY@tc##1{\textcolor[rgb]{  0.6 ,0.6 ,0.6}{##1}}}  %%% edited: COMMENTS  
\expandafter\def\csname PY@tok@mf\endcsname{\def\PY@tc##1{\textcolor[rgb]{1.00,0.40,0.00}{##1}}}
\expandafter\def\csname PY@tok@err\endcsname{\def\PY@tc##1{\textcolor[rgb]{0.67,0.00,0.00}{##1}}\def\PY@bc##1{\setlength{\fboxsep}{0pt}\colorbox[rgb]{1.00,0.67,0.67}{\strut ##1}}}
\expandafter\def\csname PY@tok@mb\endcsname{\def\PY@tc##1{\textcolor[rgb]{1.00,0.40,0.00}{##1}}}
\expandafter\def\csname PY@tok@ss\endcsname{\def\PY@tc##1{\textcolor[rgb]{1.00,0.80,0.20}{##1}}}
\expandafter\def\csname PY@tok@sr\endcsname{\def\PY@tc##1{\textcolor[rgb]{0.20,0.67,0.67}{##1}}}
\expandafter\def\csname PY@tok@mo\endcsname{\def\PY@tc##1{\textcolor[rgb]{1.00,0.40,0.00}{##1}}}
\expandafter\def\csname PY@tok@kd\endcsname{\let\PY@bf=\textbf\def\PY@tc##1{\textcolor[rgb]{0.00,0.40,0.60}{##1}}}
\expandafter\def\csname PY@tok@mi\endcsname{\def\PY@tc##1{\textcolor[rgb]{0.9333333 ,0.1725490 ,0.1725490}{##1}}} % edited:  CONSTANT/NUMBER  (C-code)
\expandafter\def\csname PY@tok@kn\endcsname{\def\PY@tc##1{\textcolor[rgb]{0.1333333, 0.5450980, 0.1333333}{##1}}} % edited: KEYWORD                like library() etc
\expandafter\def\csname PY@tok@o\endcsname{\def\PY@tc##1{\textcolor[rgb]{ 0.00,0.60,1.00}{##1}}} %%% edited: OPERATOR
\expandafter\def\csname PY@tok@kr\endcsname{\def\PY@tc##1{\textcolor[rgb]{0.1333333, 0.5450980, 0.1333333}{##1}}} %edited KEYWORD                      reserved: eg else,if,
\expandafter\def\csname PY@tok@s\endcsname{\def\PY@tc##1{\textcolor[rgb]{1.00,0.40,0.0}{##1}}} % edited: STRING 
\expandafter\def\csname PY@tok@kp\endcsname{\def\PY@tc##1{\textcolor[rgb]{0.1333333, 0.5450980, 0.1333333}{##1}}} % edited: KEYWORD                like else, length, nrow etc
\expandafter\def\csname PY@tok@w\endcsname{\def\PY@tc##1{\textcolor[rgb]{0.73,0.73,0.73}{##1}}}
\expandafter\def\csname PY@tok@kt\endcsname{\def\PY@tc##1{\textcolor[rgb]{0.1333333, 0.5450980, 0.1333333}{##1}}} % edited: KEYWORD          like matrix, max etc
\expandafter\def\csname PY@tok@sc\endcsname{\def\PY@tc##1{\textcolor[rgb]{0.80,0.20,0.00}{##1}}}
\expandafter\def\csname PY@tok@sb\endcsname{\def\PY@tc##1{\textcolor[rgb]{0.80,0.20,0.00}{##1}}}
\expandafter\def\csname PY@tok@k\endcsname{\def\PY@tc##1{\textcolor[rgb]{0.1333333, 0.5450980, 0.1333333}{##1}}} %edited: KEYWORD   e.g. void, int double
\expandafter\def\csname PY@tok@se\endcsname{\let\PY@bf=\textbf\def\PY@tc##1{\textcolor[rgb]{0.80,0.20,0.00}{##1}}}%edited
\expandafter\def\csname PY@tok@sd\endcsname{\let\PY@it=\textit\def\PY@tc##1{\textcolor[rgb]{0.80,0.20,0.00}{##1}}}%edited



%%% KEYWORD ::: green
%%% STRING ::: orange
%%% OPERATOR ::: blue
%%% CONSTANT/NUMBER ::: red

%%% green: 0.1333333, 0.5450980, 0.1333333
%%% red: 0.9333333 ,0.1725490 ,0.1725490
%%% blue: 0.00,0.60,1.00
%%% orange: 1.00,0.40,0.0



% > c(col2rgb("#40152A")/255)
% [1] 0.25098039 0.08235294 0.16470588
% > c(col2rgb("#731630")/255)
% [1] 0.45098039 0.08627451 0.18823529
% > c(col2rgb("#D9644A")/255)
% [1] 0.8509804 0.3921569 0.2901961
% > c(col2rgb("springgreen2")/255)
% [1] 0.0000000 0.9333333 0.4627451
% > c(col2rgb("#285283")/255)
% [1] 0.1568627 0.3215686 0.5137255
% > c(col2rgb("forestgreen")/255)
% [1] 0.1333333 0.5450980 0.1333333
% > c(col2rgb("firebrick2")/255)
% [1] 0.9333333 0.1725490 0.1725490
% > c(col2rgb("orange")/255)
% [1] 1.0000000 0.6470588 0.0000000
\def\PYZbs{\char`\\}
\def\PYZus{\char`\_}
\def\PYZob{\char`\{}
\def\PYZcb{\char`\}}
\def\PYZca{\char`\^}
\def\PYZam{\char`\&}
\def\PYZlt{\char`\<}
\def\PYZgt{\char`\>}
\def\PYZsh{\char`\#}
\def\PYZpc{\char`\%}
\def\PYZdl{\char`\$}
\def\PYZhy{\char`\-}
\def\PYZsq{\char`\'}
\def\PYZdq{\char`\"}
\def\PYZti{\char`\~}
% for compatibility with earlier versions
\def\PYZat{@}
\def\PYZlb{[}
\def\PYZrb{]}
\makeatother





\lhead[\fancyplain{}{\thepage}]{\fancyplain{}{\nouppercase\rightmark}}
\rhead[\fancyplain{}{\nouppercase\leftmark}]{\fancyplain{}{\thepage}}
\cfoot{}


%\pagestyle{empty}
 
 
\begin{document}

\pagenumbering{roman}

\tableofcontents

\clearpage


\pagenumbering{arabic}
\pagestyle{fancy}
% What do you want at the top and bottom of a page (see fancyheadings.sty)
%\rhead{\thepage}
%\rhead{\thepage}



%%%%%%%%%%%%%%%%%%%%%%%%%%%%%%%%%
%%%%%%%%%%%%%%%%%%%%%%%%%%%%%%%%%
%%%%%%%%%      Some commands for repeated markup        
%%%%%%%%%%%%%%%%%%%%%%%%%%%%%%%%%
%%%%%%%%%%%%%%%%%%%%%%%%%%%%%%%%%


\definecolor{linkblue}{rgb}{0.192,0.494,0.675}
\definecolor{fngrey}{rgb}{0.220,0.220,0.251}	

\newcommand{\codetab}[1]{%
\begin{center}
\begin{tabular}{m{4.5cm}m{5cm}r}
\hline
\textsf{Description} & \textsf{File} & \textsf{Functions} \\
  \hline
#1
   \hline
\end{tabular}
\end{center}
}

\newcommand{\codeentry}[3]{%
\multicolumn{3}{l}{\textsf{#1:}} \\
 &  \texttt{\textcolor{linkblue}{/#2}}  & \textcolor{fngrey}{\texttt{#3}} \\ %
}

\newcommand{\codeinp}[1]{%
\input{tex/#1}%
\clearpage%
}

\newcommand{\pagevertcentre}[1]{%
\pagestyle{plain}%
%\topskip0pt
\vspace*{\fill}
#1
\vspace*{\fill}
\clearpage%
\pagestyle{fancy}
}

%%%%%%%%%%%%%%%%%%%%%%%%%%%%%%%%%
%%%%%%%%%%%%%%%%%%%%%%%%%%%%%%%%%
%%%%%%%%%      Appendices        
%%%%%%%%%%%%%%%%%%%%%%%%%%%%%%%%%
%%%%%%%%%%%%%%%%%%%%%%%%%%%%%%%%%




% Set up the naming conventions for equations in the appendices
\renewcommand{\theequation}{\Alph{chapter}.\arabic{equation}} %\arabic{section}.
 


\begin{appendix}

\addtocontents{toc}{\setcounter{tocdepth}{3}}



\chapter{\texttt{R} code} \label{app:R}


All files referenced in the current appendix are available in the \texttt{\textcolor{linkblue}{R/}} directory at:
\begin{center}
\texttt{\textcolor{linkblue}{https://github.com/tystan/thesis/}}.
\end{center}



\clearpage



\pagevertcentre{
\section{Morphological operators}
\begin{center}
\begin{tabular}{m{3cm}m{5cm}r}
\hline
\textsf{Description} & \textsf{File} & \textsf{Functions} \\
  \hline
\codeentry{Naive erosion and top-hat}{00\_erosion\_slow.R}{erode(), dilate(), tophat()}
\codeentry{Line segment erosion}{01\_erosion\_quick.R}{erode\_quick()}
\codeentry{Naive erosion for unequally spaced values}{02\_cts\_erosion\_slow.R}{erode\_cts\_slow()}
\codeentry{Continuous line segment erosion}{03\_cts\_erosion\_quick.R}{erode\_cts\_quick()}
   \hline
\end{tabular}
\end{center}
}


\subsection{Naive erosion and top-hat} 

Below is a simple (and naive) implementation of an erosion, dilation and top-hat operator for $x \in \left\{ 1,2,\ldots,n \right\} = X$ and $ f \left( x \right) \in \mathds{R} \; \; \forall x \in X$. Because of the assumed even spacing of the elements of $X$, the {\tt R}-function below simply requires the vector of intensities, $f$, and the size of the SE. 

Please note the code checks the SE size provided is an odd integer because symmetric SE is not possible with an even SE size. The maximum and minimum statements on line 11 of the code segment below, namely {\tt max(1,i-k0)} and {\tt min(nx,i+k0)}, check for when the SE sits over the `edge' on the left or right of the series, respectively. This ensures only defined $f$ values will be used. \\
\codeinp{00_erosion_slow}

	
	
\subsection{Line segment erosion} 

\codeinp{01_erosion_quick}


		
\subsection{Naive erosion for unequally spaced values}


The function \texttt{get\_lo\_bounds()} creates a vector, \texttt{LO}, of all the lower bounds indexes such that $\texttt{LO[i]}=\arg\min_{j} x_j \geq x_i - k/2$ for each $x_i$, $i=1,2,\hdots,n$. This function implements an $O (n)$ algorithm using two pointers that move along the input vector $X$ from left to right. One pointer is the current position, the other is a lagging pointer that moves along the vector when required. To find the upper bounds, the same algorithm would is employed but the pointers start from the right and move down the vector with the second point lagging to the right.\\
\codeinp{02_cts_erosion_slow}



\subsection{Continuous line segment erosion} 
\codeinp{03_cts_erosion_quick}



\pagevertcentre{
\section{Spectra normalisation}
\codetab{
\codeentry{Empirical quantile normalisation}{04\_quant\_norm.R}{quant\_norm()}
\codeentry{Pairwise spectra MA normalisation}{05\_ma\_adj.R}{ma\_adj()}
}
}


\subsection{Empirical quantile normalisation} 
\codeinp{04_quant_norm}
	


\subsection{Pairwise spectra MA normalisation} 
\codeinp{05_ma_adj}
	




\pagevertcentre{
\section{Peak alignment} \label{pacodes}
\codetab{
\codeentry{Calculate $W$ matrix for an $N$- and $M$-alignment}{06\_create\_w.R}{w\_matrix()}
\codeentry{Dendrogram peak alignment}{07\_dendro\_peak\_align.R}{dendro\_peak\_align()}
}
}

\subsection{Calculate $W$ matrix for an $N$- and $M$-alignment} 
\codeinp{06_create_w}


\subsection{Dendrogram peak alignment} 
\codeinp{07_dendro_peak_align}




\pagevertcentre{
\section{Surrogate variable analysis} \label{sva}
\codetab{
\codeentry{Get SVA adjusted expression matrix}{08\_do\_sva.R}{do\_sva()}
}
}


\subsection{Get SVA adjusted expression matrix}
Please note the function \texttt{getH()} (line 47 below) is the code available in the \texttt{DanteR} package to determine the number of significant surrogate variables. The function \texttt{mulReg(Y,X)} performs sequential linear regressions on the columns of the input \texttt{Y} using a fixed effects design matrix \texttt{X}. \texttt{mulReg()} returns a list containing the following vectors and matrices: \texttt{RES}$_{n \times P}$, residual matrix after \texttt{Y} has been regressed; \texttt{BETA}$_{d \times P}$, matrix of the regression coefficients, $P$ columns for each regression; \texttt{TVALS}$_{d \times P}$, the corresponding $t$-statistics; \texttt{PVALS}$_{d \times P}$ the corresponding $p$-values of \texttt{TVALS}; \texttt{FPVALS}$_{P\times 1}$, $p$-value for each linear regression corresponding to the null model $F$-statistic.\\
\codeinp{08_do_sva}


\pagevertcentre{
\section{Pairwise fusion linear discriminant analysis} 
\codetab{
\codeentry{Create a PFDA object}{09\_create\_pfda\_obj.R}{create\_pfda\_obj()}
\codeentry{Predict class for new data and a PFDA object}{10\_pfda\_predict.R}{pfda\_predict()}
}
}



\subsection{Create a PFDA object}
\codeinp{09_create_pfda_obj}


\subsection{Predict class for new data and a PFDA object}
\codeinp{10_pfda_predict}



\pagevertcentre{
\section{Pareto Fronts for variable ranking} 
\codetab{
\codeentry{Calculate dominating features}{11\_dom\_feat.c}{dom\_feat()}
\codeentry{Pareto Front wrapper functions}{12\_pareto\_fronts.R}{pareto\_ranking()}
}
}


\subsection{Calculate dominating features} 
Below is the core of the Pareto Front code, finding features that are the dominated as per the definition. Written in {\tt C} to be compiled to a {\tt .so} file (or \texttt{.dll} on Windows operating systems) that in turn can be loaded into {\tt R}.\\
\codeinp{11_dom_feat}


\subsection{Pareto Front wrapper functions} 
\begin{Verbatim}[commandchars=\\\{\},codes={\catcode`\$=3\catcode`\^=7\catcode`\_=8},gobble=0,numbers=left,fontfamily=fvm,fontshape=n,fontsize=\footnotesize,tabsize=2]
\PY{k+kn}{library}\PY{p}{(}animation\PY{p}{)}
\PY{k+kp}{dyn.load}\PY{p}{(}\PY{l+s}{\PYZdq{}}\PY{l+s}{domfeat.so\PYZdq{}}\PY{p}{)}

\PY{c+c1}{\PYZsh{}\PYZsh{}\PYZsh{}\PYZsh{}\PYZsh{}\PYZsh{}\PYZsh{}\PYZsh{}\PYZsh{}\PYZsh{}\PYZsh{}\PYZsh{}\PYZsh{}\PYZsh{}\PYZsh{}\PYZsh{}\PYZsh{}\PYZsh{}\PYZsh{}\PYZsh{}\PYZsh{}\PYZsh{}\PYZsh{}\PYZsh{}\PYZsh{}\PYZsh{}\PYZsh{}\PYZsh{}\PYZsh{}\PYZsh{}\PYZsh{}\PYZsh{}\PYZsh{}\PYZsh{}\PYZsh{}\PYZsh{}\PYZsh{}\PYZsh{}\PYZsh{}\PYZsh{}\PYZsh{}\PYZsh{}}
\PY{c+c1}{\PYZsh{}\PYZsh{}\PYZsh{}\PYZsh{}\PYZsh{}\PYZsh{}\PYZsh{}\PYZsh{}\PYZsh{}\PYZsh{}\PYZsh{}\PYZsh{}\PYZsh{}\PYZsh{}\PYZsh{}\PYZsh{}\PYZsh{}\PYZsh{}\PYZsh{}\PYZsh{}\PYZsh{}\PYZsh{}\PYZsh{}\PYZsh{}\PYZsh{}\PYZsh{}\PYZsh{}\PYZsh{}\PYZsh{}\PYZsh{}\PYZsh{}\PYZsh{}\PYZsh{}\PYZsh{}\PYZsh{}\PYZsh{}\PYZsh{}\PYZsh{}\PYZsh{}\PYZsh{}\PYZsh{}\PYZsh{}}
\PY{c+c1}{\PYZsh{}}
\PY{c+c1}{\PYZsh{} Pairwise case of Pareto Fronts}
\PY{c+c1}{\PYZsh{}     obj is the $2 \times n$ matrix of the two vectors of length $n$ for  }
\PY{c+c1}{\PYZsh{}			the features/observations of the 2 criteria/objective functions}
\PY{c+c1}{\PYZsh{}     istomin is a boolean vector of whether the criteria obj$_1$, obj$_2$  }
\PY{c+c1}{\PYZsh{}			are to be minimised (=TRUE), respectively}
\PY{c+c1}{\PYZsh{}}
\PY{c+c1}{\PYZsh{}\PYZsh{}\PYZsh{}\PYZsh{}\PYZsh{}\PYZsh{}\PYZsh{}\PYZsh{}\PYZsh{}\PYZsh{}\PYZsh{}\PYZsh{}\PYZsh{}\PYZsh{}\PYZsh{}\PYZsh{}\PYZsh{}\PYZsh{}\PYZsh{}\PYZsh{}\PYZsh{}\PYZsh{}\PYZsh{}\PYZsh{}\PYZsh{}\PYZsh{}\PYZsh{}\PYZsh{}\PYZsh{}\PYZsh{}\PYZsh{}\PYZsh{}\PYZsh{}\PYZsh{}\PYZsh{}\PYZsh{}\PYZsh{}\PYZsh{}\PYZsh{}\PYZsh{}\PYZsh{}\PYZsh{}}
\PY{c+c1}{\PYZsh{}\PYZsh{}\PYZsh{}\PYZsh{}\PYZsh{}\PYZsh{}\PYZsh{}\PYZsh{}\PYZsh{}\PYZsh{}\PYZsh{}\PYZsh{}\PYZsh{}\PYZsh{}\PYZsh{}\PYZsh{}\PYZsh{}\PYZsh{}\PYZsh{}\PYZsh{}\PYZsh{}\PYZsh{}\PYZsh{}\PYZsh{}\PYZsh{}\PYZsh{}\PYZsh{}\PYZsh{}\PYZsh{}\PYZsh{}\PYZsh{}\PYZsh{}\PYZsh{}\PYZsh{}\PYZsh{}\PYZsh{}\PYZsh{}\PYZsh{}\PYZsh{}\PYZsh{}\PYZsh{}\PYZsh{}}

\PY{c+c1}{\PYZsh{}}
\PY{c+c1}{\PYZsh{} This function returns a vector of \PYZsq{}dominated\PYZsq{} observations (Boolean, }
\PY{c+c1}{\PYZsh{} length $n$ vector) FALSE=Pareto front, TRUE=dominated observation}
\PY{c+c1}{\PYZsh{}}

domFeaturesPW\PY{o}{\PYZlt{}\PYZhy{}}\PY{k+kr}{function}\PY{p}{(}obj\PY{p}{,}istomin\PY{p}{)}
\PY{p}{\PYZob{}}
	\PY{c+c1}{\PYZsh{}if to be minimised then just make negative and maximise}
	obj\PY{p}{[}\PY{p}{,}istomin\PY{p}{]}\PY{o}{\PYZlt{}\PYZhy{}} \PY{o}{\PYZhy{}}obj\PY{p}{[}\PY{p}{,}istomin\PY{p}{]} 
	n\PY{o}{\PYZlt{}\PYZhy{}}\PY{k+kp}{as.integer}\PY{p}{(}\PY{k+kp}{nrow}\PY{p}{(}obj\PY{p}{)}\PY{p}{)}
	obj1\PY{o}{\PYZlt{}\PYZhy{}}\PY{k+kp}{as.double}\PY{p}{(}obj\PY{p}{[}\PY{p}{,}\PY{l+m}{1}\PY{p}{]}\PY{p}{)}
	obj2\PY{o}{\PYZlt{}\PYZhy{}}\PY{k+kp}{as.double}\PY{p}{(}obj\PY{p}{[}\PY{p}{,}\PY{l+m}{2}\PY{p}{]}\PY{p}{)}
	domvec\PY{o}{\PYZlt{}\PYZhy{}}\PY{k+kp}{as.integer}\PY{p}{(}\PY{k+kp}{rep}\PY{p}{(}\PY{l+m}{0}\PY{p}{,}n\PY{p}{)}\PY{p}{)} \PY{c+c1}{\PYZsh{}output vector}
	
	\PY{k+kr}{return}\PY{p}{(}\PY{k+kp}{as.logical}\PY{p}{(}\PY{l+m}{.}C\PY{p}{(}\PY{l+s}{\PYZdq{}}\PY{l+s}{domfeat\PYZdq{}}\PY{p}{,}n\PY{p}{,}obj1\PY{p}{,}obj2\PY{p}{,}domvec\PY{p}{)}\PY{p}{[[}\PY{l+m}{4}\PY{p}{]]}\PY{p}{)}\PY{p}{)}
\PY{p}{\PYZcb{}}
	
\PY{c+c1}{\PYZsh{}\PYZsh{}\PYZsh{}\PYZsh{}\PYZsh{}\PYZsh{}\PYZsh{}\PYZsh{}\PYZsh{}\PYZsh{}\PYZsh{}\PYZsh{}\PYZsh{}\PYZsh{}\PYZsh{}\PYZsh{}\PYZsh{}\PYZsh{}\PYZsh{}\PYZsh{}\PYZsh{}\PYZsh{}\PYZsh{}\PYZsh{}\PYZsh{}\PYZsh{}\PYZsh{}\PYZsh{}\PYZsh{}\PYZsh{}\PYZsh{}\PYZsh{}\PYZsh{}\PYZsh{}\PYZsh{}\PYZsh{}\PYZsh{}\PYZsh{}\PYZsh{}\PYZsh{}\PYZsh{}\PYZsh{}}
\PY{c+c1}{\PYZsh{}\PYZsh{}\PYZsh{}\PYZsh{}\PYZsh{}\PYZsh{}\PYZsh{}\PYZsh{}\PYZsh{}\PYZsh{}\PYZsh{}\PYZsh{}\PYZsh{}\PYZsh{}\PYZsh{}\PYZsh{}\PYZsh{}\PYZsh{}\PYZsh{}\PYZsh{}\PYZsh{}\PYZsh{}\PYZsh{}\PYZsh{}\PYZsh{}\PYZsh{}\PYZsh{}\PYZsh{}\PYZsh{}\PYZsh{}\PYZsh{}\PYZsh{}\PYZsh{}\PYZsh{}\PYZsh{}\PYZsh{}\PYZsh{}\PYZsh{}\PYZsh{}\PYZsh{}\PYZsh{}\PYZsh{}}
\PY{c+c1}{\PYZsh{}}
\PY{c+c1}{\PYZsh{} General case}
\PY{c+c1}{\PYZsh{}     objmatrix is $n \times m$ matrix. $n$ features/observations and }
\PY{c+c1}{\PYZsh{}     $m$ criteria/objective functions istominvec is a boolean vec}
\PY{c+c1}{\PYZsh{}     of length $m$ to say whether the criteria are to be minimised}
\PY{c+c1}{\PYZsh{}}
\PY{c+c1}{\PYZsh{}\PYZsh{}\PYZsh{}\PYZsh{}\PYZsh{}\PYZsh{}\PYZsh{}\PYZsh{}\PYZsh{}\PYZsh{}\PYZsh{}\PYZsh{}\PYZsh{}\PYZsh{}\PYZsh{}\PYZsh{}\PYZsh{}\PYZsh{}\PYZsh{}\PYZsh{}\PYZsh{}\PYZsh{}\PYZsh{}\PYZsh{}\PYZsh{}\PYZsh{}\PYZsh{}\PYZsh{}\PYZsh{}\PYZsh{}\PYZsh{}\PYZsh{}\PYZsh{}\PYZsh{}\PYZsh{}\PYZsh{}\PYZsh{}\PYZsh{}\PYZsh{}\PYZsh{}\PYZsh{}\PYZsh{}}
\PY{c+c1}{\PYZsh{}\PYZsh{}\PYZsh{}\PYZsh{}\PYZsh{}\PYZsh{}\PYZsh{}\PYZsh{}\PYZsh{}\PYZsh{}\PYZsh{}\PYZsh{}\PYZsh{}\PYZsh{}\PYZsh{}\PYZsh{}\PYZsh{}\PYZsh{}\PYZsh{}\PYZsh{}\PYZsh{}\PYZsh{}\PYZsh{}\PYZsh{}\PYZsh{}\PYZsh{}\PYZsh{}\PYZsh{}\PYZsh{}\PYZsh{}\PYZsh{}\PYZsh{}\PYZsh{}\PYZsh{}\PYZsh{}\PYZsh{}\PYZsh{}\PYZsh{}\PYZsh{}\PYZsh{}\PYZsh{}\PYZsh{}}

\PY{c+c1}{\PYZsh{}}
\PY{c+c1}{\PYZsh{} This function returns a vector of \PYZsq{}dominated\PYZsq{} observations (Boolean,}
\PY{c+c1}{\PYZsh{} length $n$ vector) FALSE=Pareto front, TRUE=dominated observation}
\PY{c+c1}{\PYZsh{} same as pairwise but the input can take more than two objective functions}
\PY{c+c1}{\PYZsh{}}

domFeatures\PY{o}{\PYZlt{}\PYZhy{}}\PY{k+kr}{function}\PY{p}{(}objmatrix\PY{p}{,}istominvec\PY{p}{)}
\PY{p}{\PYZob{}}
	n\PY{o}{\PYZlt{}\PYZhy{}}\PY{k+kp}{as.integer}\PY{p}{(}\PY{k+kp}{nrow}\PY{p}{(}objmatrix\PY{p}{)}\PY{p}{)}
	m\PY{o}{\PYZlt{}\PYZhy{}}\PY{k+kp}{ncol}\PY{p}{(}objmatrix\PY{p}{)}
	objmatrix\PY{p}{[}\PY{p}{,}istominvec\PY{p}{]}\PY{o}{\PYZlt{}\PYZhy{}} \PY{o}{\PYZhy{}}objmatrix\PY{p}{[}\PY{p}{,}istominvec\PY{p}{]}
	vecdomvec\PY{o}{\PYZlt{}\PYZhy{}}\PY{k+kp}{rep}\PY{p}{(}\PY{l+m}{1}\PY{p}{,}n\PY{p}{)}
	indxs\PY{o}{\PYZlt{}\PYZhy{}}combn\PY{p}{(}m\PY{p}{,}\PY{l+m}{2}\PY{p}{)}
	nm\PY{o}{\PYZlt{}\PYZhy{}}\PY{k+kp}{ncol}\PY{p}{(}indxs\PY{p}{)}
	i\PY{o}{\PYZlt{}\PYZhy{}}\PY{l+m}{0}
	\PY{k+kr}{while}\PY{p}{(}i\PY{o}{\PYZlt{}}nm\PY{p}{)}
	\PY{p}{\PYZob{}}
		i\PY{o}{\PYZlt{}\PYZhy{}}i\PY{l+m}{+1}
		\PY{c+c1}{\PYZsh{} call pairwise function, take the intersection of previous }
		\PY{c+c1}{\PYZsh{} dominated observations remembering the intersection(s)}
		\PY{c+c1}{\PYZsh{} of dominated in the same as unions(s) of Pareto fronts}
		vecdomvec\PY{o}{\PYZlt{}\PYZhy{}}vecdomvec \PY{o}{*} \PY{l+m}{.}C\PY{p}{(}\PY{l+s}{\PYZdq{}}\PY{l+s}{domfeat\PYZdq{}}
							\PY{p}{,}n
							\PY{p}{,}\PY{k+kp}{as.double}\PY{p}{(}objmatrix\PY{p}{[}\PY{p}{,}indxs\PY{p}{[}\PY{l+m}{1}\PY{p}{,}i\PY{p}{]]}\PY{p}{)}
							\PY{p}{,}\PY{k+kp}{as.double}\PY{p}{(}objmatrix\PY{p}{[}\PY{p}{,}indxs\PY{p}{[}\PY{l+m}{2}\PY{p}{,}i\PY{p}{]]}\PY{p}{)}
							\PY{p}{,}\PY{k+kp}{as.integer}\PY{p}{(}\PY{k+kp}{rep}\PY{p}{(}\PY{l+m}{0}\PY{p}{,}n\PY{p}{)}\PY{p}{)}\PY{p}{)}\PY{p}{[[}\PY{l+m}{4}\PY{p}{]]}
	\PY{p}{\PYZcb{}}
	\PY{k+kr}{return}\PY{p}{(}\PY{k+kp}{as.logical}\PY{p}{(}vecdomvec\PY{p}{)}\PY{p}{)}
\PY{p}{\PYZcb{}}

\PY{c+c1}{\PYZsh{}\PYZsh{}\PYZsh{}\PYZsh{}\PYZsh{}\PYZsh{}\PYZsh{}\PYZsh{}\PYZsh{}\PYZsh{}\PYZsh{}\PYZsh{}\PYZsh{}\PYZsh{}\PYZsh{}\PYZsh{}\PYZsh{}\PYZsh{}\PYZsh{}\PYZsh{}\PYZsh{}\PYZsh{}\PYZsh{}\PYZsh{}\PYZsh{}\PYZsh{}\PYZsh{}\PYZsh{}\PYZsh{}\PYZsh{}\PYZsh{}\PYZsh{}\PYZsh{}\PYZsh{}\PYZsh{}\PYZsh{}\PYZsh{}\PYZsh{}\PYZsh{}\PYZsh{}\PYZsh{}\PYZsh{}}
\PY{c+c1}{\PYZsh{}\PYZsh{}\PYZsh{}\PYZsh{}\PYZsh{}\PYZsh{}\PYZsh{}\PYZsh{}\PYZsh{}\PYZsh{}\PYZsh{}\PYZsh{}\PYZsh{}\PYZsh{}\PYZsh{}\PYZsh{}\PYZsh{}\PYZsh{}\PYZsh{}\PYZsh{}\PYZsh{}\PYZsh{}\PYZsh{}\PYZsh{}\PYZsh{}\PYZsh{}\PYZsh{}\PYZsh{}\PYZsh{}\PYZsh{}\PYZsh{}\PYZsh{}\PYZsh{}\PYZsh{}\PYZsh{}\PYZsh{}\PYZsh{}\PYZsh{}\PYZsh{}\PYZsh{}\PYZsh{}\PYZsh{}}
\PY{c+c1}{\PYZsh{}}
\PY{c+c1}{\PYZsh{} Sucessive Pareto Fronts}
\PY{c+c1}{\PYZsh{}     noFronts is the \PYZsh{} of pareto fronts required}
\PY{c+c1}{\PYZsh{}}
\PY{c+c1}{\PYZsh{}     fn returns a vector of length $n$}
\PY{c+c1}{\PYZsh{}           each element is labelled the pareto front \PYZsh{},  }
\PY{c+c1}{\PYZsh{}           0 is dominated even after noFronts found}
\PY{c+c1}{\PYZsh{}}
\PY{c+c1}{\PYZsh{}\PYZsh{}\PYZsh{}\PYZsh{}\PYZsh{}\PYZsh{}\PYZsh{}\PYZsh{}\PYZsh{}\PYZsh{}\PYZsh{}\PYZsh{}\PYZsh{}\PYZsh{}\PYZsh{}\PYZsh{}\PYZsh{}\PYZsh{}\PYZsh{}\PYZsh{}\PYZsh{}\PYZsh{}\PYZsh{}\PYZsh{}\PYZsh{}\PYZsh{}\PYZsh{}\PYZsh{}\PYZsh{}\PYZsh{}\PYZsh{}\PYZsh{}\PYZsh{}\PYZsh{}\PYZsh{}\PYZsh{}\PYZsh{}\PYZsh{}\PYZsh{}\PYZsh{}\PYZsh{}\PYZsh{}}
\PY{c+c1}{\PYZsh{}\PYZsh{}\PYZsh{}\PYZsh{}\PYZsh{}\PYZsh{}\PYZsh{}\PYZsh{}\PYZsh{}\PYZsh{}\PYZsh{}\PYZsh{}\PYZsh{}\PYZsh{}\PYZsh{}\PYZsh{}\PYZsh{}\PYZsh{}\PYZsh{}\PYZsh{}\PYZsh{}\PYZsh{}\PYZsh{}\PYZsh{}\PYZsh{}\PYZsh{}\PYZsh{}\PYZsh{}\PYZsh{}\PYZsh{}\PYZsh{}\PYZsh{}\PYZsh{}\PYZsh{}\PYZsh{}\PYZsh{}\PYZsh{}\PYZsh{}\PYZsh{}\PYZsh{}\PYZsh{}\PYZsh{}}

paretoFronts\PY{o}{\PYZlt{}\PYZhy{}}\PY{k+kr}{function}\PY{p}{(}noFronts\PY{p}{,}objmatrix\PY{p}{,}istominvec\PY{p}{)}
\PY{p}{\PYZob{}}
	objmatrix\PY{p}{[}\PY{p}{,}istominvec\PY{p}{]}\PY{o}{\PYZlt{}\PYZhy{}} \PY{o}{\PYZhy{}}objmatrix\PY{p}{[}\PY{p}{,}istominvec\PY{p}{]}
	n\PY{o}{\PYZlt{}\PYZhy{}}\PY{k+kp}{as.integer}\PY{p}{(}\PY{k+kp}{nrow}\PY{p}{(}objmatrix\PY{p}{)}\PY{p}{)}
	m\PY{o}{\PYZlt{}\PYZhy{}}\PY{k+kp}{ncol}\PY{p}{(}objmatrix\PY{p}{)}
	pfvec\PY{o}{\PYZlt{}\PYZhy{}}\PY{k+kp}{rep}\PY{p}{(}\PY{l+m}{0}\PY{p}{,}n\PY{p}{)} \PY{c+c1}{\PYZsh{}output vector}
	\PY{c+c1}{\PYZsh{}once a front is found we need to set the correponding values to $\infty$ or }
	\PY{c+c1}{\PYZsh{} $-\infty$ so they won\PYZsq{}t be chosen again}
	\PY{c+c1}{\PYZsh{}try: as.numeric(c(TRUE,FALSE,TRUE))*2\PYZhy{}1 to see what the next line is doing}
	\PY{c+c1}{\PYZsh{}if Min then set 1, ifMax then set \PYZhy{}1 (the sign of the Inf if we find front }
	\PY{c+c1}{\PYZsh{}	 and have to put to a value)}
	ourInfs\PY{o}{\PYZlt{}\PYZhy{}}\PY{k+kp}{min}\PY{p}{(}objmatrix\PY{p}{)}\PY{l+m}{\PYZhy{}1}
	allFrontsFound\PY{o}{\PYZlt{}\PYZhy{}}\PY{k+kc}{FALSE}
	
	i\PY{o}{\PYZlt{}\PYZhy{}}\PY{l+m}{0}
	\PY{k+kr}{while}\PY{p}{(}i\PY{o}{\PYZlt{}}noFronts \PY{o}{\PYZam{}\PYZam{}} \PY{o}{!}allFrontsFound\PY{p}{)} \PY{c+c1}{\PYZsh{}go thru all fronts required}
	\PY{p}{\PYZob{}}
		i\PY{o}{\PYZlt{}\PYZhy{}}i\PY{l+m}{+1}
		df\PY{o}{\PYZlt{}\PYZhy{}}domFeatures\PY{p}{(}objmatrix\PY{p}{,}\PY{k+kp}{rep}\PY{p}{(}\PY{k+kc}{FALSE}\PY{p}{,}m\PY{p}{)}\PY{p}{)} \PY{c+c1}{\PYZsh{}general m obj vectors function}
		\PY{c+c1}{\PYZsh{} pf.i are the indexs of the output vector that need to be updated }
		\PY{c+c1}{\PYZsh{} with the pareto front number}
		pf.i\PY{o}{\PYZlt{}\PYZhy{}}\PY{p}{(}\PY{o}{!}df\PY{p}{)} \PY{o}{\PYZam{}} \PY{p}{(}pfvec\PY{o}{\PYZlt{}}\PY{l+m}{1}\PY{p}{)} 
		pfvec\PY{p}{[}pf.i\PY{p}{]}\PY{o}{\PYZlt{}\PYZhy{}}i
		\PY{c+c1}{\PYZsh{} re\PYZhy{}assign values were pareto front found}
		objmatrix\PY{p}{[}pf.i\PY{p}{,}\PY{p}{]}\PY{o}{\PYZlt{}\PYZhy{}}ourInfs
		\PY{k+kr}{if}\PY{p}{(}\PY{k+kp}{all}\PY{p}{(}pfvec\PY{o}{\PYZgt{}}\PY{l+m}{0}\PY{p}{)}\PY{p}{)} allFrontsFound\PY{o}{\PYZlt{}\PYZhy{}}\PY{k+kc}{TRUE}
	\PY{p}{\PYZcb{}}
	\PY{k+kr}{return}\PY{p}{(}pfvec\PY{p}{)}	
\PY{p}{\PYZcb{}}

\PY{c+c1}{\PYZsh{}\PYZsh{}\PYZsh{}\PYZsh{}\PYZsh{}\PYZsh{}\PYZsh{}\PYZsh{}\PYZsh{}\PYZsh{}\PYZsh{}\PYZsh{}\PYZsh{}\PYZsh{}\PYZsh{}\PYZsh{}\PYZsh{}\PYZsh{}\PYZsh{}\PYZsh{}\PYZsh{}\PYZsh{}\PYZsh{}\PYZsh{}\PYZsh{}\PYZsh{}\PYZsh{}\PYZsh{}\PYZsh{}\PYZsh{}\PYZsh{}\PYZsh{}\PYZsh{}\PYZsh{}\PYZsh{}\PYZsh{}\PYZsh{}\PYZsh{}\PYZsh{}\PYZsh{}\PYZsh{}\PYZsh{}}
\PY{c+c1}{\PYZsh{}\PYZsh{}\PYZsh{}\PYZsh{}\PYZsh{}\PYZsh{}\PYZsh{}\PYZsh{}\PYZsh{}\PYZsh{}\PYZsh{}\PYZsh{}\PYZsh{}\PYZsh{}\PYZsh{}\PYZsh{}\PYZsh{}\PYZsh{}\PYZsh{}\PYZsh{}\PYZsh{}\PYZsh{}\PYZsh{}\PYZsh{}\PYZsh{}\PYZsh{}\PYZsh{}\PYZsh{}\PYZsh{}\PYZsh{}\PYZsh{}\PYZsh{}\PYZsh{}\PYZsh{}\PYZsh{}\PYZsh{}\PYZsh{}\PYZsh{}\PYZsh{}\PYZsh{}\PYZsh{}\PYZsh{}}
\PY{c+c1}{\PYZsh{}}
\PY{c+c1}{\PYZsh{} Leave\PYZhy{}one\PYZhy{}out/k\PYZhy{}fold feature ranking}
\PY{c+c1}{\PYZsh{}}
\PY{c+c1}{\PYZsh{} returns a vector of length $n$ with values $\in (0,1]$ for feature importance		}
\PY{c+c1}{\PYZsh{}}
\PY{c+c1}{\PYZsh{}\PYZsh{}\PYZsh{}\PYZsh{}\PYZsh{}\PYZsh{}\PYZsh{}\PYZsh{}\PYZsh{}\PYZsh{}\PYZsh{}\PYZsh{}\PYZsh{}\PYZsh{}\PYZsh{}\PYZsh{}\PYZsh{}\PYZsh{}\PYZsh{}\PYZsh{}\PYZsh{}\PYZsh{}\PYZsh{}\PYZsh{}\PYZsh{}\PYZsh{}\PYZsh{}\PYZsh{}\PYZsh{}\PYZsh{}\PYZsh{}\PYZsh{}\PYZsh{}\PYZsh{}\PYZsh{}\PYZsh{}\PYZsh{}\PYZsh{}\PYZsh{}\PYZsh{}\PYZsh{}\PYZsh{}}
\PY{c+c1}{\PYZsh{}\PYZsh{}\PYZsh{}\PYZsh{}\PYZsh{}\PYZsh{}\PYZsh{}\PYZsh{}\PYZsh{}\PYZsh{}\PYZsh{}\PYZsh{}\PYZsh{}\PYZsh{}\PYZsh{}\PYZsh{}\PYZsh{}\PYZsh{}\PYZsh{}\PYZsh{}\PYZsh{}\PYZsh{}\PYZsh{}\PYZsh{}\PYZsh{}\PYZsh{}\PYZsh{}\PYZsh{}\PYZsh{}\PYZsh{}\PYZsh{}\PYZsh{}\PYZsh{}\PYZsh{}\PYZsh{}\PYZsh{}\PYZsh{}\PYZsh{}\PYZsh{}\PYZsh{}\PYZsh{}\PYZsh{}}

\PY{c+c1}{\PYZsh{}}
\PY{c+c1}{\PYZsh{} Same inputs of previous functions, with folds (aka $k$\PYZhy{}fold cross  }
\PY{c+c1}{\PYZsh{} validation) and reps is the number of times we re\PYZhy{}do the cross }
\PY{c+c1}{\PYZsh{} validation fold=1 or the length of the input (i.e. n) creates }
\PY{c+c1}{\PYZsh{} leave\PYZhy{}one\PYZhy{}out cross validation}

paretoRanking\PY{o}{\PYZlt{}\PYZhy{}}\PY{k+kr}{function}\PY{p}{(}objmatrix\PY{p}{,}istominvec\PY{p}{,}noFronts\PY{o}{=}\PY{l+m}{20}\PY{p}{,}folds\PY{o}{=}\PY{l+m}{1}\PY{p}{,}reps\PY{o}{=}\PY{l+m}{5}\PY{p}{)}
\PY{p}{\PYZob{}}
	objmatrix\PY{p}{[}\PY{p}{,}istominvec\PY{p}{]}\PY{o}{\PYZlt{}\PYZhy{}} \PY{o}{\PYZhy{}}objmatrix\PY{p}{[}\PY{p}{,}istominvec\PY{p}{]}
	m\PY{o}{\PYZlt{}\PYZhy{}}\PY{k+kp}{ncol}\PY{p}{(}objmatrix\PY{p}{)}
	n\PY{o}{\PYZlt{}\PYZhy{}}\PY{k+kp}{nrow}\PY{p}{(}objmatrix\PY{p}{)}
	pfmetric\PY{o}{\PYZlt{}\PYZhy{}}\PY{k+kp}{rep}\PY{p}{(}\PY{l+m}{0}\PY{p}{,}n\PY{p}{)} \PY{c+c1}{\PYZsh{}output vector}
	nfolds\PY{o}{\PYZlt{}\PYZhy{}}n
	\PY{k+kr}{if}\PY{p}{(}folds\PY{o}{\PYZgt{}}\PY{l+m}{1}\PY{p}{)} nfolds\PY{o}{\PYZlt{}\PYZhy{}}folds
	\PY{k+kr}{if}\PY{p}{(}nfolds\PY{o}{==}n\PY{p}{)} reps\PY{o}{\PYZlt{}\PYZhy{}}\PY{l+m}{1}
	
	blocks\PY{o}{\PYZlt{}\PYZhy{}}kfcv\PY{p}{(}nfolds\PY{p}{,}n\PY{p}{)}
	block.nos\PY{o}{\PYZlt{}\PYZhy{}}\PY{k+kp}{rep}\PY{p}{(}\PY{l+m}{1}\PY{o}{:}nfolds\PY{p}{,}blocks\PY{p}{)}
	
	\PY{k+kr}{for}\PY{p}{(}r \PY{k+kr}{in} \PY{l+m}{1}\PY{o}{:}reps\PY{p}{)}
	\PY{p}{\PYZob{}}
		indxs\PY{o}{\PYZlt{}\PYZhy{}}\PY{k+kp}{sample}\PY{p}{(}\PY{l+m}{1}\PY{o}{:}n\PY{p}{)} \PY{c+c1}{\PYZsh{}fresh randomisation each repetition}
		k.f.mat\PY{o}{\PYZlt{}\PYZhy{}}\PY{k+kp}{cbind}\PY{p}{(}indxs\PY{p}{,}block.nos\PY{p}{)} \PY{c+c1}{\PYZsh{} create the fold \PYZsq{}blocks\PYZsq{} of data}
		\PY{k+kr}{for}\PY{p}{(}i \PY{k+kr}{in} \PY{l+m}{1}\PY{o}{:}nfolds\PY{p}{)}
		\PY{p}{\PYZob{}}
			rows\PY{o}{\PYZlt{}\PYZhy{}}k.f.mat\PY{p}{[}k.f.mat\PY{p}{[}\PY{p}{,}\PY{l+m}{2}\PY{p}{]}\PY{o}{==}i\PY{p}{,}\PY{l+m}{1}\PY{p}{]} \PY{c+c1}{\PYZsh{} find the ith fold to leave out}
			\PY{c+c1}{\PYZsh{} call general function with ith fold removed}
			calcfronts\PY{o}{\PYZlt{}\PYZhy{}}paretoFronts\PY{p}{(}noFronts\PY{p}{,}objmatrix\PY{p}{[}\PY{o}{\PYZhy{}}rows\PY{p}{,}\PY{p}{]}\PY{p}{,}\PY{k+kp}{rep}\PY{p}{(}\PY{k+kc}{FALSE}\PY{p}{,}m\PY{p}{)}\PY{p}{)}
			\PY{c+c1}{\PYZsh{} which are non\PYZhy{}dominated}
			whichnondom\PY{o}{\PYZlt{}\PYZhy{}}calcfronts\PY{o}{\PYZgt{}}\PY{l+m}{0}
			\PY{c+c1}{\PYZsh{} if you are ont the first front you get 1, second=1/2, third=1/3,}
			\PY{c+c1}{\PYZsh{} ..., jth=1/j, else 0}
			pfmetric\PY{p}{[}\PY{o}{\PYZhy{}}rows\PY{p}{]}\PY{p}{[}whichnondom\PY{p}{]}\PY{o}{\PYZlt{}\PYZhy{}}
					pfmetric\PY{p}{[}\PY{o}{\PYZhy{}}rows\PY{p}{]}\PY{p}{[}whichnondom\PY{p}{]}\PY{l+m}{+1}\PY{o}{/}calcfronts\PY{p}{[}whichnondom\PY{p}{]}
		\PY{p}{\PYZcb{}}
	\PY{p}{\PYZcb{}}
	\PY{c+c1}{\PYZsh{}now divide by maximum posible value i.e. (nfolds\PYZhy{}1)*reps so output in [0,1]}
	pfmetric\PY{o}{\PYZlt{}\PYZhy{}}pfmetric\PY{o}{/}\PY{p}{(}\PY{p}{(}nfolds\PY{l+m}{\PYZhy{}1}\PY{p}{)}\PY{o}{*}reps\PY{p}{)} 
	\PY{k+kr}{return}\PY{p}{(}pfmetric\PY{p}{)}
\PY{p}{\PYZcb{}}

\PY{c+c1}{\PYZsh{}\PYZsh{}\PYZsh{}\PYZsh{}\PYZsh{}\PYZsh{}\PYZsh{}\PYZsh{}\PYZsh{}\PYZsh{}\PYZsh{}\PYZsh{}\PYZsh{}\PYZsh{}\PYZsh{}\PYZsh{}\PYZsh{}\PYZsh{}\PYZsh{}\PYZsh{}\PYZsh{}\PYZsh{}\PYZsh{}\PYZsh{}\PYZsh{}\PYZsh{}\PYZsh{}\PYZsh{}\PYZsh{}\PYZsh{}\PYZsh{}\PYZsh{}\PYZsh{}\PYZsh{}\PYZsh{}\PYZsh{}\PYZsh{}\PYZsh{}\PYZsh{}\PYZsh{}\PYZsh{}\PYZsh{}\PYZsh{}\PYZsh{}\PYZsh{}\PYZsh{}\PYZsh{}\PYZsh{}\PYZsh{}\PYZsh{}\PYZsh{}\PYZsh{}\PYZsh{}\PYZsh{}\PYZsh{}\PYZsh{}\PYZsh{}\PYZsh{}\PYZsh{}\PYZsh{}\PYZsh{}\PYZsh{}\PYZsh{}\PYZsh{}\PYZsh{}\PYZsh{}\PYZsh{}\PYZsh{}\PYZsh{}\PYZsh{}\PYZsh{}\PYZsh{}\PYZsh{}\PYZsh{}\PYZsh{}}
\PY{c+c1}{\PYZsh{} Below are three metrics that can possibly describe the value of      }
\PY{c+c1}{\PYZsh{}   variables/fetures to discriminate betwwen classes                                                             \PYZsh{}\PYZsh{}\PYZsh{}\PYZsh{}}
\PY{c+c1}{\PYZsh{}\PYZsh{}\PYZsh{}\PYZsh{}\PYZsh{}\PYZsh{}\PYZsh{}\PYZsh{}\PYZsh{}\PYZsh{}\PYZsh{}\PYZsh{}\PYZsh{}\PYZsh{}\PYZsh{}\PYZsh{}\PYZsh{}\PYZsh{}\PYZsh{}\PYZsh{}\PYZsh{}\PYZsh{}\PYZsh{}\PYZsh{}\PYZsh{}\PYZsh{}\PYZsh{}\PYZsh{}\PYZsh{}\PYZsh{}\PYZsh{}\PYZsh{}\PYZsh{}\PYZsh{}\PYZsh{}\PYZsh{}\PYZsh{}\PYZsh{}\PYZsh{}\PYZsh{}\PYZsh{}\PYZsh{}\PYZsh{}\PYZsh{}\PYZsh{}\PYZsh{}\PYZsh{}\PYZsh{}\PYZsh{}\PYZsh{}\PYZsh{}\PYZsh{}\PYZsh{}\PYZsh{}\PYZsh{}\PYZsh{}\PYZsh{}\PYZsh{}\PYZsh{}\PYZsh{}\PYZsh{}\PYZsh{}\PYZsh{}\PYZsh{}\PYZsh{}\PYZsh{}\PYZsh{}\PYZsh{}\PYZsh{}\PYZsh{}\PYZsh{}\PYZsh{}\PYZsh{}\PYZsh{}\PYZsh{}}

\PY{c+c1}{\PYZsh{} minIntraClassVar(): find the minimum WITHIN class variance of the K groups}
\PY{c+c1}{\PYZsh{} interClassVar(): find the variance of means/centroids of the K groups}
\PY{c+c1}{\PYZsh{} maxInterClassDist(): possibly correlated with interClassVar, find the dist }
\PY{c+c1}{\PYZsh{}							MAXIMUM between the K group\PYZsq{}s centroids/means}
                         
\PY{c+c1}{\PYZsh{} The rationale of the last metric is that a variable/feature that only }
\PY{c+c1}{\PYZsh{} seperates two of the K classes  is undervalued by the Fisher score because }
\PY{c+c1}{\PYZsh{} it may not separate the K\PYZhy{}2 classes remaining well.	}
\PY{c+c1}{\PYZsh{} ... And a separation of two classes (in conjunction with other variables) }
\PY{c+c1}{\PYZsh{}      is important information for the discriminant model}

\PY{c+c1}{\PYZsh{}\PYZsh{}\PYZsh{}\PYZsh{}\PYZsh{}\PYZsh{}\PYZsh{}\PYZsh{}\PYZsh{}\PYZsh{}\PYZsh{}\PYZsh{}\PYZsh{}\PYZsh{}\PYZsh{}\PYZsh{}\PYZsh{}\PYZsh{}\PYZsh{}\PYZsh{}\PYZsh{}\PYZsh{}\PYZsh{}\PYZsh{}\PYZsh{}\PYZsh{}\PYZsh{}\PYZsh{}\PYZsh{}\PYZsh{}\PYZsh{}\PYZsh{}\PYZsh{}\PYZsh{}\PYZsh{}\PYZsh{}\PYZsh{}\PYZsh{}\PYZsh{}\PYZsh{}\PYZsh{}\PYZsh{}\PYZsh{}\PYZsh{}\PYZsh{}\PYZsh{}\PYZsh{}\PYZsh{}\PYZsh{}\PYZsh{}\PYZsh{}\PYZsh{}\PYZsh{}\PYZsh{}\PYZsh{}\PYZsh{}\PYZsh{}\PYZsh{}\PYZsh{}\PYZsh{}\PYZsh{}\PYZsh{}\PYZsh{}\PYZsh{}\PYZsh{}\PYZsh{}\PYZsh{}\PYZsh{}\PYZsh{}\PYZsh{}\PYZsh{}\PYZsh{}\PYZsh{}\PYZsh{}\PYZsh{}}
\PY{c+c1}{\PYZsh{}\PYZsh{}\PYZsh{}\PYZsh{} ds: a data.frame or matrix (numeric values only/factors will be dealt }
\PY{c+c1}{\PYZsh{}\PYZsh{}\PYZsh{}\PYZsh{}	with as integers)  class vec: a vector correspong to the class of }
\PY{c+c1}{\PYZsh{}\PYZsh{}\PYZsh{}\PYZsh{}    the rows of ds           }
\PY{c+c1}{\PYZsh{}\PYZsh{}\PYZsh{}\PYZsh{}\PYZsh{}\PYZsh{}\PYZsh{}\PYZsh{}\PYZsh{}\PYZsh{}\PYZsh{}\PYZsh{}\PYZsh{}\PYZsh{}\PYZsh{}\PYZsh{}\PYZsh{}\PYZsh{}\PYZsh{}\PYZsh{}\PYZsh{}\PYZsh{}\PYZsh{}\PYZsh{}\PYZsh{}\PYZsh{}\PYZsh{}\PYZsh{}\PYZsh{}\PYZsh{}\PYZsh{}\PYZsh{}\PYZsh{}\PYZsh{}\PYZsh{}\PYZsh{}\PYZsh{}\PYZsh{}\PYZsh{}\PYZsh{}\PYZsh{}\PYZsh{}\PYZsh{}\PYZsh{}\PYZsh{}\PYZsh{}\PYZsh{}\PYZsh{}\PYZsh{}\PYZsh{}\PYZsh{}\PYZsh{}\PYZsh{}\PYZsh{}\PYZsh{}\PYZsh{}\PYZsh{}\PYZsh{}\PYZsh{}\PYZsh{}\PYZsh{}\PYZsh{}\PYZsh{}\PYZsh{}\PYZsh{}\PYZsh{}\PYZsh{}\PYZsh{}\PYZsh{}\PYZsh{}\PYZsh{}\PYZsh{}\PYZsh{}\PYZsh{}\PYZsh{}}

minIntraClassVar\PY{o}{\PYZlt{}\PYZhy{}}\PY{k+kr}{function}\PY{p}{(}ds\PY{p}{,}class.vec\PY{p}{)}\PY{p}{\PYZob{}}

	dsfs\PY{o}{\PYZlt{}\PYZhy{}}ds
	\PY{k+kr}{if}\PY{p}{(}\PY{o}{!}\PY{k+kp}{is.matrix}\PY{p}{(}dsfs\PY{p}{)}\PY{p}{)} dsfs\PY{o}{\PYZlt{}\PYZhy{}}\PY{k+kp}{data.matrix}\PY{p}{(}dsfs\PY{p}{)}  
	
	p\PY{o}{\PYZlt{}\PYZhy{}}\PY{k+kp}{ncol}\PY{p}{(}dsfs\PY{p}{)}
	K\PY{o}{\PYZlt{}\PYZhy{}}\PY{k+kp}{length}\PY{p}{(}\PY{k+kp}{levels}\PY{p}{(}class.vec\PY{p}{)}\PY{p}{)}
	n.all\PY{o}{\PYZlt{}\PYZhy{}}\PY{k+kp}{length}\PY{p}{(}class.vec\PY{p}{)}
	n.i\PY{o}{\PYZlt{}\PYZhy{}}\PY{l+m}{0}
	
	intraClassVar\PY{o}{\PYZlt{}\PYZhy{}}\PY{k+kc}{Inf}
	mean.j\PY{o}{\PYZlt{}\PYZhy{}}\PY{k+kp}{colMeans}\PY{p}{(}dsfs\PY{p}{)}
	
	\PY{k+kr}{for}\PY{p}{(}i \PY{k+kr}{in} \PY{l+m}{1}\PY{o}{:}K\PY{p}{)}\PY{p}{\PYZob{}}
		true.vec\PY{o}{\PYZlt{}\PYZhy{}}\PY{p}{(}\PY{k+kp}{as.integer}\PY{p}{(}class.vec\PY{p}{)}\PY{o}{==}i\PY{p}{)}
		n.i\PY{o}{\PYZlt{}\PYZhy{}}\PY{k+kp}{length}\PY{p}{(}\PY{k+kp}{which}\PY{p}{(}true.vec\PY{p}{)}\PY{p}{)}
		mean.class\PY{o}{\PYZlt{}\PYZhy{}}\PY{k+kp}{colMeans}\PY{p}{(}\PY{k+kp}{as.matrix}\PY{p}{(}dsfs\PY{p}{[}true.vec\PY{p}{,}\PY{p}{]}\PY{p}{,}ncol\PY{o}{=}p\PY{p}{)}\PY{p}{)}
		var.class\PY{o}{\PYZlt{}\PYZhy{}}\PY{k+kp}{colSums}\PY{p}{(}\PY{k+kp}{as.matrix}\PY{p}{(}\PY{p}{(}dsfs\PY{p}{[}true.vec\PY{p}{,}\PY{p}{]}\PY{o}{\PYZhy{}}\PY{k+kp}{rep}\PY{p}{(}mean.class\PY{p}{,}each\PY{o}{=}n.i\PY{p}{)}\PY{p}{)}\PY{o}{\PYZca{}}\PY{l+m}{2}\PY{p}{,}ncol\PY{o}{=}p\PY{p}{)}\PY{p}{)}
		intraClassVar\PY{o}{\PYZlt{}\PYZhy{}}\PY{k+kp}{pmin}\PY{p}{(}intraClassVar\PY{p}{,}var.class\PY{o}{/}\PY{p}{(}n.i\PY{l+m}{\PYZhy{}1}\PY{p}{)}\PY{p}{)}
	\PY{p}{\PYZcb{}}
	\PY{k+kr}{return}\PY{p}{(}intraClassVar\PY{p}{)}
\PY{p}{\PYZcb{}}

interClassVar\PY{o}{\PYZlt{}\PYZhy{}}\PY{k+kr}{function}\PY{p}{(}ds\PY{p}{,}class.vec\PY{p}{)}\PY{p}{\PYZob{}}

	dsfs\PY{o}{\PYZlt{}\PYZhy{}}ds
	\PY{k+kr}{if}\PY{p}{(}\PY{o}{!}\PY{k+kp}{is.matrix}\PY{p}{(}dsfs\PY{p}{)}\PY{p}{)} dsfs\PY{o}{\PYZlt{}\PYZhy{}}\PY{k+kp}{data.matrix}\PY{p}{(}dsfs\PY{p}{)}  
	
	p\PY{o}{\PYZlt{}\PYZhy{}}\PY{k+kp}{ncol}\PY{p}{(}dsfs\PY{p}{)}
	K\PY{o}{\PYZlt{}\PYZhy{}}\PY{k+kp}{length}\PY{p}{(}\PY{k+kp}{levels}\PY{p}{(}class.vec\PY{p}{)}\PY{p}{)}
	
	interClassVar\PY{o}{\PYZlt{}\PYZhy{}}\PY{l+m}{0}
	mean.j\PY{o}{\PYZlt{}\PYZhy{}}\PY{k+kp}{colMeans}\PY{p}{(}dsfs\PY{p}{)}
	
	\PY{k+kr}{for}\PY{p}{(}i \PY{k+kr}{in} \PY{l+m}{1}\PY{o}{:}K\PY{p}{)}\PY{p}{\PYZob{}}
		true.vec\PY{o}{\PYZlt{}\PYZhy{}}\PY{p}{(}\PY{k+kp}{as.integer}\PY{p}{(}class.vec\PY{p}{)}\PY{o}{==}i\PY{p}{)}
		n.i\PY{o}{\PYZlt{}\PYZhy{}}\PY{k+kp}{length}\PY{p}{(}\PY{k+kp}{which}\PY{p}{(}true.vec\PY{p}{)}\PY{p}{)}
		mean.class\PY{o}{\PYZlt{}\PYZhy{}}\PY{k+kp}{colMeans}\PY{p}{(}\PY{k+kp}{as.matrix}\PY{p}{(}dsfs\PY{p}{[}true.vec\PY{p}{,}\PY{p}{]}\PY{p}{,}ncol\PY{o}{=}p\PY{p}{)}\PY{p}{)}
		var.class\PY{o}{\PYZlt{}\PYZhy{}}\PY{p}{(}\PY{p}{(}mean.j\PY{o}{\PYZhy{}}mean.class\PY{p}{)}\PY{o}{\PYZca{}}\PY{l+m}{2}\PY{p}{)}
		interClassVar\PY{o}{\PYZlt{}\PYZhy{}}interClassVar\PY{o}{+}var.class
	\PY{p}{\PYZcb{}}
	\PY{k+kr}{return}\PY{p}{(}interClassVar\PY{o}{/}\PY{p}{(}K\PY{l+m}{\PYZhy{}1}\PY{p}{)}\PY{p}{)}
\PY{p}{\PYZcb{}}

maxInterClassDist\PY{o}{\PYZlt{}\PYZhy{}}\PY{k+kr}{function}\PY{p}{(}ds\PY{p}{,}class.vec\PY{p}{)}\PY{p}{\PYZob{}}

	dsfs\PY{o}{\PYZlt{}\PYZhy{}}ds
	\PY{k+kr}{if}\PY{p}{(}\PY{o}{!}\PY{k+kp}{is.matrix}\PY{p}{(}dsfs\PY{p}{)}\PY{p}{)} dsfs\PY{o}{\PYZlt{}\PYZhy{}}\PY{k+kp}{data.matrix}\PY{p}{(}dsfs\PY{p}{)}  
	
	p\PY{o}{\PYZlt{}\PYZhy{}}\PY{k+kp}{ncol}\PY{p}{(}dsfs\PY{p}{)}
	K\PY{o}{\PYZlt{}\PYZhy{}}\PY{k+kp}{length}\PY{p}{(}\PY{k+kp}{levels}\PY{p}{(}class.vec\PY{p}{)}\PY{p}{)}
	
	interClassDist\PY{o}{\PYZlt{}\PYZhy{}}\PY{o}{\PYZhy{}}\PY{k+kc}{Inf}
	mean.j\PY{o}{\PYZlt{}\PYZhy{}}\PY{k+kp}{colMeans}\PY{p}{(}dsfs\PY{p}{)}
	
	\PY{k+kr}{for}\PY{p}{(}i \PY{k+kr}{in} \PY{l+m}{1}\PY{o}{:}K\PY{p}{)}\PY{p}{\PYZob{}}
		true.vec\PY{o}{\PYZlt{}\PYZhy{}}\PY{p}{(}\PY{k+kp}{as.integer}\PY{p}{(}class.vec\PY{p}{)}\PY{o}{==}i\PY{p}{)}
		mean.class\PY{o}{\PYZlt{}\PYZhy{}}\PY{k+kp}{colMeans}\PY{p}{(}\PY{k+kp}{as.matrix}\PY{p}{(}dsfs\PY{p}{[}true.vec\PY{p}{,}\PY{p}{]}\PY{p}{,}ncol\PY{o}{=}p\PY{p}{)}\PY{p}{)}
		interClassDist\PY{o}{\PYZlt{}\PYZhy{}}\PY{k+kp}{pmax}\PY{p}{(}interClassDist\PY{p}{,}\PY{k+kp}{abs}\PY{p}{(}mean.j\PY{o}{\PYZhy{}}mean.class\PY{p}{)}\PY{p}{)}
	\PY{p}{\PYZcb{}}
	\PY{k+kr}{return}\PY{p}{(}interClassDist\PY{p}{)}
\PY{p}{\PYZcb{}}
\end{Verbatim}



\end{appendix}








\end{document}  


