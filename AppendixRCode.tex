\documentclass[12pt,a4paper,oneside]{report}
\usepackage{fancyhdr}	
%\usepackage{fancyheadings}	
\usepackage{geometry}%[margin=2.5cm, a4paper][top=15mm, bottom=15mm, left=35mm, right=15mm]
\usepackage{amsmath}			% packages to give lots of maths stuff
\usepackage{amssymb}
\usepackage{latexsym}
\usepackage{natbib}				% easy reference package
%\usepackage{units}	
%\usepackage{subfig}
\usepackage{color}
\usepackage[parfill]{parskip}    % Activate to begin paragraphs with an empty line rather than an inden
\usepackage{graphicx}	
\usepackage{fancyvrb}

%\usepackage[utf8]{inputenc}
%\usepackage[latin1]{inputenc}

\usepackage{listings}
\usepackage{algorithm}
\usepackage{algorithmic}
\usepackage{color}
\usepackage{array}
\usepackage{alltt}
\usepackage{upgreek}
\usepackage{tikz}
%\usepackage{threeparttable}
\usepackage{fixltx2e}

\usepackage{enumitem}
\setenumerate[1]{label=(\arabic*)}

%\usepackage{tocvsec2}


%\usepackage{arydshln,mathtools}
%\usepackage[thinlines,thiklines]{easybmat}
%\usepackage{kbordermatrix}% http://www.hss.caltech.edu/~kcb/TeX/kbordermatrix.sty
\usetikzlibrary{arrows,decorations.pathmorphing,backgrounds,positioning,fit,petri}



\usepackage{amsthm}
\usepackage{dsfont}
%\usepackage{caption}
%\usepackage[font={it}]{caption} %font={small,it}
\usepackage[hang,small,bf]{caption}
\usepackage{subcaption}

\setlength{\captionmargin}{20pt}
\captionsetup[subfigure]{labelformat=simple}
\renewcommand\thesubfigure{(\alph{subfigure})}

\usepackage{pdflscape}
\usepackage{multirow}
 
% \usepackage[section]{placeins}
\newtheorem{csprop}{Code Segment}[chapter]
\newtheorem{algprop}{Algorithm}[chapter]
 \def\bcd#1{\begin{csprop}[\scshape{#1}]\mbox{}}
\def\ecd{\end{csprop}}
\def\balg#1{\begin{algprop}[\scshape{#1}]\mbox{}}
\def\ealg{\end{algprop}}

%\usepackage[scaled]{beramono}
%\usepackage[utf8]{inputenc}
%\usepackage[T1]{fontenc}
%\usepackage{lmodern}

%\usepackage[latin1]{inputenc}
%%% make dejvu monospace default tt font
\usepackage[scaled=.8]{beramono}
\usepackage[T1]{fontenc}

\setlength{\textwidth}{150.0mm}
\setlength{\textheight}{220.0mm}
%\setlength{\oddsidemargin}{14.0mm}
%\setlength{\evensidemargin}{7.0mm}
%\setlength{\topmargin}{-7.0mm}
%\renewcommand{\topfraction}{0.99}
%\renewcommand{\textfraction}{0.01}


\setcounter{topnumber}{2}
\setcounter{bottomnumber}{2}
\setcounter{totalnumber}{4}
\renewcommand{\topfraction}{0.85}
\renewcommand{\bottomfraction}{0.85}
\renewcommand{\textfraction}{0.15}
\renewcommand{\floatpagefraction}{0.7}
\renewcommand{\dbltopfraction}{.66}
\renewcommand{\dblfloatpagefraction}{.66}


%\usepackage{enumitem}

%\numberwithin{equation}{section}
%\renewcommand{\thefigure}{\arabic{section}.\arabic{figure}}
%\renewcommand{\thetable}{\arabic{section}.\arabic{table}}

% Set up the naming conventions for equations etc
\renewcommand{\theequation}{\arabic{chapter}.\arabic{equation}}%.\arabic{section}
\renewcommand{\thefigure}{\arabic{chapter}.\arabic{figure}}%.\arabic{section}
\renewcommand{\thetable}{\arabic{chapter}.\arabic{table}}%.\arabic{section}



\clubpenalty10000
\widowpenalty10000
\allowdisplaybreaks
\raggedbottom

\interfootnotelinepenalty=10000

%\input{Preamble/Preamble}

%% This is for the Verbatim environments using pygments
%% shortlist: paraiso-light, manni, autumn, xcode, default, trac, native, friendly

\makeatletter
\def\PY@reset{\let\PY@it=\relax \let\PY@bf=\relax%
    \let\PY@ul=\relax \let\PY@tc=\relax%
    \let\PY@bc=\relax \let\PY@ff=\relax}
\def\PY@tok#1{\csname PY@tok@#1\endcsname}
\def\PY@toks#1+{\ifx\relax#1\empty\else%
    \PY@tok{#1}\expandafter\PY@toks\fi}
\def\PY@do#1{\PY@bc{\PY@tc{\PY@ul{%
    \PY@it{\PY@bf{\PY@ff{#1}}}}}}}
\def\PY#1#2{\PY@reset\PY@toks#1+\relax+\PY@do{#2}}

\expandafter\def\csname PY@tok@gd\endcsname{\def\PY@bc##1{\setlength{\fboxsep}{0pt}\fcolorbox[rgb]{0.80,0.00,0.00}{1.00,0.80,0.80}{\strut ##1}}}
\expandafter\def\csname PY@tok@gu\endcsname{\let\PY@bf=\textbf\def\PY@tc##1{\textcolor[rgb]{0.00,0.20,0.00}{##1}}}
\expandafter\def\csname PY@tok@gt\endcsname{\def\PY@tc##1{\textcolor[rgb]{0.60,0.80,0.40}{##1}}}
\expandafter\def\csname PY@tok@gs\endcsname{\let\PY@bf=\textbf}
\expandafter\def\csname PY@tok@gr\endcsname{\def\PY@tc##1{\textcolor[rgb]{1.00,0.00,0.00}{##1}}}
\expandafter\def\csname PY@tok@cm\endcsname{\let\PY@it=\textit\def\PY@tc##1{\textcolor[rgb]{0.00,0.60,1.00}{##1}}}
\expandafter\def\csname PY@tok@vg\endcsname{\def\PY@tc##1{\textcolor[rgb]{0.00,0.20,0.20}{##1}}}
\expandafter\def\csname PY@tok@m\endcsname{\def\PY@tc##1{\textcolor[rgb]{0.9333333 ,0.1725490 ,0.1725490}{##1}}}   %%%% edited:  CONSTANT/NUMBER
\expandafter\def\csname PY@tok@mh\endcsname{\def\PY@tc##1{\textcolor[rgb]{1.00,0.40,0.00}{##1}}}
\expandafter\def\csname PY@tok@cs\endcsname{\let\PY@bf=\textbf\let\PY@it=\textit\def\PY@tc##1{\textcolor[rgb]{0.00,0.60,1.00}{##1}}}
\expandafter\def\csname PY@tok@ge\endcsname{\let\PY@it=\textit}
\expandafter\def\csname PY@tok@vc\endcsname{\def\PY@tc##1{\textcolor[rgb]{0.00,0.20,0.20}{##1}}}
\expandafter\def\csname PY@tok@il\endcsname{\def\PY@tc##1{\textcolor[rgb]{1.00,0.40,0.00}{##1}}}
\expandafter\def\csname PY@tok@go\endcsname{\def\PY@tc##1{\textcolor[rgb]{0.67,0.67,0.67}{##1}}}
\expandafter\def\csname PY@tok@cp\endcsname{\def\PY@tc##1{\textcolor[rgb]{1.00,0.40,0.0}{##1}}} %%%% "#include" statement in C code
\expandafter\def\csname PY@tok@gi\endcsname{\def\PY@bc##1{\setlength{\fboxsep}{0pt}\fcolorbox[rgb]{0.00,0.80,0.00}{0.80,1.00,0.80}{\strut ##1}}}
\expandafter\def\csname PY@tok@gh\endcsname{\let\PY@bf=\textbf\def\PY@tc##1{\textcolor[rgb]{0.00,0.20,0.00}{##1}}}
\expandafter\def\csname PY@tok@ni\endcsname{\let\PY@bf=\textbf\def\PY@tc##1{\textcolor[rgb]{0.60,0.60,0.60}{##1}}}
\expandafter\def\csname PY@tok@nl\endcsname{\def\PY@tc##1{\textcolor[rgb]{0.60,0.60,1.00}{##1}}}
\expandafter\def\csname PY@tok@nn\endcsname{\def\PY@tc##1{\textcolor[rgb]{0.8509804, 0.3921569, 0.2901961}{##1}}}  %edited %namespace
\expandafter\def\csname PY@tok@no\endcsname{\def\PY@tc##1{\textcolor[rgb]{0.20,0.40,0.00}{##1}}}
\expandafter\def\csname PY@tok@na\endcsname{\def\PY@tc##1{\textcolor[rgb]{0.20,0.00,0.60}{##1}}}
\expandafter\def\csname PY@tok@nb\endcsname{\def\PY@tc##1{\textcolor[rgb]{0.20,0.40,0.40}{##1}}}
\expandafter\def\csname PY@tok@nc\endcsname{\def\PY@tc##1{\textcolor[rgb]{0.8509804, 0.3921569, 0.2901961}{##1}}}  %edited %name class
\expandafter\def\csname PY@tok@nd\endcsname{\def\PY@tc##1{\textcolor[rgb]{0.60,0.60,1.00}{##1}}}
\expandafter\def\csname PY@tok@ne\endcsname{\let\PY@bf=\textbf\def\PY@tc##1{\textcolor[rgb]{0.80,0.00,0.00}{##1}}}
\expandafter\def\csname PY@tok@nf\endcsname{\def\PY@tc##1{\textcolor[rgb]{0.00,0.60,1.00}{##1}}} %%% edited: FUNCTION NAME (C)
\expandafter\def\csname PY@tok@si\endcsname{\def\PY@tc##1{\textcolor[rgb]{0.67,0.00,0.00}{##1}}}
\expandafter\def\csname PY@tok@s2\endcsname{\def\PY@tc##1{\textcolor[rgb]{0.80,0.20,0.00}{##1}}}
\expandafter\def\csname PY@tok@vi\endcsname{\def\PY@tc##1{\textcolor[rgb]{0.00,0.20,0.20}{##1}}}
\expandafter\def\csname PY@tok@nt\endcsname{\def\PY@tc##1{\textcolor[rgb]{0.8509804, 0.3921569, 0.2901961}{##1}}} %edited %name tag
\expandafter\def\csname PY@tok@nv\endcsname{\def\PY@tc##1{\textcolor[rgb]{0.00,0.20,0.20}{##1}}}
\expandafter\def\csname PY@tok@s1\endcsname{\def\PY@tc##1{\textcolor[rgb]{0.80,0.20,0.00}{##1}}}
\expandafter\def\csname PY@tok@gp\endcsname{\let\PY@bf=\textbf\def\PY@tc##1{\textcolor[rgb]{0.00,0.00,0.60}{##1}}}
\expandafter\def\csname PY@tok@sh\endcsname{\def\PY@tc##1{\textcolor[rgb]{0.80,0.20,0.00}{##1}}}
\expandafter\def\csname PY@tok@ow\endcsname{\let\PY@bf=\textbf\def\PY@tc##1{\textcolor[rgb]{0.00,0.00,0.00}{##1}}}
\expandafter\def\csname PY@tok@sx\endcsname{\def\PY@tc##1{\textcolor[rgb]{0.80,0.20,0.00}{##1}}}
\expandafter\def\csname PY@tok@bp\endcsname{\def\PY@tc##1{\textcolor[rgb]{0.20,0.40,0.40}{##1}}}
\expandafter\def\csname PY@tok@c1\endcsname{\let\PY@it=\textit\def\PY@tc##1{\textcolor[rgb]{ 0.6 ,0.6 ,0.6}{##1}}}  %%% edited: COMMENTS 
\expandafter\def\csname PY@tok@kc\endcsname{\def\PY@tc##1{\textcolor[rgb]{0.9333333 ,0.1725490 ,0.1725490}{##1}}} % edited:  CONSTANT/NUMBER                      eg NULL NA,TRUE,FALSE
\expandafter\def\csname PY@tok@c\endcsname{\let\PY@it=\textit\def\PY@tc##1{\textcolor[rgb]{  0.6 ,0.6 ,0.6}{##1}}}  %%% edited: COMMENTS  
\expandafter\def\csname PY@tok@mf\endcsname{\def\PY@tc##1{\textcolor[rgb]{1.00,0.40,0.00}{##1}}}
\expandafter\def\csname PY@tok@err\endcsname{\def\PY@tc##1{\textcolor[rgb]{0.67,0.00,0.00}{##1}}\def\PY@bc##1{\setlength{\fboxsep}{0pt}\colorbox[rgb]{1.00,0.67,0.67}{\strut ##1}}}
\expandafter\def\csname PY@tok@mb\endcsname{\def\PY@tc##1{\textcolor[rgb]{1.00,0.40,0.00}{##1}}}
\expandafter\def\csname PY@tok@ss\endcsname{\def\PY@tc##1{\textcolor[rgb]{1.00,0.80,0.20}{##1}}}
\expandafter\def\csname PY@tok@sr\endcsname{\def\PY@tc##1{\textcolor[rgb]{0.20,0.67,0.67}{##1}}}
\expandafter\def\csname PY@tok@mo\endcsname{\def\PY@tc##1{\textcolor[rgb]{1.00,0.40,0.00}{##1}}}
\expandafter\def\csname PY@tok@kd\endcsname{\let\PY@bf=\textbf\def\PY@tc##1{\textcolor[rgb]{0.00,0.40,0.60}{##1}}}
\expandafter\def\csname PY@tok@mi\endcsname{\def\PY@tc##1{\textcolor[rgb]{0.9333333 ,0.1725490 ,0.1725490}{##1}}} % edited:  CONSTANT/NUMBER  (C-code)
\expandafter\def\csname PY@tok@kn\endcsname{\def\PY@tc##1{\textcolor[rgb]{0.1333333, 0.5450980, 0.1333333}{##1}}} % edited: KEYWORD                like library() etc
\expandafter\def\csname PY@tok@o\endcsname{\def\PY@tc##1{\textcolor[rgb]{ 0.00,0.60,1.00}{##1}}} %%% edited: OPERATOR
\expandafter\def\csname PY@tok@kr\endcsname{\def\PY@tc##1{\textcolor[rgb]{0.1333333, 0.5450980, 0.1333333}{##1}}} %edited KEYWORD                      reserved: eg else,if,
\expandafter\def\csname PY@tok@s\endcsname{\def\PY@tc##1{\textcolor[rgb]{1.00,0.40,0.0}{##1}}} % edited: STRING 
\expandafter\def\csname PY@tok@kp\endcsname{\def\PY@tc##1{\textcolor[rgb]{0.1333333, 0.5450980, 0.1333333}{##1}}} % edited: KEYWORD                like else, length, nrow etc
\expandafter\def\csname PY@tok@w\endcsname{\def\PY@tc##1{\textcolor[rgb]{0.73,0.73,0.73}{##1}}}
\expandafter\def\csname PY@tok@kt\endcsname{\def\PY@tc##1{\textcolor[rgb]{0.1333333, 0.5450980, 0.1333333}{##1}}} % edited: KEYWORD          like matrix, max etc
\expandafter\def\csname PY@tok@sc\endcsname{\def\PY@tc##1{\textcolor[rgb]{0.80,0.20,0.00}{##1}}}
\expandafter\def\csname PY@tok@sb\endcsname{\def\PY@tc##1{\textcolor[rgb]{0.80,0.20,0.00}{##1}}}
\expandafter\def\csname PY@tok@k\endcsname{\def\PY@tc##1{\textcolor[rgb]{0.1333333, 0.5450980, 0.1333333}{##1}}} %edited: KEYWORD   e.g. void, int double
\expandafter\def\csname PY@tok@se\endcsname{\let\PY@bf=\textbf\def\PY@tc##1{\textcolor[rgb]{0.80,0.20,0.00}{##1}}}%edited
\expandafter\def\csname PY@tok@sd\endcsname{\let\PY@it=\textit\def\PY@tc##1{\textcolor[rgb]{0.80,0.20,0.00}{##1}}}%edited



%%% KEYWORD ::: green
%%% STRING ::: orange
%%% OPERATOR ::: blue
%%% CONSTANT/NUMBER ::: red

%%% green: 0.1333333, 0.5450980, 0.1333333
%%% red: 0.9333333 ,0.1725490 ,0.1725490
%%% blue: 0.00,0.60,1.00
%%% orange: 1.00,0.40,0.0



% > c(col2rgb("#40152A")/255)
% [1] 0.25098039 0.08235294 0.16470588
% > c(col2rgb("#731630")/255)
% [1] 0.45098039 0.08627451 0.18823529
% > c(col2rgb("#D9644A")/255)
% [1] 0.8509804 0.3921569 0.2901961
% > c(col2rgb("springgreen2")/255)
% [1] 0.0000000 0.9333333 0.4627451
% > c(col2rgb("#285283")/255)
% [1] 0.1568627 0.3215686 0.5137255
% > c(col2rgb("forestgreen")/255)
% [1] 0.1333333 0.5450980 0.1333333
% > c(col2rgb("firebrick2")/255)
% [1] 0.9333333 0.1725490 0.1725490
% > c(col2rgb("orange")/255)
% [1] 1.0000000 0.6470588 0.0000000
\def\PYZbs{\char`\\}
\def\PYZus{\char`\_}
\def\PYZob{\char`\{}
\def\PYZcb{\char`\}}
\def\PYZca{\char`\^}
\def\PYZam{\char`\&}
\def\PYZlt{\char`\<}
\def\PYZgt{\char`\>}
\def\PYZsh{\char`\#}
\def\PYZpc{\char`\%}
\def\PYZdl{\char`\$}
\def\PYZhy{\char`\-}
\def\PYZsq{\char`\'}
\def\PYZdq{\char`\"}
\def\PYZti{\char`\~}
% for compatibility with earlier versions
\def\PYZat{@}
\def\PYZlb{[}
\def\PYZrb{]}
\makeatother





\lhead[\fancyplain{}{\thepage}]{\fancyplain{}{\nouppercase\rightmark}}
\rhead[\fancyplain{}{\nouppercase\leftmark}]{\fancyplain{}{\thepage}}
\cfoot{}


%\pagestyle{empty}

%\renewcommand{\baselinestretch}{1.5}
 
 
 
\begin{document}

% Change the page numbering system for the preamble
\pagenumbering{roman}
 %\addtolength{\headwidth}{\marginparsep}
% \addtolength{\headwidth}{\marginparwidth}




\tableofcontents

\clearpage


\pagenumbering{arabic}
\pagestyle{fancy}
% What do you want at the top and bottom of a page (see fancyheadings.sty)
%\rhead{\thepage}
%\rhead{\thepage}








\definecolor{linkblue}{rgb}{0.192,0.494,0.675}
\definecolor{fngrey}{rgb}{0.220,0.220,0.251}	

\newcommand{\codetab}[1]{%
\begin{center}
\begin{tabular}{m{4.5cm}m{5cm}r}
\hline
\textsf{Description} & \textsf{File} & \textsf{Functions} \\
  \hline
#1
   \hline
\end{tabular}
\end{center}
}
\newcommand{\codeentry}[3]{%
\multicolumn{3}{l}{\textsf{#1:}} \\
 &  \texttt{\textcolor{linkblue}{/#2}}  & \textcolor{fngrey}{\texttt{#3}} \\ %
}



%%%%%%%%%%%%%%%%%%%%%%%%%%%%%%%%%
%%%%%%%%%%%%%%%%%%%%%%%%%%%%%%%%%
%%%%%%%%%      Appendices        %%%%%%%%%%
%%%%%%%%%%%%%%%%%%%%%%%%%%%%%%%%%
%%%%%%%%%%%%%%%%%%%%%%%%%%%%%%%%%


  
%\setcounter{table}{0}  
% Set up the naming conventions for equations etc

% Set up the naming conventions for equations in the appendices
\renewcommand{\theequation}{\Alph{chapter}.\arabic{equation}} %\arabic{section}.
\renewcommand\thetable{\thechapter.\arabic{table}}    

\renewcommand\thefigure{\thechapter.\arabic{figure}}  

%\renewcommand{\theequation}{\arabic{chapter}.\arabic{equation}}%.\arabic{section}
%\renewcommand{\thefigure}{\arabic{chapter}.\arabic{figure}}%.\arabic{section}
%\renewcommand{\thetable}{\arabic{chapter}.\arabic{table}}%.\arabic{section}

\begin{appendix}
%\addtocontents{toc}{\protect\setcounter{tocdepth}{0}} %% this stops subsections listed in TOC
%\addcontentsline{toc}{chapter}{Appendix}

%\settocdepth{chapter}

\addtocontents{toc}{\setcounter{tocdepth}{3}}
%\addtocontents{toc}{\setcounter{tocdepth}{0}}



\chapter{\texttt{R} code} \label{app:R}


All files referenced in the current appendix are available under the directory at:
\begin{center}
\texttt{\textcolor{linkblue}{https://github.com/tystan/thesis/tree/master/R}}.
\end{center}



\clearpage


\section{Morphological operators}



\begin{center}
\begin{tabular}{m{3cm}m{5cm}r}
\hline
\textsf{Description} & \textsf{File} & \textsf{Functions} \\
  \hline
\codeentry{Naive erosion and top-hat}{00\_erosion\_slow.R}{erode(), dilate(), tophat()}
\codeentry{Line segment erosion}{01\_erosion\_quick.R}{erode\_quick()}
\codeentry{Naive erosion for unequally spaced values}{02\_cts\_erosion\_slow.R}{erode\_cts\_slow()}
\codeentry{Continuous line segment erosion}{03\_cts\_erosion\_quick.R}{erode\_cts\_quick()}
   \hline
\end{tabular}
\end{center}

\clearpage

\subsection{Naive erosion and top-hat} 

Consider $x \in \left\{ 1,2,\ldots,n \right\} = X$ and $ f \left( x \right) \in \mathds{R} \; \; \forall x \in X$. Because of the assumed even spacing of the elements of $X$, the {\tt R}-function below simply requires the vector of intensities, $f$, and the size of the SE. \\


	\begin{Verbatim}[commandchars=\\\{\},codes={\catcode`\$=3\catcode`\^=7\catcode`\_=8},gobble=0,numbers=left,fontfamily=fvm,fontshape=n,fontsize=\footnotesize,tabsize=2]
erode\PY{o}{\PYZlt{}\PYZhy{}}\PY{k+kr}{function}\PY{p}{(}f\PY{p}{,}sesize\PY{p}{)} \PY{c+c1}{\PYZsh{}f are the intensities}
\PY{p}{\PYZob{}} 
   \PY{k+kr}{if}\PY{p}{(}\PY{o}{!}\PY{p}{(}sesize \PY{o}{\PYZpc{}\PYZpc{}} \PY{l+m}{2}\PY{p}{)}\PY{p}{)} \PY{k+kr}{return}\PY{p}{(}\PY{k+kc}{NULL}\PY{p}{)} \PY{c+c1}{\PYZsh{}SE must be of odd length}
   nx\PY{o}{\PYZlt{}\PYZhy{}}\PY{k+kp}{length}\PY{p}{(}f\PY{p}{)}\PY{p}{;} erode\PY{o}{\PYZlt{}\PYZhy{}}\PY{k+kp}{rep}\PY{p}{(}\PY{l+m}{0}\PY{p}{,}nx\PY{p}{)}\PY{p}{;} halfse\PY{o}{\PYZlt{}\PYZhy{}}\PY{p}{(}sesize\PY{l+m}{\PYZhy{}1}\PY{p}{)}\PY{o}{/}\PY{l+m}{2}\PY{p}{;}
   \PY{k+kr}{for}\PY{p}{(}i \PY{k+kr}{in} \PY{l+m}{1}\PY{o}{:}nx\PY{p}{)} \PY{c+c1}{\PYZsh{}for each m/z point across the spectrum}
      erode\PY{p}{[}i\PY{p}{]}\PY{o}{\PYZlt{}\PYZhy{}}\PY{k+kp}{min}\PY{p}{(}f\PY{p}{[}\PY{k+kp}{max}\PY{p}{(}\PY{l+m}{1}\PY{p}{,}i\PY{o}{\PYZhy{}}halfse\PY{p}{)}\PY{o}{:}\PY{k+kp}{min}\PY{p}{(}nx\PY{p}{,}i\PY{o}{+}halfse\PY{p}{)}\PY{p}{]}\PY{p}{)}
   \PY{k+kr}{return}\PY{p}{(}erode\PY{p}{)}
\PY{p}{\PYZcb{}} 
\PY{c+c1}{\PYZsh{}\PYZsh{}\PYZsh{} dilate is the same as erode except use max instead of min ...OR...}
dilate\PY{o}{\PYZlt{}\PYZhy{}}\PY{k+kr}{function}\PY{p}{(}f\PY{p}{,}sesize\PY{p}{)} \PY{k+kr}{return}\PY{p}{(}\PY{o}{\PYZhy{}}erode\PY{p}{(}\PY{o}{\PYZhy{}}f\PY{p}{,}sesize\PY{p}{)}\PY{p}{)}
\PY{c+c1}{\PYZsh{}\PYZsh{}\PYZsh{} as defined $\tau_B\left(f\right)=f-\left(f\ominus{B}\right)\oplus{B}$}
tophat\PY{o}{\PYZlt{}\PYZhy{}}\PY{k+kr}{function}\PY{p}{(}f\PY{p}{,}sesize\PY{p}{)} \PY{k+kr}{return}\PY{p}{(}f\PY{o}{\PYZhy{}}dilate\PY{p}{(}erode\PY{p}{(}f\PY{p}{,}sesize\PY{p}{)}\PY{p}{,}sesize\PY{p}{)}\PY{p}{)}
\end{Verbatim}


\vskip12pt


Note, the code checks the SE provided is an odd integer as a centred, symmetric SE is required. The maximum and minimum statements on line 6 of the code segment above, namely {\tt max(1,i-halfse)} and {\tt min(nx,i+halfse)}, check for when the SE sits over the `edge' on the left or right of the series, respectively. This ensures only defined $f$ values will be used. 
	
\clearpage	
	
\subsection{Line segment erosion} 


	\begin{Verbatim}[commandchars=\\\{\},codes={\catcode`\$=3\catcode`\^=7\catcode`\_=8},gobble=0,numbers=left,fontfamily=fvm,fontshape=n,fontsize=\footnotesize,tabsize=2]
erode.quick\PY{o}{\PYZlt{}\PYZhy{}}\PY{k+kr}{function}\PY{p}{(}f\PY{p}{,}se.size\PY{p}{)}\PY{p}{\PYZob{}}
	nx\PY{o}{\PYZlt{}\PYZhy{}}\PY{k+kp}{length}\PY{p}{(}f\PY{p}{)}
	k\PY{o}{\PYZlt{}\PYZhy{}}se.size
	t1\PY{o}{\PYZlt{}\PYZhy{}}\PY{k+kp}{proc.time}\PY{p}{(}\PY{p}{)}\PY{p}{[}\PY{l+m}{3}\PY{p}{]} \PY{c+c1}{\PYZsh{}\PYZsh{}\PYZsh{} get start time}
	\PY{k+kr}{if}\PY{p}{(}k\PY{o}{\PYZgt{}=}nx\PY{p}{)}\PY{p}{\PYZob{}}
		\PY{k+kp}{cat}\PY{p}{(}\PY{l+s}{\PYZdq{}}\PY{l+s}{Warning: structuring element is \PYZgt{}= in length as the input\PYZbs{}n\PYZdq{}}\PY{p}{)}
		\PY{k+kp}{cat}\PY{p}{(}\PY{l+s}{\PYZdq{}}\PY{l+s}{The input vector has been output \PYZbs{}n\PYZdq{}}\PY{p}{)}
		\PY{k+kr}{return}\PY{p}{(}f\PY{p}{)}
	\PY{p}{\PYZcb{}}\PY{k+kp}{else}\PY{p}{\PYZob{}}
		\PY{k+kr}{if}\PY{p}{(}\PY{p}{(}k\PY{o}{\PYZpc{}\PYZpc{}}\PY{l+m}{2}\PY{p}{)} \PY{o}{!=} \PY{l+m}{1}\PY{p}{)}\PY{p}{\PYZob{}}
			k\PY{o}{\PYZlt{}\PYZhy{}}k\PY{l+m}{\PYZhy{}1}
			\PY{k+kp}{cat}\PY{p}{(}\PY{l+s}{\PYZdq{}}\PY{l+s}{Structring Element not symmetric, using SE length \PYZhy{}1 =\PYZdq{}}\PY{p}{,}k\PY{p}{,}\PY{l+s}{\PYZdq{}}\PY{l+s}{\PYZbs{}n\PYZdq{}}\PY{p}{)}
		\PY{p}{\PYZcb{}}
		k.left\PY{o}{\PYZlt{}\PYZhy{}}\PY{p}{(}k\PY{l+m}{\PYZhy{}1}\PY{p}{)}\PY{o}{/}\PY{l+m}{2}
		add.pix\PY{o}{\PYZlt{}\PYZhy{}}k\PY{o}{\PYZhy{}}\PY{p}{(}nx\PY{o}{\PYZpc{}\PYZpc{}}k\PY{p}{)}
		isMultiple\PY{o}{\PYZlt{}\PYZhy{}}\PY{p}{(}add.pix\PY{o}{==}k\PY{p}{)}
		\PY{k+kr}{if}\PY{p}{(}\PY{o}{!}isMultiple\PY{p}{)}\PY{p}{\PYZob{}}
			f\PY{o}{\PYZlt{}\PYZhy{}}\PY{k+kt}{c}\PY{p}{(}f\PY{p}{,}\PY{k+kp}{rep}\PY{p}{(}\PY{o}{+}\PY{k+kc}{Inf}\PY{p}{,}add.pix\PY{p}{)}\PY{p}{)}
			rem.indxs\PY{o}{\PYZlt{}\PYZhy{}}\PY{p}{(}nx\PY{l+m}{+1}\PY{p}{)}\PY{o}{:}\PY{p}{(}nx\PY{o}{+}add.pix\PY{p}{)}
			nx\PY{o}{\PYZlt{}\PYZhy{}}nx\PY{o}{+}add.pix
		\PY{p}{\PYZcb{}}
		g\PY{o}{\PYZlt{}\PYZhy{}}\PY{k+kp}{rep}\PY{p}{(}\PY{l+m}{0}\PY{p}{,}nx\PY{p}{)}
		h\PY{o}{\PYZlt{}\PYZhy{}}\PY{k+kp}{rep}\PY{p}{(}\PY{l+m}{0}\PY{p}{,}nx\PY{p}{)}
		r\PY{o}{\PYZlt{}\PYZhy{}}\PY{k+kp}{rep}\PY{p}{(}\PY{l+m}{0}\PY{p}{,}nx\PY{p}{)}
		j\PY{o}{\PYZlt{}\PYZhy{}}nx
		\PY{k+kr}{for}\PY{p}{(}i \PY{k+kr}{in} \PY{l+m}{1}\PY{o}{:}nx\PY{p}{)}\PY{p}{\PYZob{}}
			\PY{k+kr}{if}\PY{p}{(}i\PY{o}{\PYZpc{}\PYZpc{}}k\PY{o}{==}\PY{l+m}{1}\PY{p}{)}\PY{p}{\PYZob{}}
				g\PY{p}{[}i\PY{p}{]}\PY{o}{\PYZlt{}\PYZhy{}}f\PY{p}{[}i\PY{p}{]}
			\PY{p}{\PYZcb{}}\PY{k+kp}{else}\PY{p}{\PYZob{}}
				g\PY{p}{[}i\PY{p}{]}\PY{o}{\PYZlt{}\PYZhy{}}\PY{k+kp}{min}\PY{p}{(}g\PY{p}{[}i\PY{l+m}{\PYZhy{}1}\PY{p}{]}\PY{p}{,}f\PY{p}{[}i\PY{p}{]}\PY{p}{)}
			\PY{p}{\PYZcb{}}
			\PY{k+kr}{if}\PY{p}{(}j\PY{o}{\PYZpc{}\PYZpc{}}k\PY{o}{==}\PY{l+m}{0}\PY{p}{)}\PY{p}{\PYZob{}}
				h\PY{p}{[}j\PY{p}{]}\PY{o}{\PYZlt{}\PYZhy{}}f\PY{p}{[}j\PY{p}{]}
			\PY{p}{\PYZcb{}}\PY{k+kp}{else}\PY{p}{\PYZob{}}
				h\PY{p}{[}j\PY{p}{]}\PY{o}{\PYZlt{}\PYZhy{}}\PY{k+kp}{min}\PY{p}{(}h\PY{p}{[}j\PY{l+m}{+1}\PY{p}{]}\PY{p}{,}f\PY{p}{[}j\PY{p}{]}\PY{p}{)}
			\PY{p}{\PYZcb{}}
			j\PY{o}{\PYZlt{}\PYZhy{}}j\PY{l+m}{\PYZhy{}1}
		\PY{p}{\PYZcb{}}
		r\PY{p}{[}\PY{l+m}{1}\PY{o}{:}\PY{p}{(}k.left\PY{l+m}{+1}\PY{p}{)}\PY{p}{]}\PY{o}{\PYZlt{}\PYZhy{}}g\PY{p}{[}\PY{p}{(}k.left\PY{l+m}{+1}\PY{p}{)}\PY{o}{:}k\PY{p}{]}
		r\PY{p}{[}\PY{p}{(}k.left\PY{l+m}{+2}\PY{p}{)}\PY{o}{:}\PY{p}{(}nx\PY{o}{\PYZhy{}}k.left\PY{l+m}{\PYZhy{}1}\PY{p}{)}\PY{p}{]}\PY{o}{\PYZlt{}\PYZhy{}}\PY{k+kp}{pmin}\PY{p}{(}g\PY{p}{[}\PY{p}{(}k\PY{l+m}{+1}\PY{p}{)}\PY{o}{:}\PY{p}{(}nx\PY{l+m}{\PYZhy{}1}\PY{p}{)}\PY{p}{]}\PY{p}{,}h\PY{p}{[}\PY{l+m}{2}\PY{o}{:}\PY{p}{(}nx\PY{o}{\PYZhy{}}k\PY{p}{)}\PY{p}{]}\PY{p}{)}
		r\PY{p}{[}\PY{p}{(}nx\PY{o}{\PYZhy{}}k.left\PY{p}{)}\PY{o}{:}nx\PY{p}{]}\PY{o}{\PYZlt{}\PYZhy{}}h\PY{p}{[}\PY{p}{(}nx\PY{o}{\PYZhy{}}k\PY{l+m}{+1}\PY{p}{)}\PY{o}{:}\PY{p}{(}nx\PY{o}{\PYZhy{}}k.left\PY{p}{)}\PY{p}{]}
		\PY{k+kr}{if}\PY{p}{(}\PY{o}{!}isMultiple\PY{p}{)} r\PY{o}{\PYZlt{}\PYZhy{}}r\PY{p}{[}\PY{o}{\PYZhy{}}rem.indxs\PY{p}{]}
		
		delta.t\PY{o}{\PYZlt{}\PYZhy{}}\PY{k+kp}{sprintf}\PY{p}{(}\PY{l+s}{\PYZdq{}}\PY{l+s}{\PYZpc{}.2f\PYZdq{}}\PY{p}{,}\PY{k+kp}{proc.time}\PY{p}{(}\PY{p}{)}\PY{p}{[}\PY{l+m}{3}\PY{p}{]}\PY{o}{\PYZhy{}}t1\PY{p}{)} \PY{c+c1}{\PYZsh{}\PYZsh{}\PYZsh{} time elapsed}
		\PY{k+kp}{cat}\PY{p}{(}\PY{l+s}{\PYZdq{}}\PY{l+s}{Completed morphological erosion in\PYZdq{}}\PY{p}{,}delta.t\PY{p}{,}\PY{l+s}{\PYZdq{}}\PY{l+s}{seconds\PYZbs{}n\PYZdq{}}\PY{p}{)}
		\PY{k+kr}{return}\PY{p}{(}r\PY{p}{)}
	\PY{p}{\PYZcb{}}
\PY{p}{\PYZcb{}}
\end{Verbatim}


\clearpage
		
\subsection{Naive erosion for unequally spaced values}


A vector of all the lower bounds (LB) $x_l$ in Algorithm \ref{alg1} for each corresponding $x_i$ can be created using a $O (n)$ algorithm using two pointers along the input vector $X$. One pointer is the current position, the other is a lagging pointer that moves along the vector when required. A simple algorithm to do this is described in Algorithm \ref{alg2}. To find the upper bound the same algorithm would be employed, but the pointers will start from the right and move down the vector with the second point lagging to the right.\\[15pt]


	\begin{Verbatim}[commandchars=\\\{\},codes={\catcode`\$=3\catcode`\^=7\catcode`\_=8},gobble=0,numbers=left,fontfamily=fvm,fontshape=n,fontsize=\footnotesize,tabsize=2]
get.lo.bounds\PY{o}{\PYZlt{}\PYZhy{}}\PY{k+kr}{function}\PY{p}{(}x\PY{p}{,}k\PY{p}{)}\PY{p}{\PYZob{}}
	nx \PY{o}{\PYZlt{}\PYZhy{}} \PY{k+kp}{length}\PY{p}{(}x\PY{p}{)}
	k0 \PY{o}{\PYZlt{}\PYZhy{}} k\PY{o}{/}\PY{l+m}{2}
	LO \PY{o}{\PYZlt{}\PYZhy{}} \PY{k+kp}{rep}\PY{p}{(}\PY{l+m}{0}\PY{p}{,}nx\PY{p}{)}
	i \PY{o}{\PYZlt{}\PYZhy{}} \PY{l+m}{2}
	i\PYZus{}lo \PY{o}{\PYZlt{}\PYZhy{}} \PY{l+m}{1}
	LO\PY{p}{[}i\PYZus{}lo\PY{p}{]} \PY{o}{\PYZlt{}\PYZhy{}} i\PYZus{}lo \PY{c+c1}{\PYZsh{} x[1] has lower bound 1}
	\PY{k+kr}{while}\PY{p}{(}i\PY{o}{\PYZlt{}=}nx\PY{p}{)}
	\PY{p}{\PYZob{}}
		\PY{k+kr}{if}\PY{p}{(}\PY{p}{(}x\PY{p}{[}i\PY{p}{]}\PY{o}{\PYZhy{}}x\PY{p}{[}i\PYZus{}lo\PY{p}{]}\PY{p}{)}\PY{o}{\PYZlt{}=}k0\PY{p}{)}\PY{p}{\PYZob{}}
			LO\PY{p}{[}i\PY{p}{]} \PY{o}{\PYZlt{}\PYZhy{}} i\PYZus{}lo
			i \PY{o}{\PYZlt{}\PYZhy{}} i\PY{l+m}{+1}
		\PY{p}{\PYZcb{}}\PY{k+kp}{else}\PY{p}{\PYZob{}}
			i\PYZus{}lo \PY{o}{\PYZlt{}\PYZhy{}} i\PYZus{}lo\PY{l+m}{+1} 
		\PY{p}{\PYZcb{}}
	\PY{p}{\PYZcb{}}
	\PY{k+kr}{return}\PY{p}{(}LO\PY{p}{)}
\PY{p}{\PYZcb{}}
get.hi.bounds\PY{o}{\PYZlt{}\PYZhy{}}\PY{k+kr}{function}\PY{p}{(}x\PY{p}{,}k\PY{p}{)}\PY{p}{\PYZob{}}
	nx \PY{o}{\PYZlt{}\PYZhy{}} \PY{k+kp}{length}\PY{p}{(}x\PY{p}{)}
	k0 \PY{o}{\PYZlt{}\PYZhy{}} k\PY{o}{/}\PY{l+m}{2}
	HI \PY{o}{\PYZlt{}\PYZhy{}} \PY{k+kp}{rep}\PY{p}{(}\PY{l+m}{0}\PY{p}{,}nx\PY{p}{)}
	i \PY{o}{\PYZlt{}\PYZhy{}} nx\PY{l+m}{\PYZhy{}1}
	i\PYZus{}hi \PY{o}{\PYZlt{}\PYZhy{}} nx
	HI\PY{p}{[}i\PYZus{}hi\PY{p}{]} \PY{o}{\PYZlt{}\PYZhy{}} i\PYZus{}hi \PY{c+c1}{\PYZsh{} x[nx] has upper bound nx}
	\PY{k+kr}{while}\PY{p}{(}i\PY{o}{\PYZgt{}}\PY{l+m}{0}\PY{p}{)}
	\PY{p}{\PYZob{}}
		\PY{k+kr}{if}\PY{p}{(}\PY{p}{(}x\PY{p}{[}i\PYZus{}hi\PY{p}{]}\PY{o}{\PYZhy{}}x\PY{p}{[}i\PY{p}{]}\PY{p}{)}\PY{o}{\PYZlt{}=}k0\PY{p}{)}\PY{p}{\PYZob{}}
			HI\PY{p}{[}i\PY{p}{]} \PY{o}{\PYZlt{}\PYZhy{}} i\PYZus{}hi
			i \PY{o}{\PYZlt{}\PYZhy{}} i\PY{l+m}{\PYZhy{}1}
		\PY{p}{\PYZcb{}}\PY{k+kp}{else}\PY{p}{\PYZob{}}
			i\PYZus{}hi \PY{o}{\PYZlt{}\PYZhy{}} i\PYZus{}hi\PY{l+m}{\PYZhy{}1} 
		\PY{p}{\PYZcb{}}
	\PY{p}{\PYZcb{}}
	\PY{k+kr}{return}\PY{p}{(}HI\PY{p}{)}
\PY{p}{\PYZcb{}}
erode.cts.slow\PY{o}{\PYZlt{}\PYZhy{}}\PY{k+kr}{function}\PY{p}{(}x\PY{p}{,}f\PY{p}{,}k\PY{p}{)}\PY{p}{\PYZob{}}
	nx\PY{o}{\PYZlt{}\PYZhy{}}\PY{k+kp}{length}\PY{p}{(}x\PY{p}{)}
	r\PYZus{}min\PY{o}{\PYZlt{}\PYZhy{}}\PY{k+kp}{rep}\PY{p}{(}\PY{l+m}{0}\PY{p}{,}nx\PY{p}{)}
	LO\PY{o}{\PYZlt{}\PYZhy{}}get.lo.bounds\PY{p}{(}x\PY{p}{,}k\PY{p}{)}
	HI\PY{o}{\PYZlt{}\PYZhy{}}get.hi.bounds\PY{p}{(}x\PY{p}{,}k\PY{p}{)}
	\PY{k+kr}{for}\PY{p}{(}i \PY{k+kr}{in} \PY{l+m}{1}\PY{o}{:}nx\PY{p}{)} r\PYZus{}min\PY{p}{[}i\PY{p}{]}\PY{o}{\PYZlt{}\PYZhy{}}\PY{k+kp}{min}\PY{p}{(}f\PY{p}{[}LO\PY{p}{[}i\PY{p}{]}\PY{o}{:}HI\PY{p}{[}i\PY{p}{]]}\PY{p}{)}
	\PY{k+kr}{return}\PY{p}{(}r\PYZus{}min\PY{p}{)}
\PY{p}{\PYZcb{}}
\end{Verbatim}



	\clearpage



\subsection{Continuous line segment erosion} 


	\begin{Verbatim}[commandchars=\\\{\},codes={\catcode`\$=3\catcode`\^=7\catcode`\_=8},gobble=0,numbers=left,fontfamily=fvm,fontshape=n,fontsize=\footnotesize,tabsize=2]
erode.cts.quick\PY{o}{\PYZlt{}\PYZhy{}}\PY{k+kr}{function}\PY{p}{(}x\PY{p}{,}f\PY{p}{,}se.span\PY{p}{)}\PY{p}{\PYZob{}}

	out.vector\PY{o}{\PYZlt{}\PYZhy{}}\PY{k+kc}{NULL}
	nx\PY{o}{\PYZlt{}\PYZhy{}}\PY{k+kp}{length}\PY{p}{(}x\PY{p}{)}
	x.span\PY{o}{\PYZlt{}\PYZhy{}}x\PY{p}{[}nx\PY{p}{]}\PY{o}{\PYZhy{}}x\PY{p}{[}\PY{l+m}{1}\PY{p}{]}
	k\PY{o}{\PYZlt{}\PYZhy{}}se.span
	t1\PY{o}{\PYZlt{}\PYZhy{}}\PY{k+kp}{proc.time}\PY{p}{(}\PY{p}{)}\PY{p}{[}\PY{l+m}{3}\PY{p}{]}
	isAppend\PY{o}{\PYZlt{}\PYZhy{}}\PY{k+kc}{FALSE}
	
	\PY{k+kr}{if}\PY{p}{(}k\PY{o}{\PYZgt{}=}x.span\PY{p}{)}\PY{p}{\PYZob{}}
		\PY{k+kp}{cat}\PY{p}{(}\PY{l+s}{\PYZdq{}}\PY{l+s}{Warning: structuring element spans the entire input set \PYZbs{}n\PYZdq{}}\PY{p}{)}
		\PY{k+kp}{cat}\PY{p}{(}\PY{l+s}{\PYZdq{}}\PY{l+s}{The input f vector has been output \PYZbs{}n\PYZdq{}}\PY{p}{)}
		\PY{k+kr}{return}\PY{p}{(}f\PY{p}{)}
	\PY{p}{\PYZcb{}}\PY{k+kp}{else}\PY{p}{\PYZob{}}
		m\PY{o}{\PYZlt{}\PYZhy{}}\PY{k+kp}{ceiling}\PY{p}{(}x.span\PY{o}{/}k\PY{p}{)}
		mk\PY{o}{\PYZlt{}\PYZhy{}}m\PY{o}{*}k
		\PY{k+kr}{if}\PY{p}{(}\PY{o}{!}\PY{p}{(}\PY{p}{(}x\PY{p}{[}\PY{l+m}{1}\PY{p}{]}\PY{o}{+}mk\PY{p}{)} \PY{o}{==} x\PY{p}{[}nx\PY{p}{]}\PY{p}{)}\PY{p}{)}\PY{p}{\PYZob{}}
			x\PY{o}{\PYZlt{}\PYZhy{}}\PY{k+kt}{c}\PY{p}{(}x\PY{p}{,}x\PY{p}{[}\PY{l+m}{1}\PY{p}{]}\PY{o}{+}mk\PY{p}{)}
			f\PY{o}{\PYZlt{}\PYZhy{}}\PY{k+kt}{c}\PY{p}{(}f\PY{p}{,}\PY{o}{+}\PY{k+kc}{Inf}\PY{p}{)}
			isAppend\PY{o}{\PYZlt{}\PYZhy{}}\PY{k+kc}{TRUE}
			nx\PY{o}{\PYZlt{}\PYZhy{}}nx\PY{l+m}{+1}
			x.span\PY{o}{\PYZlt{}\PYZhy{}}x\PY{p}{[}nx\PY{p}{]}\PY{o}{\PYZhy{}}x\PY{p}{[}\PY{l+m}{1}\PY{p}{]}
		\PY{p}{\PYZcb{}}
		k.blocks\PY{o}{\PYZlt{}\PYZhy{}}\PY{k+kt}{c}\PY{p}{(}\PY{l+m}{0}\PY{p}{,}\PY{k+kp}{findInterval}\PY{p}{(}x\PY{p}{,}\PY{k+kp}{seq}\PY{p}{(}x\PY{p}{[}\PY{l+m}{1}\PY{p}{]}\PY{p}{,}x\PY{p}{[}\PY{l+m}{1}\PY{p}{]}\PY{o}{+}\PY{p}{(}m\PY{l+m}{\PYZhy{}1}\PY{p}{)}\PY{o}{*}k\PY{p}{,}by\PY{o}{=}k\PY{p}{)}\PY{p}{)}\PY{p}{,}m\PY{l+m}{+1}\PY{p}{)}
		p\PY{o}{\PYZlt{}\PYZhy{}}k.blocks\PY{p}{[}\PY{l+m}{1}\PY{p}{]}
		\PY{k+kp}{q}\PY{o}{\PYZlt{}\PYZhy{}}k.blocks\PY{p}{[}nx\PY{l+m}{+2}\PY{p}{]}
		g\PY{o}{\PYZlt{}\PYZhy{}}\PY{k+kp}{rep}\PY{p}{(}\PY{l+m}{0}\PY{p}{,}nx\PY{p}{)}
		h\PY{o}{\PYZlt{}\PYZhy{}}\PY{k+kp}{rep}\PY{p}{(}\PY{l+m}{0}\PY{p}{,}nx\PY{p}{)}
		r\PY{o}{\PYZlt{}\PYZhy{}}\PY{k+kp}{rep}\PY{p}{(}\PY{l+m}{0}\PY{p}{,}nx\PY{p}{)}
		i\PY{o}{\PYZlt{}\PYZhy{}}\PY{l+m}{1}
		j\PY{o}{\PYZlt{}\PYZhy{}}nx
		\PY{k+kr}{while}\PY{p}{(}i\PY{o}{\PYZlt{}=}nx\PY{p}{)}\PY{p}{\PYZob{}}
			this.p\PY{o}{\PYZlt{}\PYZhy{}}k.blocks\PY{p}{[}i\PY{l+m}{+1}\PY{p}{]}
			this.q\PY{o}{\PYZlt{}\PYZhy{}}k.blocks\PY{p}{[}j\PY{l+m}{+1}\PY{p}{]}
			\PY{k+kr}{if}\PY{p}{(}p\PY{o}{==}this.p\PY{p}{)}\PY{p}{\PYZob{}}
				g\PY{p}{[}i\PY{p}{]}\PY{o}{\PYZlt{}\PYZhy{}}\PY{k+kp}{min}\PY{p}{(}g\PY{p}{[}i\PY{l+m}{\PYZhy{}1}\PY{p}{]}\PY{p}{,}f\PY{p}{[}i\PY{p}{]}\PY{p}{)}
			\PY{p}{\PYZcb{}}\PY{k+kp}{else}\PY{p}{\PYZob{}}
				g\PY{p}{[}i\PY{p}{]}\PY{o}{\PYZlt{}\PYZhy{}}f\PY{p}{[}i\PY{p}{]}
			\PY{p}{\PYZcb{}}
			\PY{k+kr}{if}\PY{p}{(}q\PY{o}{==}this.q\PY{p}{)}\PY{p}{\PYZob{}}
				h\PY{p}{[}j\PY{p}{]}\PY{o}{\PYZlt{}\PYZhy{}}\PY{k+kp}{min}\PY{p}{(}h\PY{p}{[}j\PY{l+m}{+1}\PY{p}{]}\PY{p}{,}f\PY{p}{[}j\PY{p}{]}\PY{p}{)}
			\PY{p}{\PYZcb{}}\PY{k+kp}{else}\PY{p}{\PYZob{}}
				h\PY{p}{[}j\PY{p}{]}\PY{o}{\PYZlt{}\PYZhy{}}f\PY{p}{[}j\PY{p}{]}
			\PY{p}{\PYZcb{}}
			p\PY{o}{\PYZlt{}\PYZhy{}}this.p
			\PY{k+kp}{q}\PY{o}{\PYZlt{}\PYZhy{}}this.q
			i\PY{o}{\PYZlt{}\PYZhy{}}i\PY{l+m}{+1}
			j\PY{o}{\PYZlt{}\PYZhy{}}j\PY{l+m}{\PYZhy{}1}
		\PY{p}{\PYZcb{}}
		k.left\PY{o}{\PYZlt{}\PYZhy{}}k\PY{o}{/}\PY{l+m}{2}
		low.bound.index\PY{o}{\PYZlt{}\PYZhy{}}nx\PY{o}{\PYZhy{}}\PY{k+kp}{rev}\PY{p}{(}\PY{k+kp}{findInterval}\PY{p}{(}\PY{k+kp}{rev}\PY{p}{(}\PY{o}{\PYZhy{}}x\PY{p}{)}\PY{p}{,}\PY{k+kp}{rev}\PY{p}{(}\PY{o}{\PYZhy{}}\PY{p}{(}x\PY{o}{+}k.left\PY{p}{)}\PY{p}{)}\PY{p}{)}\PY{p}{)}\PY{l+m}{+1}
		upp.bound.index\PY{o}{\PYZlt{}\PYZhy{}}\PY{k+kp}{findInterval}\PY{p}{(}x\PY{o}{+}k.left\PY{p}{,}x\PY{p}{)}
		out.vector\PY{o}{\PYZlt{}\PYZhy{}}\PY{k+kp}{pmin}\PY{p}{(}h\PY{p}{[}low.bound.index\PY{p}{]}\PY{p}{,}g\PY{p}{[}upp.bound.index\PY{p}{]}\PY{p}{)}
		which.low\PY{o}{\PYZlt{}\PYZhy{}}\PY{k+kp}{which}\PY{p}{(}k.blocks\PY{p}{[}low.bound.index\PY{p}{]}\PY{o}{==}k.blocks\PY{p}{[}upp.bound.index\PY{l+m}{+1}\PY{p}{]}\PY{p}{)}
		out.vector\PY{p}{[}which.low\PY{p}{]}\PY{o}{\PYZlt{}\PYZhy{}}h\PY{p}{[}low.bound.index\PY{p}{[}which.low\PY{p}{]]}
		which.hi\PY{o}{\PYZlt{}\PYZhy{}}\PY{k+kp}{which}\PY{p}{(}k.blocks\PY{p}{[}low.bound.index\PY{l+m}{+1}\PY{p}{]}\PY{o}{==}k.blocks\PY{p}{[}upp.bound.index\PY{l+m}{+2}\PY{p}{]}\PY{p}{)}
		out.vector\PY{p}{[}which.hi\PY{p}{]}\PY{o}{\PYZlt{}\PYZhy{}}g\PY{p}{[}upp.bound.index\PY{p}{[}which.hi\PY{p}{]]}
		\PY{k+kr}{if}\PY{p}{(}isAppend\PY{p}{)} out.vector\PY{o}{\PYZlt{}\PYZhy{}}out.vector\PY{p}{[}\PY{o}{\PYZhy{}}nx\PY{p}{]}
		\PY{k+kp}{cat}\PY{p}{(}\PY{l+s}{\PYZdq{}}\PY{l+s}{Completed morphological erosion (cts scale) in\PYZdq{}}
			\PY{p}{,}\PY{k+kp}{sprintf}\PY{p}{(}\PY{l+s}{\PYZdq{}}\PY{l+s}{\PYZpc{}.2f\PYZdq{}}\PY{p}{,}\PY{k+kp}{proc.time}\PY{p}{(}\PY{p}{)}\PY{p}{[}\PY{l+m}{3}\PY{p}{]}\PY{o}{\PYZhy{}}t1\PY{p}{)}\PY{p}{,}\PY{l+s}{\PYZdq{}}\PY{l+s}{seconds \PYZbs{}n\PYZdq{}}\PY{p}{)}
		\PY{k+kr}{return}\PY{p}{(}out.vector\PY{p}{)}
	\PY{p}{\PYZcb{}}		
\PY{p}{\PYZcb{}}
\end{Verbatim}







\clearpage


\section{Spectra normalisation}

\codetab{
\codeentry{Empirical quantile normalisation}{04\_quant\_norm.R}{msEQN()}
\codeentry{Pairwise spectra MA normalisation}{05\_ma\_adj.R}{intensAdj()}
}

\clearpage

\subsection{Empirical quantile normalisation} 

	\begin{Verbatim}[commandchars=\\\{\},codes={\catcode`\$=3\catcode`\^=7\catcode`\_=8},gobble=0,numbers=left,fontfamily=fvm,fontshape=n,fontsize=\footnotesize,tabsize=2]
\PY{c+c1}{\PYZsh{}\PYZsh{}\PYZsh{} Input: msData \PYZhy{} the spectra intensities in matrix}
\PY{c+c1}{\PYZsh{}\PYZsh{}\PYZsh{}        where columns are spectra $1,2,\hdots,n$}
quant\PYZus{}norm\PY{o}{\PYZlt{}\PYZhy{}}\PY{k+kr}{function}\PY{p}{(}msData\PY{p}{)}
\PY{p}{\PYZob{}}
	msD\PY{o}{\PYZlt{}\PYZhy{}}msData
	nSpec\PY{o}{\PYZlt{}\PYZhy{}}\PY{k+kp}{ncol}\PY{p}{(}msD\PY{p}{)}
	nDim\PY{o}{\PYZlt{}\PYZhy{}}\PY{k+kp}{nrow}\PY{p}{(}msD\PY{p}{)}

	orders\PY{o}{\PYZlt{}\PYZhy{}}\PY{k+kp}{apply}\PY{p}{(}msD\PY{p}{,} \PY{l+m}{2}\PY{p}{,} \PY{k+kp}{order}\PY{p}{)}
	reorders\PY{o}{\PYZlt{}\PYZhy{}}\PY{k+kp}{apply}\PY{p}{(}orders\PY{p}{,} \PY{l+m}{2}\PY{p}{,} \PY{k+kp}{order}\PY{p}{)}
	\PY{c+c1}{\PYZsh{} order each column into ascending order}
	\PY{k+kr}{for}\PY{p}{(}i \PY{k+kr}{in} \PY{l+m}{1}\PY{o}{:}nSpec\PY{p}{)} msD\PY{p}{[}\PY{p}{,}i\PY{p}{]}\PY{o}{\PYZlt{}\PYZhy{}}msD\PY{p}{[}orders\PY{p}{[}\PY{p}{,}i\PY{p}{]}\PY{p}{,}i\PY{p}{]}
	\PY{c+c1}{\PYZsh{}replace ordered columns with row means}
	rmeans\PY{o}{\PYZlt{}\PYZhy{}}\PY{k+kp}{rowMeans}\PY{p}{(}msD\PY{p}{)}
	\PY{k+kr}{for}\PY{p}{(}i \PY{k+kr}{in} \PY{l+m}{1}\PY{o}{:}nSpec\PY{p}{)} msD\PY{p}{[}\PY{p}{,}i\PY{p}{]}\PY{o}{\PYZlt{}\PYZhy{}}rmeans
	\PY{c+c1}{\PYZsh{}put back into the original order (with changed values)}
	\PY{k+kr}{for}\PY{p}{(}i \PY{k+kr}{in} \PY{l+m}{1}\PY{o}{:}nSpec\PY{p}{)} msD\PY{p}{[}\PY{p}{,}i\PY{p}{]}\PY{o}{\PYZlt{}\PYZhy{}}msD\PY{p}{[}reorders\PY{p}{[}\PY{p}{,}i\PY{p}{]}\PY{p}{,}i\PY{p}{]}

	\PY{k+kr}{return}\PY{p}{(}msD\PY{p}{)}
\PY{p}{\PYZcb{}}
\end{Verbatim}

	
\clearpage

\subsection{Pairwise spectra MA normalisation} 

	\begin{Verbatim}[commandchars=\\\{\},codes={\catcode`\$=3\catcode`\^=7\catcode`\_=8},gobble=0,numbers=left,fontfamily=fvm,fontshape=n,fontsize=\footnotesize,tabsize=2]
\PY{c+c1}{\PYZsh{}\PYZsh{}\PYZsh{} m\PYZus{}adj(): Used by ma\PYZus{}adj(), performs LOESS regression on ordered MA\PYZhy{}vals}
\PY{c+c1}{\PYZsh{}\PYZsh{}\PYZsh{} Input: ordered dependent variable $A$ with corresponding $M$}
\PY{c+c1}{\PYZsh{}\PYZsh{}\PYZsh{} Returns: adjusted $M$ values, $M_t^*$}
m\PYZus{}adj\PY{o}{\PYZlt{}\PYZhy{}}\PY{k+kr}{function}\PY{p}{(}ordered\PYZus{}M\PY{p}{,}ordered\PYZus{}A\PY{p}{)}
\PY{p}{\PYZob{}}
	MA\PYZus{}finites\PY{o}{\PYZlt{}\PYZhy{}}\PY{k+kp}{is.finite}\PY{p}{(}ordered\PYZus{}M\PY{p}{)} \PY{c+c1}{\PYZsh{}only include values $>-\infty$}
	finites\PYZus{}M\PY{o}{\PYZlt{}\PYZhy{}}ordered\PYZus{}M\PY{p}{[}MA\PYZus{}finites\PY{p}{]}
	finites\PYZus{}A\PY{o}{\PYZlt{}\PYZhy{}}ordered\PYZus{}A\PY{p}{[}MA\PYZus{}finites\PY{p}{]}
	
	MAloess\PY{o}{\PYZlt{}\PYZhy{}}loess\PY{p}{(}finites\PYZus{}M\PY{o}{\PYZti{}}finites\PYZus{}A
		\PY{p}{,}span\PY{o}{=}\PY{l+m}{0.40}\PY{p}{,}degree\PY{o}{=}\PY{l+m}{2}\PY{p}{,}family\PY{o}{=}\PY{l+s}{\PYZdq{}}\PY{l+s}{symmetric\PYZdq{}}\PY{p}{,}normalize\PY{o}{=}\PY{k+kc}{FALSE}\PY{p}{)}

	finites\PYZus{}M\PY{o}{\PYZlt{}\PYZhy{}}finites\PYZus{}M\PY{o}{\PYZhy{}}MAloess\PY{o}{\PYZdl{}}fitted \PY{c+c1}{\PYZsh{} make adjustments }
	ordered\PYZus{}M\PY{p}{[}MA\PYZus{}finites\PY{p}{]}\PY{o}{\PYZlt{}\PYZhy{}}finites\PYZus{}M \PY{c+c1}{\PYZsh{} and return adjusted values}
	\PY{k+kr}{return}\PY{p}{(}ordered\PYZus{}M\PY{p}{)}
\PY{p}{\PYZcb{}}

\PY{c+c1}{\PYZsh{}\PYZsh{}\PYZsh{} ma\PYZus{}adj(): Perform MA adjustment on two vectors}
\PY{c+c1}{\PYZsh{}\PYZsh{}\PYZsh{} Input: Two spectra vectors $F_1$ and $F_2$}
\PY{c+c1}{\PYZsh{}\PYZsh{}\PYZsh{} Returns: MA adjusted $F_1$ and $F_2$ values, $F_1^*$ and $F_2^*$ respectively}
ma\PYZus{}adj\PY{o}{\PYZlt{}\PYZhy{}}\PY{k+kr}{function}\PY{p}{(}F1\PY{p}{,}F2\PY{p}{)}
\PY{p}{\PYZob{}}
	t1\PY{o}{\PYZlt{}\PYZhy{}}\PY{k+kp}{proc.time}\PY{p}{(}\PY{p}{)}\PY{p}{[}\PY{l+m}{3}\PY{p}{]} \PY{c+c1}{\PYZsh{}\PYZsh{}\PYZsh{} get start time}
	V1\PY{o}{\PYZlt{}\PYZhy{}}\PY{k+kp}{log2}\PY{p}{(}F1\PY{p}{)} \PY{c+c1}{\PYZsh{} Will produce $-\infty$ for $\log_2(0)$}
	V2\PY{o}{\PYZlt{}\PYZhy{}}\PY{k+kp}{log2}\PY{p}{(}F2\PY{p}{)}
	M\PY{o}{\PYZlt{}\PYZhy{}}V1\PY{o}{\PYZhy{}}V2
	A\PY{o}{\PYZlt{}\PYZhy{}}\PY{p}{(}V1\PY{o}{+}V2\PY{p}{)}\PY{o}{/}\PY{l+m}{2}
	\PY{c+c1}{\PYZsh{}\PYZsh{}\PYZsh{} $A$ is the dependent regression variable,}
	\PY{c+c1}{\PYZsh{}\PYZsh{}\PYZsh{} ordering required for m\PYZus{}adj function}
	ordered\PYZus{}indxs\PY{o}{\PYZlt{}\PYZhy{}}\PY{k+kp}{order}\PY{p}{(}A\PY{p}{)}
	ordered\PYZus{}A\PY{o}{\PYZlt{}\PYZhy{}}A\PY{p}{[}ordered\PYZus{}indxs\PY{p}{]}
	ordered\PYZus{}M\PY{o}{\PYZlt{}\PYZhy{}}M\PY{p}{[}ordered\PYZus{}indxs\PY{p}{]}
	ordered\PYZus{}M\PY{o}{\PYZlt{}\PYZhy{}}m\PYZus{}adj\PY{p}{(}ordered\PYZus{}M\PY{p}{,}ordered\PYZus{}A\PY{p}{)}
	\PY{c+c1}{\PYZsh{}\PYZsh{}\PYZsh{} get indexes of original ordering}
	orig\PYZus{}order\PY{o}{\PYZlt{}\PYZhy{}}\PY{k+kp}{order}\PY{p}{(}ordered\PYZus{}indxs\PY{p}{)}
	M\PYZus{}dash\PY{o}{\PYZlt{}\PYZhy{}}ordered\PYZus{}M\PY{p}{[}orig\PYZus{}order\PY{p}{]}
	
	orig\PYZus{}finites\PY{o}{\PYZlt{}\PYZhy{}}\PY{k+kp}{is.finite}\PY{p}{(}M\PY{p}{)} \PY{c+c1}{\PYZsh{}update values requiring updating}
	F1\PY{p}{[}orig\PYZus{}finites\PY{p}{]}\PY{o}{\PYZlt{}\PYZhy{}}\PY{l+m}{2}\PY{o}{\PYZca{}}\PY{p}{(}A\PY{p}{[}orig\PYZus{}finites\PY{p}{]}\PY{o}{+}M\PYZus{}dash\PY{p}{[}orig\PYZus{}finites\PY{p}{]}\PY{o}{/}\PY{l+m}{2}\PY{p}{)}
	F2\PY{p}{[}orig\PYZus{}finites\PY{p}{]}\PY{o}{\PYZlt{}\PYZhy{}}\PY{l+m}{2}\PY{o}{\PYZca{}}\PY{p}{(}A\PY{p}{[}orig\PYZus{}finites\PY{p}{]}\PY{o}{\PYZhy{}}M\PYZus{}dash\PY{p}{[}orig\PYZus{}finites\PY{p}{]}\PY{o}{/}\PY{l+m}{2}\PY{p}{)}
	
	delta\PYZus{}t\PY{o}{\PYZlt{}\PYZhy{}}\PY{k+kp}{sprintf}\PY{p}{(}\PY{l+s}{\PYZdq{}}\PY{l+s}{\PYZpc{}.2f\PYZdq{}}\PY{p}{,}\PY{k+kp}{proc.time}\PY{p}{(}\PY{p}{)}\PY{p}{[}\PY{l+m}{3}\PY{p}{]}\PY{o}{\PYZhy{}}t1\PY{p}{)} \PY{c+c1}{\PYZsh{}\PYZsh{}\PYZsh{} time elapsed}
	\PY{k+kp}{cat}\PY{p}{(}\PY{l+s}{\PYZdq{}}\PY{l+s}{Completed MA Normalisation in\PYZdq{}}\PY{p}{,}delta\PYZus{}t\PY{p}{,}\PY{l+s}{\PYZdq{}}\PY{l+s}{seconds \PYZbs{}n\PYZdq{}}\PY{p}{)}
	\PY{k+kr}{return}\PY{p}{(}\PY{k+kt}{list}\PY{p}{(}F1adj\PY{o}{=}F1\PY{p}{,}F2adj\PY{o}{=}F2\PY{p}{)}\PY{p}{)}
\PY{p}{\PYZcb{}}
\end{Verbatim}

	
\clearpage




\section{Peak alignment} \label{pacodes}


\codetab{
\codeentry{Calculate $W$ matrix for an $N$- and $M$-alignment}{06\_create\_w.R}{Wmatrix()}
\codeentry{Dendrogram peak alignment}{07\_dendro\_peak\_align.R}{dendro\_peak\_align()}
}
\clearpage

\subsection{Calculate $W$ matrix for an $N$- and $M$-alignment} 

	\begin{Verbatim}[commandchars=\\\{\},codes={\catcode`\$=3\catcode`\^=7\catcode`\_=8},gobble=0,numbers=left,fontfamily=fvm,fontshape=n,fontsize=\footnotesize,tabsize=2]
\PY{c+c1}{\PYZsh{}\PYZsh{}\PYZsh{}\PYZsh{}\PYZsh{}\PYZsh{}\PYZsh{}\PYZsh{}\PYZsh{}\PYZsh{}\PYZsh{}\PYZsh{}\PYZsh{}\PYZsh{}\PYZsh{}\PYZsh{}\PYZsh{}\PYZsh{}\PYZsh{}\PYZsh{}\PYZsh{}\PYZsh{}\PYZsh{}\PYZsh{}\PYZsh{}\PYZsh{}\PYZsh{}\PYZsh{}\PYZsh{}\PYZsh{}\PYZsh{}\PYZsh{}\PYZsh{}\PYZsh{}\PYZsh{}\PYZsh{}\PYZsh{}\PYZsh{}\PYZsh{}\PYZsh{}\PYZsh{}\PYZsh{}\PYZsh{}\PYZsh{}\PYZsh{}\PYZsh{}\PYZsh{}\PYZsh{}\PYZsh{}\PYZsh{}\PYZsh{}\PYZsh{}\PYZsh{}\PYZsh{}\PYZsh{}\PYZsh{}\PYZsh{}\PYZsh{}\PYZsh{}\PYZsh{}\PYZsh{}\PYZsh{}\PYZsh{}\PYZsh{}\PYZsh{}\PYZsh{}\PYZsh{}\PYZsh{}\PYZsh{}\PYZsh{}\PYZsh{}\PYZsh{}\PYZsh{}\PYZsh{}\PYZsh{}\PYZsh{}\PYZsh{}\PYZsh{}\PYZsh{}}
\PY{c+c1}{\PYZsh{}\PYZsh{}\PYZsh{}\PYZsh{}\PYZsh{}\PYZsh{}\PYZsh{}\PYZsh{}\PYZsh{}\PYZsh{}\PYZsh{}\PYZsh{}\PYZsh{}\PYZsh{}\PYZsh{}\PYZsh{}\PYZsh{}\PYZsh{}\PYZsh{}\PYZsh{}\PYZsh{}\PYZsh{}\PYZsh{}\PYZsh{}\PYZsh{}\PYZsh{}\PYZsh{}\PYZsh{}\PYZsh{}\PYZsh{}\PYZsh{}\PYZsh{}\PYZsh{}\PYZsh{}\PYZsh{}\PYZsh{}\PYZsh{} Function \PYZsh{}\PYZsh{}\PYZsh{}\PYZsh{}\PYZsh{}\PYZsh{}\PYZsh{}\PYZsh{}\PYZsh{}\PYZsh{}\PYZsh{}\PYZsh{}\PYZsh{}\PYZsh{}\PYZsh{}\PYZsh{}\PYZsh{}\PYZsh{}\PYZsh{}\PYZsh{}\PYZsh{}\PYZsh{}\PYZsh{}\PYZsh{}\PYZsh{}\PYZsh{}\PYZsh{}\PYZsh{}\PYZsh{}\PYZsh{}\PYZsh{}\PYZsh{}}
\PY{c+c1}{\PYZsh{}\PYZsh{}\PYZsh{}\PYZsh{}\PYZsh{}\PYZsh{}\PYZsh{}\PYZsh{}\PYZsh{}\PYZsh{}\PYZsh{}\PYZsh{}\PYZsh{}\PYZsh{}\PYZsh{}\PYZsh{}\PYZsh{}\PYZsh{}\PYZsh{}\PYZsh{}\PYZsh{}\PYZsh{}\PYZsh{}\PYZsh{}\PYZsh{}\PYZsh{}\PYZsh{}\PYZsh{}\PYZsh{}\PYZsh{}\PYZsh{}\PYZsh{}\PYZsh{}\PYZsh{}\PYZsh{}\PYZsh{}\PYZsh{}\PYZsh{}\PYZsh{}\PYZsh{}\PYZsh{}\PYZsh{}\PYZsh{}\PYZsh{}\PYZsh{}\PYZsh{}\PYZsh{}\PYZsh{}\PYZsh{}\PYZsh{}\PYZsh{}\PYZsh{}\PYZsh{}\PYZsh{}\PYZsh{}\PYZsh{}\PYZsh{}\PYZsh{}\PYZsh{}\PYZsh{}\PYZsh{}\PYZsh{}\PYZsh{}\PYZsh{}\PYZsh{}\PYZsh{}\PYZsh{}\PYZsh{}\PYZsh{}\PYZsh{}\PYZsh{}\PYZsh{}\PYZsh{}\PYZsh{}\PYZsh{}\PYZsh{}\PYZsh{}\PYZsh{}\PYZsh{}}
\PY{c+c1}{\PYZsh{} }
\PY{c+c1}{\PYZsh{} Wmatrix(): Create a peak similarity matrix between an N\PYZhy{} and M\PYZhy{}alignment}
\PY{c+c1}{\PYZsh{} }
\PY{c+c1}{\PYZsh{}\PYZsh{}\PYZsh{}\PYZsh{}\PYZsh{}\PYZsh{}\PYZsh{}\PYZsh{}\PYZsh{}\PYZsh{}\PYZsh{}\PYZsh{}\PYZsh{}\PYZsh{}\PYZsh{}\PYZsh{}\PYZsh{}\PYZsh{}\PYZsh{}\PYZsh{}\PYZsh{}\PYZsh{}\PYZsh{}\PYZsh{}\PYZsh{}\PYZsh{}\PYZsh{}\PYZsh{}\PYZsh{}\PYZsh{}\PYZsh{}\PYZsh{}\PYZsh{}\PYZsh{}\PYZsh{}\PYZsh{}\PYZsh{}\PYZsh{}\PYZsh{}\PYZsh{}\PYZsh{}\PYZsh{}\PYZsh{}\PYZsh{}\PYZsh{}\PYZsh{}\PYZsh{}\PYZsh{}\PYZsh{}\PYZsh{}\PYZsh{}\PYZsh{}\PYZsh{}\PYZsh{}\PYZsh{}\PYZsh{}\PYZsh{}\PYZsh{}\PYZsh{}\PYZsh{}\PYZsh{}\PYZsh{}\PYZsh{}\PYZsh{}\PYZsh{}\PYZsh{}\PYZsh{}\PYZsh{}\PYZsh{}\PYZsh{}\PYZsh{}\PYZsh{}\PYZsh{}\PYZsh{}\PYZsh{}\PYZsh{}\PYZsh{}\PYZsh{}\PYZsh{}}
\PY{c+c1}{\PYZsh{}\PYZsh{}\PYZsh{}\PYZsh{}\PYZsh{}\PYZsh{}\PYZsh{}\PYZsh{}\PYZsh{}\PYZsh{}\PYZsh{}\PYZsh{}\PYZsh{}\PYZsh{}\PYZsh{}\PYZsh{}\PYZsh{}\PYZsh{}\PYZsh{}\PYZsh{}\PYZsh{}\PYZsh{}\PYZsh{}\PYZsh{}\PYZsh{}\PYZsh{}\PYZsh{}\PYZsh{}\PYZsh{}\PYZsh{}\PYZsh{}\PYZsh{}\PYZsh{}\PYZsh{}\PYZsh{}\PYZsh{}\PYZsh{}\PYZsh{}\PYZsh{} Input \PYZsh{}\PYZsh{}\PYZsh{}\PYZsh{}\PYZsh{}\PYZsh{}\PYZsh{}\PYZsh{}\PYZsh{}\PYZsh{}\PYZsh{}\PYZsh{}\PYZsh{}\PYZsh{}\PYZsh{}\PYZsh{}\PYZsh{}\PYZsh{}\PYZsh{}\PYZsh{}\PYZsh{}\PYZsh{}\PYZsh{}\PYZsh{}\PYZsh{}\PYZsh{}\PYZsh{}\PYZsh{}\PYZsh{}\PYZsh{}\PYZsh{}\PYZsh{}\PYZsh{}}
\PY{c+c1}{\PYZsh{}\PYZsh{}\PYZsh{}\PYZsh{}\PYZsh{}\PYZsh{}\PYZsh{}\PYZsh{}\PYZsh{}\PYZsh{}\PYZsh{}\PYZsh{}\PYZsh{}\PYZsh{}\PYZsh{}\PYZsh{}\PYZsh{}\PYZsh{}\PYZsh{}\PYZsh{}\PYZsh{}\PYZsh{}\PYZsh{}\PYZsh{}\PYZsh{}\PYZsh{}\PYZsh{}\PYZsh{}\PYZsh{}\PYZsh{}\PYZsh{}\PYZsh{}\PYZsh{}\PYZsh{}\PYZsh{}\PYZsh{}\PYZsh{}\PYZsh{}\PYZsh{}\PYZsh{}\PYZsh{}\PYZsh{}\PYZsh{}\PYZsh{}\PYZsh{}\PYZsh{}\PYZsh{}\PYZsh{}\PYZsh{}\PYZsh{}\PYZsh{}\PYZsh{}\PYZsh{}\PYZsh{}\PYZsh{}\PYZsh{}\PYZsh{}\PYZsh{}\PYZsh{}\PYZsh{}\PYZsh{}\PYZsh{}\PYZsh{}\PYZsh{}\PYZsh{}\PYZsh{}\PYZsh{}\PYZsh{}\PYZsh{}\PYZsh{}\PYZsh{}\PYZsh{}\PYZsh{}\PYZsh{}\PYZsh{}\PYZsh{}\PYZsh{}\PYZsh{}\PYZsh{}}
\PY{c+c1}{\PYZsh{}}
\PY{c+c1}{\PYZsh{} Nmatchedpeaks: K x N matrix of matched pairs of the N alignment}
\PY{c+c1}{\PYZsh{} Npeaklists: a list object containing N matricies of  }
\PY{c+c1}{\PYZsh{}         (time,intensityVector) pairs: ($n_a$ x ($n_{Comp}$+1) matrix, a=1,...,N)}
\PY{c+c1}{\PYZsh{} Mmatchedpeaks: L x M matrix of matched pairs of the M alignment}
\PY{c+c1}{\PYZsh{} Mpeaklists: a list object containing M matricies of  }
\PY{c+c1}{\PYZsh{}         (time,intensityVector) pairs: ($n_b$ x ($n_{Comp}$+1) matrix, b=1,...,M)}
\PY{c+c1}{\PYZsh{}}
\PY{c+c1}{\PYZsh{}\PYZsh{}\PYZsh{}\PYZsh{}\PYZsh{}\PYZsh{}\PYZsh{}\PYZsh{}\PYZsh{}\PYZsh{}\PYZsh{}\PYZsh{}\PYZsh{}\PYZsh{}\PYZsh{}\PYZsh{}\PYZsh{}\PYZsh{}\PYZsh{}\PYZsh{}\PYZsh{}\PYZsh{}\PYZsh{}\PYZsh{}\PYZsh{}\PYZsh{}\PYZsh{}\PYZsh{}\PYZsh{}\PYZsh{}\PYZsh{}\PYZsh{}\PYZsh{}\PYZsh{}\PYZsh{}\PYZsh{}\PYZsh{}\PYZsh{}\PYZsh{}\PYZsh{}\PYZsh{}\PYZsh{}\PYZsh{}\PYZsh{}\PYZsh{}\PYZsh{}\PYZsh{}\PYZsh{}\PYZsh{}\PYZsh{}\PYZsh{}\PYZsh{}\PYZsh{}\PYZsh{}\PYZsh{}\PYZsh{}\PYZsh{}\PYZsh{}\PYZsh{}\PYZsh{}\PYZsh{}\PYZsh{}\PYZsh{}\PYZsh{}\PYZsh{}\PYZsh{}\PYZsh{}\PYZsh{}\PYZsh{}\PYZsh{}\PYZsh{}\PYZsh{}\PYZsh{}\PYZsh{}\PYZsh{}\PYZsh{}\PYZsh{}\PYZsh{}\PYZsh{}}
\PY{c+c1}{\PYZsh{} }
\PY{c+c1}{\PYZsh{} e.g. Nmatchedpeaks =}
\PY{c+c1}{\PYZsh{} [ 1 0 1 0 ]}
\PY{c+c1}{\PYZsh{} [ 0 1 2 1 ]}
\PY{c+c1}{\PYZsh{} [ 0 0 0 2 ]}
\PY{c+c1}{\PYZsh{} [ 2 2 0 0 ]}
\PY{c+c1}{\PYZsh{} [ 3 3 3 3 ]}
\PY{c+c1}{\PYZsh{} [ 0 4 0 4 ]}
\PY{c+c1}{\PYZsh{} [ . . . . ]}
\PY{c+c1}{\PYZsh{} [ . . . . ]}
\PY{c+c1}{\PYZsh{} [ . . . . ]}
\PY{c+c1}{\PYZsh{} }
\PY{c+c1}{\PYZsh{} here K x N (N=4) matrix}
\PY{c+c1}{\PYZsh{} }
\PY{c+c1}{\PYZsh{} Npeaklist=}
\PY{c+c1}{\PYZsh{} [[1]]}
\PY{c+c1}{\PYZsh{} [$t_{1,1}$ $t_{1,2}$ ... $t_{1,n_1}$ ]}
\PY{c+c1}{\PYZsh{} [$x_{1,1}$ $x_{1,2}$ ... $x_{1,n_1}$ ]}
\PY{c+c1}{\PYZsh{} }
\PY{c+c1}{\PYZsh{} [[2]]}
\PY{c+c1}{\PYZsh{} [$t_{2,1}$ $t_{2,2}$ ... $t_{2,n_2}$ ]}
\PY{c+c1}{\PYZsh{} [$x_{2,1}$ $x_{2,2}$ ... $x_{2,n_2}$ ]}
\PY{c+c1}{\PYZsh{} }
\PY{c+c1}{\PYZsh{} ...}
\PY{c+c1}{\PYZsh{} }
\PY{c+c1}{\PYZsh{} [[N]]}
\PY{c+c1}{\PYZsh{} [$t_{N,1}$ $t_{N,2}$ ... $t_{N,n_N}$ ]}
\PY{c+c1}{\PYZsh{} [$x_{N,1}$ $x_{N,2}$ ... $x_{N,n_N}$ ]}
\PY{c+c1}{\PYZsh{} }
\PY{c+c1}{\PYZsh{} where $t_{i,j}$ is the time point j\PYZhy{}th peak for the i\PYZhy{}th spectrum}
\PY{c+c1}{\PYZsh{} where $x_{i,j}$ is the vector of intensities ($n_{Comp}$ long)  }
\PY{c+c1}{\PYZsh{}      i.e. $n_{Comp}$ x 1 matrix for the j\PYZhy{}th peak for the i\PYZhy{}th spectrum}
\PY{c+c1}{\PYZsh{} NB: each list item is a $(n_{Comp}+1)$ x $n_N$ matrix}
\PY{c+c1}{\PYZsh{}}
\PY{c+c1}{\PYZsh{} the \PYZus{}i\PYZus{} co\PYZhy{}ord is the row in the Nmatchedpeaks}
\PY{c+c1}{\PYZsh{} the \PYZus{}a\PYZus{} co\PYZhy{}ord is the column number of Nmatchedpeaks (the spectrum number)}
\PY{c+c1}{\PYZsh{} the \PYZus{}p\PYZus{} co\PYZhy{}ord is the peak number for the \PYZus{}a\PYZus{}th spectrum}
\PY{c+c1}{\PYZsh{}}
\PY{c+c1}{\PYZsh{} the \PYZus{}j\PYZus{},\PYZus{}b\PYZus{} and \PYZus{}q\PYZus{} co\PYZhy{}ords are defined similarly for the M\PYZhy{}alignment}

Wmatrix\PY{o}{\PYZlt{}\PYZhy{}}\PY{k+kr}{function}\PY{p}{(}Nmatchedpeaks\PY{p}{,}Npeaklists\PY{p}{,}Mmatchedpeaks\PY{p}{,}Mpeaklists
					\PY{p}{,}D\PY{p}{,}expon\PY{p}{,}lambda\PY{p}{)}\PY{p}{\PYZob{}}

   K\PY{o}{\PYZlt{}\PYZhy{}}\PY{k+kp}{nrow}\PY{p}{(}Nmatchedpeaks\PY{p}{)}
   N\PY{o}{\PYZlt{}\PYZhy{}}\PY{k+kp}{ncol}\PY{p}{(}Nmatchedpeaks\PY{p}{)}
   L\PY{o}{\PYZlt{}\PYZhy{}}\PY{k+kp}{nrow}\PY{p}{(}Mmatchedpeaks\PY{p}{)}
   M\PY{o}{\PYZlt{}\PYZhy{}}\PY{k+kp}{ncol}\PY{p}{(}Mmatchedpeaks\PY{p}{)}
   W\PY{o}{\PYZlt{}\PYZhy{}}\PY{k+kt}{matrix}\PY{p}{(}\PY{l+m}{0}\PY{p}{,}nrow\PY{o}{=}K\PY{p}{,}ncol\PY{o}{=}L\PY{p}{)}
   
   \PY{k+kr}{for}\PY{p}{(}i \PY{k+kr}{in} \PY{l+m}{1}\PY{o}{:}K\PY{p}{)}\PY{p}{\PYZob{}}
      \PY{k+kr}{for}\PY{p}{(}j \PY{k+kr}{in} \PY{l+m}{1}\PY{o}{:}L\PY{p}{)}\PY{p}{\PYZob{}}
         numerator\PY{o}{\PYZlt{}\PYZhy{}}\PY{l+m}{0}
         denominator\PY{o}{\PYZlt{}\PYZhy{}}\PY{l+m}{0}
         \PY{k+kr}{for}\PY{p}{(}a \PY{k+kr}{in} \PY{l+m}{1}\PY{o}{:}N\PY{p}{)}\PY{p}{\PYZob{}}
            p\PY{o}{\PYZlt{}\PYZhy{}}Nmatchedpeaks\PY{p}{[}i\PY{p}{,}a\PY{p}{]}
            \PY{k+kr}{if}\PY{p}{(}p\PY{o}{\PYZgt{}}\PY{l+m}{0}\PY{p}{)}\PY{p}{\PYZob{}}
               \PY{k+kr}{for}\PY{p}{(}b \PY{k+kr}{in} \PY{l+m}{1}\PY{o}{:}M\PY{p}{)}\PY{p}{\PYZob{}}
                  \PY{k+kp}{q}\PY{o}{\PYZlt{}\PYZhy{}}Mmatchedpeaks\PY{p}{[}j\PY{p}{,}b\PY{p}{]}
                  \PY{k+kr}{if}\PY{p}{(}\PY{k+kp}{q}\PY{o}{\PYZgt{}}\PY{l+m}{0}\PY{p}{)}\PY{p}{\PYZob{}}
                     t\PYZus{}a\PY{o}{\PYZlt{}\PYZhy{}}Npeaklists\PY{p}{[[}a\PY{p}{]]}\PY{p}{[}\PY{l+m}{1}\PY{p}{,}p\PY{p}{]}
                     p\PYZus{}a\PY{o}{\PYZlt{}\PYZhy{}}Npeaklists\PY{p}{[[}a\PY{p}{]]}\PY{p}{[}\PY{l+m}{\PYZhy{}1}\PY{p}{,}p\PY{p}{]}
                     t\PYZus{}b\PY{o}{\PYZlt{}\PYZhy{}}Mpeaklists\PY{p}{[[}b\PY{p}{]]}\PY{p}{[}\PY{l+m}{1}\PY{p}{,}\PY{k+kp}{q}\PY{p}{]}
                     p\PYZus{}b\PY{o}{\PYZlt{}\PYZhy{}}Mpeaklists\PY{p}{[[}b\PY{p}{]]}\PY{p}{[}\PY{l+m}{\PYZhy{}1}\PY{p}{,}\PY{k+kp}{q}\PY{p}{]}
                     numerator\PY{o}{\PYZlt{}\PYZhy{}}numerator\PY{o}{+}
                     	PeakSim\PY{p}{(}p\PYZus{}a\PY{p}{,}t\PYZus{}a\PY{p}{,}p\PYZus{}b\PY{p}{,}t\PYZus{}b\PY{p}{,}D\PY{p}{,}expon\PY{p}{,}lambda\PY{p}{)}
                     denominator\PY{o}{\PYZlt{}\PYZhy{}}denominator\PY{l+m}{+1}
                  \PY{p}{\PYZcb{}}
               \PY{p}{\PYZcb{}}
            \PY{p}{\PYZcb{}}
         \PY{p}{\PYZcb{}}
         \PY{k+kr}{if}\PY{p}{(}denominator\PY{o}{\PYZgt{}}\PY{l+m}{0}\PY{p}{)} W\PY{p}{[}i\PY{p}{,}j\PY{p}{]}\PY{o}{\PYZlt{}\PYZhy{}}numerator\PY{o}{/}denominator
         \PY{k+kr}{else} W\PY{p}{[}i\PY{p}{,}j\PY{p}{]}\PY{o}{\PYZlt{}\PYZhy{}}\PY{l+m}{0}
      \PY{p}{\PYZcb{}}
   \PY{p}{\PYZcb{}}
   \PY{k+kr}{return}\PY{p}{(}W\PY{p}{)}
\PY{p}{\PYZcb{}}
\end{Verbatim}


\clearpage

\subsection{Dendrogram peak alignment} 


	\begin{Verbatim}[commandchars=\\\{\},codes={\catcode`\$=3\catcode`\^=7\catcode`\_=8},gobble=0,numbers=left,fontfamily=fvm,fontshape=n,fontsize=\footnotesize,tabsize=2]
\PY{c+c1}{\PYZsh{}\PYZsh{}\PYZsh{}\PYZsh{}\PYZsh{}\PYZsh{}\PYZsh{}\PYZsh{}\PYZsh{}\PYZsh{}\PYZsh{}\PYZsh{}\PYZsh{}\PYZsh{}\PYZsh{}\PYZsh{}\PYZsh{}\PYZsh{}\PYZsh{}\PYZsh{}\PYZsh{}\PYZsh{}\PYZsh{}\PYZsh{}\PYZsh{}\PYZsh{}\PYZsh{}\PYZsh{}\PYZsh{}\PYZsh{}\PYZsh{}\PYZsh{}\PYZsh{}\PYZsh{}\PYZsh{}\PYZsh{}\PYZsh{}\PYZsh{}\PYZsh{}\PYZsh{}\PYZsh{}\PYZsh{}\PYZsh{}\PYZsh{}\PYZsh{}\PYZsh{}\PYZsh{}\PYZsh{}\PYZsh{}\PYZsh{}\PYZsh{}\PYZsh{}\PYZsh{}\PYZsh{}\PYZsh{}\PYZsh{}\PYZsh{}\PYZsh{}\PYZsh{}\PYZsh{}\PYZsh{}\PYZsh{}\PYZsh{}\PYZsh{}\PYZsh{}\PYZsh{}\PYZsh{}\PYZsh{}\PYZsh{}\PYZsh{}\PYZsh{}\PYZsh{}\PYZsh{}\PYZsh{}\PYZsh{}\PYZsh{}\PYZsh{}\PYZsh{}\PYZsh{}}
\PY{c+c1}{\PYZsh{}\PYZsh{}\PYZsh{}\PYZsh{}\PYZsh{}\PYZsh{}\PYZsh{}\PYZsh{}\PYZsh{}\PYZsh{}\PYZsh{}\PYZsh{}\PYZsh{}\PYZsh{}\PYZsh{}\PYZsh{}\PYZsh{}\PYZsh{}\PYZsh{}\PYZsh{}\PYZsh{}\PYZsh{}\PYZsh{}\PYZsh{}\PYZsh{}\PYZsh{}\PYZsh{}\PYZsh{}\PYZsh{}\PYZsh{}\PYZsh{}\PYZsh{}\PYZsh{}\PYZsh{}\PYZsh{}\PYZsh{}\PYZsh{} Function \PYZsh{}\PYZsh{}\PYZsh{}\PYZsh{}\PYZsh{}\PYZsh{}\PYZsh{}\PYZsh{}\PYZsh{}\PYZsh{}\PYZsh{}\PYZsh{}\PYZsh{}\PYZsh{}\PYZsh{}\PYZsh{}\PYZsh{}\PYZsh{}\PYZsh{}\PYZsh{}\PYZsh{}\PYZsh{}\PYZsh{}\PYZsh{}\PYZsh{}\PYZsh{}\PYZsh{}\PYZsh{}\PYZsh{}\PYZsh{}\PYZsh{}\PYZsh{}}
\PY{c+c1}{\PYZsh{}\PYZsh{}\PYZsh{}\PYZsh{}\PYZsh{}\PYZsh{}\PYZsh{}\PYZsh{}\PYZsh{}\PYZsh{}\PYZsh{}\PYZsh{}\PYZsh{}\PYZsh{}\PYZsh{}\PYZsh{}\PYZsh{}\PYZsh{}\PYZsh{}\PYZsh{}\PYZsh{}\PYZsh{}\PYZsh{}\PYZsh{}\PYZsh{}\PYZsh{}\PYZsh{}\PYZsh{}\PYZsh{}\PYZsh{}\PYZsh{}\PYZsh{}\PYZsh{}\PYZsh{}\PYZsh{}\PYZsh{}\PYZsh{}\PYZsh{}\PYZsh{}\PYZsh{}\PYZsh{}\PYZsh{}\PYZsh{}\PYZsh{}\PYZsh{}\PYZsh{}\PYZsh{}\PYZsh{}\PYZsh{}\PYZsh{}\PYZsh{}\PYZsh{}\PYZsh{}\PYZsh{}\PYZsh{}\PYZsh{}\PYZsh{}\PYZsh{}\PYZsh{}\PYZsh{}\PYZsh{}\PYZsh{}\PYZsh{}\PYZsh{}\PYZsh{}\PYZsh{}\PYZsh{}\PYZsh{}\PYZsh{}\PYZsh{}\PYZsh{}\PYZsh{}\PYZsh{}\PYZsh{}\PYZsh{}\PYZsh{}\PYZsh{}\PYZsh{}\PYZsh{}}
\PY{c+c1}{\PYZsh{} }
\PY{c+c1}{\PYZsh{} dendro\PYZus{}peak\PYZus{}align(): for peak list data, create successive N\PYZhy{} and M\PYZhy{}alignments}
\PY{c+c1}{\PYZsh{}              until all spectra are aligned.}
\PY{c+c1}{\PYZsh{} }
\PY{c+c1}{\PYZsh{}\PYZsh{}\PYZsh{}\PYZsh{}\PYZsh{}\PYZsh{}\PYZsh{}\PYZsh{}\PYZsh{}\PYZsh{}\PYZsh{}\PYZsh{}\PYZsh{}\PYZsh{}\PYZsh{}\PYZsh{}\PYZsh{}\PYZsh{}\PYZsh{}\PYZsh{}\PYZsh{}\PYZsh{}\PYZsh{}\PYZsh{}\PYZsh{}\PYZsh{}\PYZsh{}\PYZsh{}\PYZsh{}\PYZsh{}\PYZsh{}\PYZsh{}\PYZsh{}\PYZsh{}\PYZsh{}\PYZsh{}\PYZsh{}\PYZsh{}\PYZsh{}\PYZsh{}\PYZsh{}\PYZsh{}\PYZsh{}\PYZsh{}\PYZsh{}\PYZsh{}\PYZsh{}\PYZsh{}\PYZsh{}\PYZsh{}\PYZsh{}\PYZsh{}\PYZsh{}\PYZsh{}\PYZsh{}\PYZsh{}\PYZsh{}\PYZsh{}\PYZsh{}\PYZsh{}\PYZsh{}\PYZsh{}\PYZsh{}\PYZsh{}\PYZsh{}\PYZsh{}\PYZsh{}\PYZsh{}\PYZsh{}\PYZsh{}\PYZsh{}\PYZsh{}\PYZsh{}\PYZsh{}\PYZsh{}\PYZsh{}\PYZsh{}\PYZsh{}\PYZsh{}}
\PY{c+c1}{\PYZsh{}\PYZsh{}\PYZsh{}\PYZsh{}\PYZsh{}\PYZsh{}\PYZsh{}\PYZsh{}\PYZsh{}\PYZsh{}\PYZsh{}\PYZsh{}\PYZsh{}\PYZsh{}\PYZsh{}\PYZsh{}\PYZsh{}\PYZsh{}\PYZsh{}\PYZsh{}\PYZsh{}\PYZsh{}\PYZsh{}\PYZsh{}\PYZsh{}\PYZsh{}\PYZsh{}\PYZsh{}\PYZsh{}\PYZsh{}\PYZsh{}\PYZsh{}\PYZsh{}\PYZsh{}\PYZsh{}\PYZsh{}\PYZsh{}\PYZsh{}\PYZsh{} Input \PYZsh{}\PYZsh{}\PYZsh{}\PYZsh{}\PYZsh{}\PYZsh{}\PYZsh{}\PYZsh{}\PYZsh{}\PYZsh{}\PYZsh{}\PYZsh{}\PYZsh{}\PYZsh{}\PYZsh{}\PYZsh{}\PYZsh{}\PYZsh{}\PYZsh{}\PYZsh{}\PYZsh{}\PYZsh{}\PYZsh{}\PYZsh{}\PYZsh{}\PYZsh{}\PYZsh{}\PYZsh{}\PYZsh{}\PYZsh{}\PYZsh{}\PYZsh{}\PYZsh{}}
\PY{c+c1}{\PYZsh{}\PYZsh{}\PYZsh{}\PYZsh{}\PYZsh{}\PYZsh{}\PYZsh{}\PYZsh{}\PYZsh{}\PYZsh{}\PYZsh{}\PYZsh{}\PYZsh{}\PYZsh{}\PYZsh{}\PYZsh{}\PYZsh{}\PYZsh{}\PYZsh{}\PYZsh{}\PYZsh{}\PYZsh{}\PYZsh{}\PYZsh{}\PYZsh{}\PYZsh{}\PYZsh{}\PYZsh{}\PYZsh{}\PYZsh{}\PYZsh{}\PYZsh{}\PYZsh{}\PYZsh{}\PYZsh{}\PYZsh{}\PYZsh{}\PYZsh{}\PYZsh{}\PYZsh{}\PYZsh{}\PYZsh{}\PYZsh{}\PYZsh{}\PYZsh{}\PYZsh{}\PYZsh{}\PYZsh{}\PYZsh{}\PYZsh{}\PYZsh{}\PYZsh{}\PYZsh{}\PYZsh{}\PYZsh{}\PYZsh{}\PYZsh{}\PYZsh{}\PYZsh{}\PYZsh{}\PYZsh{}\PYZsh{}\PYZsh{}\PYZsh{}\PYZsh{}\PYZsh{}\PYZsh{}\PYZsh{}\PYZsh{}\PYZsh{}\PYZsh{}\PYZsh{}\PYZsh{}\PYZsh{}\PYZsh{}\PYZsh{}\PYZsh{}\PYZsh{}\PYZsh{}}
\PY{c+c1}{\PYZsh{}}
\PY{c+c1}{\PYZsh{} msD: MS Data, a $T x n$ matrix of MS intensities. One column per spectra.}
\PY{c+c1}{\PYZsh{} peaklistlist: see below}
\PY{c+c1}{\PYZsh{} in.param: [ D expon lambda G maxM ]\PYZhy{}tuple as a vector}
\PY{c+c1}{\PYZsh{}}
\PY{c+c1}{\PYZsh{}\PYZsh{}\PYZsh{}\PYZsh{}\PYZsh{}\PYZsh{}\PYZsh{}\PYZsh{}\PYZsh{}\PYZsh{}\PYZsh{}\PYZsh{}\PYZsh{}\PYZsh{}\PYZsh{}\PYZsh{}\PYZsh{}\PYZsh{}\PYZsh{}\PYZsh{}\PYZsh{}\PYZsh{}\PYZsh{}\PYZsh{}\PYZsh{}\PYZsh{}\PYZsh{}\PYZsh{}\PYZsh{}\PYZsh{}\PYZsh{}\PYZsh{}\PYZsh{}\PYZsh{}\PYZsh{}\PYZsh{}\PYZsh{}\PYZsh{}\PYZsh{}\PYZsh{}\PYZsh{}\PYZsh{}\PYZsh{}\PYZsh{}\PYZsh{}\PYZsh{}\PYZsh{}\PYZsh{}\PYZsh{}\PYZsh{}\PYZsh{}\PYZsh{}\PYZsh{}\PYZsh{}\PYZsh{}\PYZsh{}\PYZsh{}\PYZsh{}\PYZsh{}\PYZsh{}\PYZsh{}\PYZsh{}\PYZsh{}\PYZsh{}\PYZsh{}\PYZsh{}\PYZsh{}\PYZsh{}\PYZsh{}\PYZsh{}\PYZsh{}\PYZsh{}\PYZsh{}\PYZsh{}\PYZsh{}\PYZsh{}\PYZsh{}\PYZsh{}\PYZsh{}}
\PY{c+c1}{\PYZsh{}\PYZsh{}\PYZsh{}\PYZsh{}\PYZsh{}\PYZsh{}\PYZsh{}\PYZsh{}\PYZsh{}\PYZsh{}\PYZsh{}\PYZsh{}\PYZsh{}\PYZsh{}\PYZsh{}\PYZsh{}\PYZsh{}\PYZsh{}\PYZsh{}\PYZsh{}\PYZsh{}\PYZsh{}\PYZsh{}\PYZsh{}\PYZsh{}\PYZsh{}\PYZsh{}\PYZsh{}\PYZsh{}\PYZsh{}\PYZsh{}\PYZsh{}\PYZsh{}\PYZsh{}\PYZsh{}\PYZsh{}\PYZsh{}\PYZsh{} Output \PYZsh{}\PYZsh{}\PYZsh{}\PYZsh{}\PYZsh{}\PYZsh{}\PYZsh{}\PYZsh{}\PYZsh{}\PYZsh{}\PYZsh{}\PYZsh{}\PYZsh{}\PYZsh{}\PYZsh{}\PYZsh{}\PYZsh{}\PYZsh{}\PYZsh{}\PYZsh{}\PYZsh{}\PYZsh{}\PYZsh{}\PYZsh{}\PYZsh{}\PYZsh{}\PYZsh{}\PYZsh{}\PYZsh{}\PYZsh{}\PYZsh{}\PYZsh{}\PYZsh{}}
\PY{c+c1}{\PYZsh{}\PYZsh{}\PYZsh{}\PYZsh{}\PYZsh{}\PYZsh{}\PYZsh{}\PYZsh{}\PYZsh{}\PYZsh{}\PYZsh{}\PYZsh{}\PYZsh{}\PYZsh{}\PYZsh{}\PYZsh{}\PYZsh{}\PYZsh{}\PYZsh{}\PYZsh{}\PYZsh{}\PYZsh{}\PYZsh{}\PYZsh{}\PYZsh{}\PYZsh{}\PYZsh{}\PYZsh{}\PYZsh{}\PYZsh{}\PYZsh{}\PYZsh{}\PYZsh{}\PYZsh{}\PYZsh{}\PYZsh{}\PYZsh{}\PYZsh{}\PYZsh{}\PYZsh{}\PYZsh{}\PYZsh{}\PYZsh{}\PYZsh{}\PYZsh{}\PYZsh{}\PYZsh{}\PYZsh{}\PYZsh{}\PYZsh{}\PYZsh{}\PYZsh{}\PYZsh{}\PYZsh{}\PYZsh{}\PYZsh{}\PYZsh{}\PYZsh{}\PYZsh{}\PYZsh{}\PYZsh{}\PYZsh{}\PYZsh{}\PYZsh{}\PYZsh{}\PYZsh{}\PYZsh{}\PYZsh{}\PYZsh{}\PYZsh{}\PYZsh{}\PYZsh{}\PYZsh{}\PYZsh{}\PYZsh{}\PYZsh{}\PYZsh{}\PYZsh{}\PYZsh{}}
\PY{c+c1}{\PYZsh{}}
\PY{c+c1}{\PYZsh{}\PYZsh{}\PYZsh{} A list containing the following elements:}
\PY{c+c1}{\PYZsh{} dendro: }
\PY{c+c1}{\PYZsh{} stepwise.peaks: a list where each element is the successive amalgamation data}
\PY{c+c1}{\PYZsh{} amalpeaks: the final matrix of aligned peaks. Columns are named spectra, rows}
\PY{c+c1}{\PYZsh{}    are}
\PY{c+c1}{\PYZsh{}}
\PY{c+c1}{\PYZsh{}\PYZsh{}\PYZsh{}\PYZsh{}\PYZsh{}\PYZsh{}\PYZsh{}\PYZsh{}\PYZsh{}\PYZsh{}\PYZsh{}\PYZsh{}\PYZsh{}\PYZsh{}\PYZsh{}\PYZsh{}\PYZsh{}\PYZsh{}\PYZsh{}\PYZsh{}\PYZsh{}\PYZsh{}\PYZsh{}\PYZsh{}\PYZsh{}\PYZsh{}\PYZsh{}\PYZsh{}\PYZsh{}\PYZsh{}\PYZsh{}\PYZsh{}\PYZsh{}\PYZsh{}\PYZsh{}\PYZsh{}\PYZsh{}\PYZsh{}\PYZsh{}\PYZsh{}\PYZsh{}\PYZsh{}\PYZsh{}\PYZsh{}\PYZsh{}\PYZsh{}\PYZsh{}\PYZsh{}\PYZsh{}\PYZsh{}\PYZsh{}\PYZsh{}\PYZsh{}\PYZsh{}\PYZsh{}\PYZsh{}\PYZsh{}\PYZsh{}\PYZsh{}\PYZsh{}\PYZsh{}\PYZsh{}\PYZsh{}\PYZsh{}\PYZsh{}\PYZsh{}\PYZsh{}\PYZsh{}\PYZsh{}\PYZsh{}\PYZsh{}\PYZsh{}\PYZsh{}\PYZsh{}\PYZsh{}\PYZsh{}\PYZsh{}\PYZsh{}\PYZsh{}}

\PY{c+c1}{\PYZsh{}\PYZsh{}\PYZsh{} peaklistlist=}
\PY{c+c1}{\PYZsh{} [[1]]}
\PY{c+c1}{\PYZsh{} [t\PYZus{}\PYZob{}1,1\PYZcb{} t\PYZus{}\PYZob{}1,2\PYZcb{} ... t\PYZus{}\PYZob{}1,n\PYZus{}1\PYZcb{} ]}
\PY{c+c1}{\PYZsh{} [x\PYZus{}\PYZob{}1,1\PYZcb{} x\PYZus{}\PYZob{}1,2\PYZcb{} ... x\PYZus{}\PYZob{}1,n\PYZus{}1\PYZcb{} ]}
\PY{c+c1}{\PYZsh{} }
\PY{c+c1}{\PYZsh{} [[2]]}
\PY{c+c1}{\PYZsh{} [t\PYZus{}\PYZob{}2,1\PYZcb{} t\PYZus{}\PYZob{}2,2\PYZcb{} ... t\PYZus{}\PYZob{}2,n\PYZus{}2\PYZcb{} ]}
\PY{c+c1}{\PYZsh{} [x\PYZus{}\PYZob{}2,1\PYZcb{} x\PYZus{}\PYZob{}2,2\PYZcb{} ... x\PYZus{}\PYZob{}2,n\PYZus{}2\PYZcb{} ]}
\PY{c+c1}{\PYZsh{} }
\PY{c+c1}{\PYZsh{} .}
\PY{c+c1}{\PYZsh{} .}
\PY{c+c1}{\PYZsh{} .}
\PY{c+c1}{\PYZsh{} }
\PY{c+c1}{\PYZsh{} [[N]]}
\PY{c+c1}{\PYZsh{} [t\PYZus{}\PYZob{}N,1\PYZcb{} t\PYZus{}\PYZob{}N,2\PYZcb{} ... t\PYZus{}\PYZob{}N,n\PYZus{}N\PYZcb{} ]}
\PY{c+c1}{\PYZsh{} [x\PYZus{}\PYZob{}N,1\PYZcb{} x\PYZus{}\PYZob{}N,2\PYZcb{} ... x\PYZus{}\PYZob{}N,n\PYZus{}N\PYZcb{} ]}
\PY{c+c1}{\PYZsh{} }
\PY{c+c1}{\PYZsh{} where t\PYZus{}\PYZob{}i,j\PYZcb{} is the time point j\PYZhy{}th peak for the i\PYZhy{}th spectrum}
\PY{c+c1}{\PYZsh{} where x\PYZus{}\PYZob{}i,j\PYZcb{} is the vector of intensities (nComp long i.e. nComp x 1 matrix) }
\PY{c+c1}{\PYZsh{} 				for the j\PYZhy{}th peak for the i\PYZhy{}th spectrum}
\PY{c+c1}{\PYZsh{} NB: each list item is a ($n_{Comp}$+1) x $n_N$ matrix}

dendro\PYZus{}peak\PYZus{}align\PY{o}{\PYZlt{}\PYZhy{}}\PY{k+kr}{function}\PY{p}{(}msD\PY{p}{,}peaklistlist\PY{p}{,}in.param\PY{p}{)}
\PY{p}{\PYZob{}}

	D\PY{o}{\PYZlt{}\PYZhy{}}in.param\PY{p}{[}\PY{l+m}{1}\PY{p}{]}
	nC\PY{o}{\PYZlt{}\PYZhy{}}\PY{k+kp}{nrow}\PY{p}{(}peaklistlist\PY{p}{[[}\PY{l+m}{1}\PY{p}{]]}\PY{p}{)}\PY{l+m}{\PYZhy{}1}
	expon\PY{o}{\PYZlt{}\PYZhy{}}in.param\PY{p}{[}\PY{l+m}{2}\PY{p}{]}
	lambda\PY{o}{\PYZlt{}\PYZhy{}}in.param\PY{p}{[}\PY{l+m}{3}\PY{p}{]}
	G\PY{o}{\PYZlt{}\PYZhy{}}in.param\PY{p}{[}\PY{l+m}{4}\PY{p}{]}
	maxM\PY{o}{\PYZlt{}\PYZhy{}}in.param\PY{p}{[}\PY{l+m}{5}\PY{p}{]}
	
	nPat\PY{o}{\PYZlt{}\PYZhy{}}\PY{k+kp}{length}\PY{p}{(}peaklistlist\PY{p}{)}
	Pats\PY{o}{\PYZlt{}\PYZhy{}}\PY{l+m}{1}\PY{o}{:}nPat
	
	\PY{k+kp}{cat}\PY{p}{(}\PY{l+s}{\PYZdq{}}\PY{l+s}{Calculating merge sequence for spectra \PYZbs{}n\PYZdq{}}\PY{p}{)}
		
	fordist\PY{o}{\PYZlt{}\PYZhy{}}\PY{k+kp}{t}\PY{p}{(}msD\PY{o}{\PYZdl{}}intensity\PY{p}{)}
	hc\PY{o}{\PYZlt{}\PYZhy{}}hclust\PY{p}{(}as.dist\PY{p}{(}fordist\PY{p}{,}diag\PY{o}{=}\PY{k+kc}{FALSE}\PY{p}{,}upper\PY{o}{=}\PY{k+kc}{FALSE}\PY{p}{)}\PY{p}{,}\PY{l+s}{\PYZdq{}}\PY{l+s}{average\PYZdq{}}\PY{p}{)}
	
	\PY{c+c1}{\PYZsh{}\PYZsh{}\PYZsh{} find amalgamation sequence}
	\PY{c+c1}{\PYZsh{}\PYZsh{}\PYZsh{} see ?hclust for information on the merge matrix:}
	\PY{c+c1}{\PYZsh{}    \PYZdq{}an n\PYZhy{}1 by 2 matrix. Row i of merge describes the merging of clusters }
	\PY{c+c1}{\PYZsh{}    at step i of the clustering. If an element j in the row is negative, }
	\PY{c+c1}{\PYZsh{}    then observation \PYZhy{}j was merged at this stage. If j is positive then }
	\PY{c+c1}{\PYZsh{}    the merge was with the cluster formed at the (earlier) stage j of the }
	\PY{c+c1}{\PYZsh{}    algorithm. Thus negative entries in merge indicate agglomerations of }
	\PY{c+c1}{\PYZsh{}    singletons, and positive entries indicate agglomerations of }
	\PY{c+c1}{\PYZsh{}    non\PYZhy{}singletons.\PYZdq{}}
	amalg\PY{o}{\PYZlt{}\PYZhy{}}hc\PY{o}{\PYZdl{}}\PY{k+kp}{merge}

	nAmal\PY{o}{\PYZlt{}\PYZhy{}}\PY{k+kp}{nrow}\PY{p}{(}amalg\PY{p}{)}
	\PY{c+c1}{\PYZsh{} alignment of peaklistlist (a.pll)}
	a.pll\PY{o}{\PYZlt{}\PYZhy{}}\PY{k+kt}{vector}\PY{p}{(}length\PY{o}{=}nAmal\PY{p}{,}mode\PY{o}{=}\PY{l+s}{\PYZdq{}}\PY{l+s}{list\PYZdq{}}\PY{p}{)}
	Npeaks\PY{o}{\PYZlt{}\PYZhy{}}\PY{k+kc}{NULL}
	Npeaklist\PY{o}{\PYZlt{}\PYZhy{}}\PY{k+kc}{NULL}
	Mpeaks\PY{o}{\PYZlt{}\PYZhy{}}\PY{k+kc}{NULL}
	Mpeaklist\PY{o}{\PYZlt{}\PYZhy{}}\PY{k+kc}{NULL}
	
	\PY{c+c1}{\PYZsh{} start}
	\PY{k+kr}{for}\PY{p}{(}aindx \PY{k+kr}{in} \PY{l+m}{1}\PY{o}{:}nAmal\PY{p}{)}
	\PY{p}{\PYZob{}}
	
		\PY{c+c1}{\PYZsh{} if patsToGetN or patsToGetM are positive,}
		\PY{c+c1}{\PYZsh{}    ... it is a single spectrum (1\PYZhy{}alignment)}
		\PY{c+c1}{\PYZsh{} if negative, it is a previous N/M\PYZhy{}alignment (N,M\PYZgt{}1) }
		patsToGetN\PY{o}{\PYZlt{}\PYZhy{}}\PY{o}{\PYZhy{}}amalg\PY{p}{[}aindx\PY{p}{,}\PY{l+m}{1}\PY{p}{]} 
		patsToGetM\PY{o}{\PYZlt{}\PYZhy{}}\PY{o}{\PYZhy{}}amalg\PY{p}{[}aindx\PY{p}{,}\PY{l+m}{2}\PY{p}{]}
		
		printPatsN\PY{o}{\PYZlt{}\PYZhy{}}\PY{k+kp}{sprintf}\PY{p}{(}\PY{l+s}{\PYZdq{}}\PY{l+s}{\PYZpc{}03d\PYZdq{}}\PY{p}{,}patsToGetN\PY{p}{)}
		printPatsM\PY{o}{\PYZlt{}\PYZhy{}}\PY{k+kp}{sprintf}\PY{p}{(}\PY{l+s}{\PYZdq{}}\PY{l+s}{\PYZpc{}03d\PYZdq{}}\PY{p}{,}patsToGetM\PY{p}{)}
		amalg.str\PY{o}{\PYZlt{}\PYZhy{}}\PY{l+s}{\PYZdq{}}\PY{l+s}{Amalgamting patient\PYZdq{}}
		\PY{k+kr}{if}\PY{p}{(}patsToGetN\PY{o}{\PYZgt{}}\PY{l+m}{0} \PY{o}{\PYZam{}\PYZam{}} patsToGetM\PY{o}{\PYZgt{}}\PY{l+m}{0}\PY{p}{)}\PY{p}{\PYZob{}}
			\PY{k+kp}{cat}\PY{p}{(}amalg.str\PY{p}{,}\PY{l+s}{\PYZdq{}}\PY{l+s}{s \PYZdq{}}\PY{p}{,}printPatsN\PY{p}{,}\PY{l+s}{\PYZdq{}}\PY{l+s}{ and \PYZdq{}}\PY{p}{,}printPatsM\PY{p}{,}\PY{l+s}{\PYZdq{}}\PY{l+s}{\PYZbs{}n\PYZdq{}}\PY{p}{,}sep\PY{o}{=}\PY{l+s}{\PYZdq{}}\PY{l+s}{\PYZdq{}}\PY{p}{)}
		\PY{p}{\PYZcb{}}\PY{k+kr}{else} \PY{k+kr}{if}\PY{p}{(}patsToGetN\PY{o}{\PYZgt{}}\PY{l+m}{0} \PY{o}{\PYZam{}\PYZam{}} patsToGetM\PY{o}{\PYZlt{}}\PY{l+m}{0}\PY{p}{)}\PY{p}{\PYZob{}} 
			\PY{k+kp}{cat}\PY{p}{(}amalg.str\PY{p}{,}printPatsN\PY{p}{,}\PY{l+s}{\PYZdq{}}\PY{l+s}{to previously amalgamated patients\PYZbs{}n\PYZdq{}}\PY{p}{)}
		\PY{p}{\PYZcb{}}\PY{k+kr}{else} \PY{k+kr}{if}\PY{p}{(}patsToGetN\PY{o}{\PYZlt{}}\PY{l+m}{0} \PY{o}{\PYZam{}\PYZam{}} patsToGetM\PY{o}{\PYZgt{}}\PY{l+m}{0}\PY{p}{)}\PY{p}{\PYZob{}} 
			\PY{k+kp}{cat}\PY{p}{(}amalg.str\PY{p}{,}printPatsM\PY{p}{,}\PY{l+s}{\PYZdq{}}\PY{l+s}{to previously amalgamated patients\PYZbs{}n\PYZdq{}}\PY{p}{)}
		\PY{p}{\PYZcb{}}\PY{k+kp}{else}\PY{p}{\PYZob{}} 
			\PY{k+kp}{cat}\PY{p}{(}\PY{l+s}{\PYZdq{}}\PY{l+s}{Amalgamting two clusters of previously amalgamated patients\PYZbs{}n\PYZdq{}}\PY{p}{)}
		\PY{p}{\PYZcb{}}
		
		\PY{c+c1}{\PYZsh{}\PYZsh{}\PYZsh{} prepare N\PYZhy{}Alignment data}
		\PY{k+kr}{if}\PY{p}{(}patsToGetN\PY{o}{\PYZgt{}}\PY{l+m}{0}\PY{p}{)}\PY{p}{\PYZob{}} \PY{c+c1}{\PYZsh{} if a single spectrum (1\PYZhy{}alignment)}
			Npeaks\PY{o}{\PYZlt{}\PYZhy{}}\PY{k+kt}{matrix}\PY{p}{(}\PY{l+m}{1}\PY{o}{:}\PY{k+kp}{ncol}\PY{p}{(}peaklistlist\PY{p}{[[}patsToGetN\PY{p}{]]}\PY{p}{)}\PY{p}{,}ncol\PY{o}{=}\PY{l+m}{1}\PY{p}{)}
			Npeaklist\PY{o}{\PYZlt{}\PYZhy{}}peaklistlist\PY{p}{[}patsToGetN\PY{p}{]}
		\PY{p}{\PYZcb{}}\PY{k+kp}{else}\PY{p}{\PYZob{}} \PY{c+c1}{\PYZsh{} if a previously aligned N\PYZhy{}alignment (N\PYZgt{}1)}
			Npeaks\PY{o}{\PYZlt{}\PYZhy{}}a.pll\PY{p}{[[}\PY{o}{\PYZhy{}}patsToGetN\PY{p}{]]}
			patsToGetN\PY{o}{\PYZlt{}\PYZhy{}}\PY{k+kp}{as.numeric}\PY{p}{(}\PY{k+kp}{colnames}\PY{p}{(}Npeaks\PY{p}{)}\PY{p}{)}
			Npeaklist\PY{o}{\PYZlt{}\PYZhy{}}peaklistlist\PY{p}{[}patsToGetN\PY{p}{]}
		\PY{p}{\PYZcb{}}

		\PY{c+c1}{\PYZsh{}\PYZsh{}\PYZsh{} prepare M\PYZhy{}Alignment data}
		\PY{k+kr}{if}\PY{p}{(}patsToGetM\PY{o}{\PYZgt{}}\PY{l+m}{0}\PY{p}{)}\PY{p}{\PYZob{}} \PY{c+c1}{\PYZsh{} if a single spectrum (1\PYZhy{}alignment)}
			Mpeaks\PY{o}{\PYZlt{}\PYZhy{}}\PY{k+kt}{matrix}\PY{p}{(}\PY{l+m}{1}\PY{o}{:}\PY{k+kp}{ncol}\PY{p}{(}peaklistlist\PY{p}{[[}patsToGetM\PY{p}{]]}\PY{p}{)}\PY{p}{,}ncol\PY{o}{=}\PY{l+m}{1}\PY{p}{)}
			Mpeaklist\PY{o}{\PYZlt{}\PYZhy{}}peaklistlist\PY{p}{[}patsToGetM\PY{p}{]}
		\PY{p}{\PYZcb{}}\PY{k+kp}{else}\PY{p}{\PYZob{}} \PY{c+c1}{\PYZsh{} if a previously aligned M\PYZhy{}alignment (M\PYZgt{}1)}
			Mpeaks\PY{o}{\PYZlt{}\PYZhy{}}a.pll\PY{p}{[[}\PY{o}{\PYZhy{}}patsToGetM\PY{p}{]]}
			patsToGetM\PY{o}{\PYZlt{}\PYZhy{}}\PY{k+kp}{as.numeric}\PY{p}{(}\PY{k+kp}{colnames}\PY{p}{(}Mpeaks\PY{p}{)}\PY{p}{)}
			Mpeaklist\PY{o}{\PYZlt{}\PYZhy{}}peaklistlist\PY{p}{[}patsToGetM\PY{p}{]}
		\PY{p}{\PYZcb{}}
		
		\PY{c+c1}{\PYZsh{}\PYZsh{}\PYZsh{} use Wmatrix() function}
		Wm\PY{o}{\PYZlt{}\PYZhy{}}Wmatrix\PY{p}{(}Npeaks\PY{p}{,}Npeaklist\PY{p}{,}Mpeaks\PY{p}{,}Mpeaklist\PY{p}{,}D\PY{p}{,}expon\PY{p}{,}lambda\PY{p}{)}
		\PY{c+c1}{\PYZsh{}\PYZsh{}\PYZsh{} use S\PYZhy{}W alignment function to estimate maximum path}
		\PY{c+c1}{\PYZsh{}\PYZsh{}\PYZsh{} see: https://code.google.com/p/swalign/}
		estPM\PY{o}{\PYZlt{}\PYZhy{}}SWalign\PY{p}{(}Wm\PY{p}{,}G\PY{p}{,}maxM\PY{p}{)} 
		\PY{c+c1}{\PYZsh{}\PYZsh{}\PYZsh{} estPM is a data.frame of (i,j) locations of the maximum path}
		\PY{c+c1}{\PYZsh{}\PYZsh{}\PYZsh{} the data.frame is 2 columns for i,j points}
		
		nN\PY{o}{\PYZlt{}\PYZhy{}}\PY{k+kp}{ncol}\PY{p}{(}Npeaks\PY{p}{)} \PY{c+c1}{\PYZsh{}\PYZsh{}\PYZsh{} no. of peaks in N\PYZhy{}align}
		nM\PY{o}{\PYZlt{}\PYZhy{}}\PY{k+kp}{ncol}\PY{p}{(}Mpeaks\PY{p}{)} \PY{c+c1}{\PYZsh{}\PYZsh{}\PYZsh{} no. of peaks in M\PYZhy{}align}
		nK\PY{o}{\PYZlt{}\PYZhy{}}\PY{k+kp}{nrow}\PY{p}{(}estPM\PY{p}{)}  \PY{c+c1}{\PYZsh{}\PYZsh{}\PYZsh{} no. of peaks in new N:M\PYZhy{}align}
		\PY{c+c1}{\PYZsh{}\PYZsh{}\PYZsh{} apllTemp: }
		\PY{c+c1}{\PYZsh{}\PYZsh{}\PYZsh{}      (a)lignment of (p)eak (l)ist (l)ist, (temp)orary}
		\PY{c+c1}{\PYZsh{}\PYZsh{}\PYZsh{} Matrix of peak indicators. The $n_K$ rows represent the $n_K$ peaks }
		\PY{c+c1}{\PYZsh{}\PYZsh{}\PYZsh{}    from the N:M\PYZhy{}alignment. }
		\PY{c+c1}{\PYZsh{}\PYZsh{}\PYZsh{} Entries apllTemp[i,j] are ==}
		\PY{c+c1}{\PYZsh{}\PYZsh{}\PYZsh{} \PYZob{} 0 if that N:M\PYZhy{}aligned peak does not exist in spec $j$ (column $j$)}
		\PY{c+c1}{\PYZsh{}\PYZsh{}\PYZsh{} \PYZob{} \PYZus{}else\PYZus{} a non\PYZhy{}zero indicator, the peak number from within the  }
		\PY{c+c1}{\PYZsh{}\PYZsh{}\PYZsh{}                           1\PYZhy{}alignment from spectrum $j$ (column $j$)}
		apllTemp\PY{o}{\PYZlt{}\PYZhy{}}\PY{k+kt}{matrix}\PY{p}{(}\PY{l+m}{0}\PY{p}{,}nrow\PY{o}{=}nK\PY{p}{,}ncol\PY{o}{=}nN\PY{o}{+}nM\PY{p}{)}
		mzValsTemp\PY{o}{\PYZlt{}\PYZhy{}}\PY{k+kc}{NULL}
		AveMzValsTemp\PY{o}{\PYZlt{}\PYZhy{}}\PY{k+kc}{NULL}
		\PY{k+kr}{for}\PY{p}{(}n.k \PY{k+kr}{in} \PY{l+m}{1}\PY{o}{:}nK\PY{p}{)}
		\PY{p}{\PYZob{}}
			\PY{k+kr}{if}\PY{p}{(}estPM\PY{p}{[}n.k\PY{p}{,}\PY{l+m}{1}\PY{p}{]}\PY{o}{\PYZgt{}}\PY{l+m}{0}\PY{p}{)} \PY{c+c1}{\PYZsh{} if the peak exists in the N\PYZhy{}alignment}
			\PY{p}{\PYZob{}}
				\PY{c+c1}{\PYZsh{} transfer peak info from N\PYZhy{}align to new N:M\PYZhy{}align matrix}
				apllTemp\PY{p}{[}n.k\PY{p}{,}\PY{l+m}{1}\PY{o}{:}nN\PY{p}{]}\PY{o}{\PYZlt{}\PYZhy{}}Npeaks\PY{p}{[}estPM\PY{p}{[}n.k\PY{p}{,}\PY{l+m}{1}\PY{p}{]}\PY{p}{,}\PY{p}{]}
				\PY{k+kr}{for}\PY{p}{(}i \PY{k+kr}{in} \PY{l+m}{1}\PY{o}{:}nN\PY{p}{)} \PY{k+kr}{if}\PY{p}{(}apllTemp\PY{p}{[}n.k\PY{p}{,}i\PY{p}{]}\PY{o}{\PYZgt{}}\PY{l+m}{0}\PY{p}{)} mzValsTemp\PY{o}{\PYZlt{}\PYZhy{}}
						\PY{k+kt}{c}\PY{p}{(}mzValsTemp\PY{p}{,}Npeaklist\PY{p}{[[}i\PY{p}{]]}\PY{p}{[}\PY{l+m}{1}\PY{p}{,}apllTemp\PY{p}{[}n.k\PY{p}{,}i\PY{p}{]]}\PY{p}{)}
			\PY{p}{\PYZcb{}}
			\PY{k+kr}{if}\PY{p}{(}estPM\PY{p}{[}n.k\PY{p}{,}\PY{l+m}{2}\PY{p}{]}\PY{o}{\PYZgt{}}\PY{l+m}{0}\PY{p}{)} \PY{c+c1}{\PYZsh{} if the peak exists in the M\PYZhy{}alignment}
			\PY{p}{\PYZob{}}
				\PY{c+c1}{\PYZsh{} transfer peak info from N\PYZhy{}align to new N:M\PYZhy{}align matrix}
				apllTemp\PY{p}{[}n.k\PY{p}{,}\PY{p}{(}nN\PY{l+m}{+1}\PY{p}{)}\PY{o}{:}\PY{p}{(}nN\PY{o}{+}nM\PY{p}{)}\PY{p}{]}\PY{o}{\PYZlt{}\PYZhy{}}Mpeaks\PY{p}{[}estPM\PY{p}{[}n.k\PY{p}{,}\PY{l+m}{2}\PY{p}{]}\PY{p}{,}\PY{p}{]}
				\PY{k+kr}{for}\PY{p}{(}i \PY{k+kr}{in} \PY{p}{(}nN\PY{l+m}{+1}\PY{p}{)}\PY{o}{:}\PY{p}{(}nN\PY{o}{+}nM\PY{p}{)}\PY{p}{)} \PY{k+kr}{if}\PY{p}{(}apllTemp\PY{p}{[}n.k\PY{p}{,}i\PY{p}{]}\PY{o}{\PYZgt{}}\PY{l+m}{0}\PY{p}{)} mzValsTemp\PY{o}{\PYZlt{}\PYZhy{}}
						\PY{k+kt}{c}\PY{p}{(}mzValsTemp\PY{p}{,}Mpeaklist\PY{p}{[[}i\PY{o}{\PYZhy{}}nN\PY{p}{]]}\PY{p}{[}\PY{l+m}{1}\PY{p}{,}apllTemp\PY{p}{[}n.k\PY{p}{,}i\PY{p}{]]}\PY{p}{)}
			\PY{p}{\PYZcb{}}
			\PY{c+c1}{\PYZsh{} get ave m/z of all aligned peaks}
			AveMzValsTemp\PY{o}{\PYZlt{}\PYZhy{}}\PY{k+kt}{c}\PY{p}{(}AveMzValsTemp\PY{p}{,}\PY{k+kp}{mean}\PY{p}{(}mzValsTemp\PY{p}{)}\PY{p}{)} 
			mzValsTemp\PY{o}{\PYZlt{}\PYZhy{}}\PY{k+kc}{NULL}
		\PY{p}{\PYZcb{}}
		\PY{c+c1}{\PYZsh{}\PYZsh{}\PYZsh{} change row order if averaging m/z has changed peak location order}
		mzReOrder\PY{o}{\PYZlt{}\PYZhy{}}\PY{k+kp}{order}\PY{p}{(}AveMzValsTemp\PY{p}{)}
		apllTemp\PY{o}{\PYZlt{}\PYZhy{}}apllTemp\PY{p}{[}mzReOrder\PY{p}{,}\PY{p}{]}
		
		allPat\PY{o}{\PYZlt{}\PYZhy{}}\PY{k+kt}{c}\PY{p}{(}patsToGetN\PY{p}{,}patsToGetM\PY{p}{)}
		\PY{k+kp}{colnames}\PY{p}{(}apllTemp\PY{p}{)}\PY{o}{\PYZlt{}\PYZhy{}}allPat
		\PY{c+c1}{\PYZsh{}\PYZsh{}\PYZsh{} clean up N:M\PYZhy{}alignment to preserve spectrum order}
		patOrder\PY{o}{\PYZlt{}\PYZhy{}}\PY{k+kp}{order}\PY{p}{(}allPat\PY{p}{)}
		apllTemp\PY{o}{\PYZlt{}\PYZhy{}}apllTemp\PY{p}{[}\PY{p}{,}patOrder\PY{p}{]}
		
		a.pll\PY{p}{[[}aindx\PY{p}{]]}\PY{o}{\PYZlt{}\PYZhy{}}apllTemp
	\PY{p}{\PYZcb{}}
	\PY{c+c1}{\PYZsh{}\PYZsh{}\PYZsh{} return list() object of peak amalgamation/alignment,}
	\PY{c+c1}{\PYZsh{}\PYZsh{}\PYZsh{}     including intermediate steps}
	outlist\PY{o}{\PYZlt{}\PYZhy{}}\PY{k+kt}{list}\PY{p}{(}dendro\PY{o}{=}hc\PY{p}{,}stepwise.peaks\PY{o}{=}a.pll\PY{p}{[}\PY{o}{\PYZhy{}}nAmal\PY{p}{]}\PY{p}{,}amalpeaks\PY{o}{=}a.pll\PY{p}{[[}nAmal\PY{p}{]]}\PY{p}{)}
	\PY{k+kr}{return}\PY{p}{(}outlist\PY{p}{)}

\PY{p}{\PYZcb{}}
\end{Verbatim}







	
	
\clearpage

\section{Surrogate variable analysis} \label{sva}


\codetab{
\codeentry{Get SVA adjusted expression matrix}{08\_do\_sva.R}{doSVA()}
}

\clearpage

\subsection{Get SVA adjusted expression matrix}

Please note the function \texttt{getH()} (line 47 below) is the code available in the \texttt{DanteR} package to determine the number of significant surrogate variables. The function \texttt{mulReg(Y,X)} performs sequential linear regressions on the columns of the input \texttt{Y} using a fixed effects design matrix \texttt{X}. \texttt{mulReg()} returns a list containing the following vectors and matrices: \texttt{RES}$_{n \times P}$, residual matrix after \texttt{Y} has been regressed; \texttt{BETA}$_{d \times P}$, matrix of the regression coefficients, $P$ columns for each regression; \texttt{TVALS}$_{d \times P}$, the corresponding $t$-statistics; \texttt{PVALS}$_{d \times P}$ the corresponding $p$-values of \texttt{TVALS}; \texttt{FPVALS}$_{P\times 1}$, $p$-value for each linear regression corresponding to the null model $F$-statistic.

	\begin{Verbatim}[commandchars=\\\{\},codes={\catcode`\$=3\catcode`\^=7\catcode`\_=8},gobble=0,numbers=left,fontfamily=fvm,fontshape=n,fontsize=\footnotesize,tabsize=2]
\PY{c+c1}{\PYZsh{}\PYZsh{}\PYZsh{}\PYZsh{}\PYZsh{}\PYZsh{}\PYZsh{}\PYZsh{}\PYZsh{}\PYZsh{}\PYZsh{}\PYZsh{}\PYZsh{}\PYZsh{}\PYZsh{}\PYZsh{}\PYZsh{}\PYZsh{}\PYZsh{}\PYZsh{}\PYZsh{}\PYZsh{}\PYZsh{}\PYZsh{}\PYZsh{}\PYZsh{}\PYZsh{}\PYZsh{}\PYZsh{}\PYZsh{}\PYZsh{}\PYZsh{}\PYZsh{}\PYZsh{}\PYZsh{}\PYZsh{} FUNCTION \PYZsh{}\PYZsh{}\PYZsh{}\PYZsh{}\PYZsh{}\PYZsh{}\PYZsh{}\PYZsh{}\PYZsh{}\PYZsh{}\PYZsh{}\PYZsh{}\PYZsh{}\PYZsh{}\PYZsh{}\PYZsh{}\PYZsh{}\PYZsh{}\PYZsh{}\PYZsh{}\PYZsh{}\PYZsh{}\PYZsh{}\PYZsh{}\PYZsh{}\PYZsh{}\PYZsh{}\PYZsh{}\PYZsh{}\PYZsh{}\PYZsh{}\PYZsh{}}
\PY{c+c1}{\PYZsh{}\PYZsh{}\PYZsh{}\PYZsh{} doSVA: Perform SVA using the model:}
\PY{c+c1}{\PYZsh{}\PYZsh{}\PYZsh{}\PYZsh{}        $Y_j = \mu_j + X\alpha_j + Z\beta_j + W\delta_j + \mathbf{e}_j$}
\PY{c+c1}{\PYZsh{}\PYZsh{}\PYZsh{}\PYZsh{}\PYZsh{}\PYZsh{}\PYZsh{}\PYZsh{}\PYZsh{}\PYZsh{}\PYZsh{}\PYZsh{}\PYZsh{}\PYZsh{}\PYZsh{}\PYZsh{}\PYZsh{}\PYZsh{}\PYZsh{}\PYZsh{}\PYZsh{}\PYZsh{}\PYZsh{}\PYZsh{}\PYZsh{}\PYZsh{}\PYZsh{}\PYZsh{}\PYZsh{}\PYZsh{}\PYZsh{}\PYZsh{}\PYZsh{}\PYZsh{}\PYZsh{}\PYZsh{} INPUTS \PYZsh{}\PYZsh{}\PYZsh{}\PYZsh{}\PYZsh{}\PYZsh{}\PYZsh{}\PYZsh{}\PYZsh{}\PYZsh{}\PYZsh{}\PYZsh{}\PYZsh{}\PYZsh{}\PYZsh{}\PYZsh{}\PYZsh{}\PYZsh{}\PYZsh{}\PYZsh{}\PYZsh{}\PYZsh{}\PYZsh{}\PYZsh{}\PYZsh{}\PYZsh{}\PYZsh{}\PYZsh{}\PYZsh{}\PYZsh{}\PYZsh{}\PYZsh{}\PYZsh{}\PYZsh{}}
\PY{c+c1}{\PYZsh{}\PYZsh{}\PYZsh{}\PYZsh{} Y: is a $n \times p$ matrix, where each p columns are regressed}
\PY{c+c1}{\PYZsh{}\PYZsh{}\PYZsh{}\PYZsh{} Intecept: boolean; do we want to fit a mean value? (yes, in most cases)}
\PY{c+c1}{\PYZsh{}\PYZsh{}\PYZsh{}\PYZsh{} X: is a $n \times d_{\alpha}$ design matrix of the factors of interest}
\PY{c+c1}{\PYZsh{}\PYZsh{}\PYZsh{}\PYZsh{} Z: is a $n \times d_{\beta}$ design matrix of the incidental experimental factors}
\PY{c+c1}{\PYZsh{}\PYZsh{}\PYZsh{}\PYZsh{} nosigsv: the number (referred to as $H$ in some papers) of significant }
\PY{c+c1}{\PYZsh{}\PYZsh{}\PYZsh{}\PYZsh{}         eigen vecs if $NULL$, the function will determine. If less than }
\PY{c+c1}{\PYZsh{}\PYZsh{}\PYZsh{}\PYZsh{}         $1$, no $W$ computed}
\PY{c+c1}{\PYZsh{}\PYZsh{}\PYZsh{}\PYZsh{} verbose: boolean, whether the surragate variable matrix, $W$ is returned}
\PY{c+c1}{\PYZsh{}\PYZsh{}\PYZsh{}\PYZsh{} seed: an integer to feed into \PYZsq{}set.seed()\PYZsq{} for reproducable results}
\PY{c+c1}{\PYZsh{}\PYZsh{}\PYZsh{}\PYZsh{}\PYZsh{}\PYZsh{}\PYZsh{}\PYZsh{}\PYZsh{}\PYZsh{}\PYZsh{}\PYZsh{}\PYZsh{}\PYZsh{}\PYZsh{}\PYZsh{}\PYZsh{}\PYZsh{}\PYZsh{}\PYZsh{}\PYZsh{}\PYZsh{}\PYZsh{}\PYZsh{}\PYZsh{}\PYZsh{}\PYZsh{}\PYZsh{}\PYZsh{}\PYZsh{}\PYZsh{}\PYZsh{}\PYZsh{}\PYZsh{}\PYZsh{}\PYZsh{} OUTPUTS \PYZsh{}\PYZsh{}\PYZsh{}\PYZsh{}\PYZsh{}\PYZsh{}\PYZsh{}\PYZsh{}\PYZsh{}\PYZsh{}\PYZsh{}\PYZsh{}\PYZsh{}\PYZsh{}\PYZsh{}\PYZsh{}\PYZsh{}\PYZsh{}\PYZsh{}\PYZsh{}\PYZsh{}\PYZsh{}\PYZsh{}\PYZsh{}\PYZsh{}\PYZsh{}\PYZsh{}\PYZsh{}\PYZsh{}\PYZsh{}\PYZsh{}\PYZsh{}\PYZsh{}}
\PY{c+c1}{\PYZsh{}\PYZsh{}\PYZsh{}\PYZsh{} Ytilde: the Y matrix with $Z\beta_j + W\delta_j$ removed}
\PY{c+c1}{\PYZsh{}\PYZsh{}\PYZsh{}\PYZsh{} pvals: the p\PYZhy{}values of Ytilde regressed on $\mu_j + X\alpha_j + Z\beta_j + W\delta_j$}
\PY{c+c1}{\PYZsh{}\PYZsh{}\PYZsh{}\PYZsh{} tvals: the corresponding t\PYZhy{}statistics}
\PY{c+c1}{\PYZsh{}\PYZsh{}\PYZsh{}\PYZsh{} betas: the corresponding $\alpha_j,\beta_j,\delta_j$ estimates}
\PY{c+c1}{\PYZsh{}\PYZsh{}\PYZsh{}\PYZsh{} paramlabels: a combination of I (intercept), X, Z, W to signify the }
\PY{c+c1}{\PYZsh{}\PYZsh{}\PYZsh{}\PYZsh{}         relevent rows of p\PYZhy{}vals/tvals/betas}
\PY{c+c1}{\PYZsh{}\PYZsh{}\PYZsh{}\PYZsh{} W: the eigen vectors matrix}
\PY{c+c1}{\PYZsh{}\PYZsh{}\PYZsh{}\PYZsh{} H: the number of columns of W (used eigen\PYZhy{}vectors)}
\PY{c+c1}{\PYZsh{}\PYZsh{}\PYZsh{}\PYZsh{}\PYZsh{}\PYZsh{}\PYZsh{}\PYZsh{}\PYZsh{}\PYZsh{}\PYZsh{}\PYZsh{}\PYZsh{}\PYZsh{}\PYZsh{}\PYZsh{}\PYZsh{}\PYZsh{}\PYZsh{}\PYZsh{}\PYZsh{}\PYZsh{}\PYZsh{}\PYZsh{}\PYZsh{}\PYZsh{}\PYZsh{}\PYZsh{}\PYZsh{}\PYZsh{}\PYZsh{}\PYZsh{}\PYZsh{}\PYZsh{}\PYZsh{}\PYZsh{}\PYZsh{}\PYZsh{}\PYZsh{}\PYZsh{}\PYZsh{}\PYZsh{}\PYZsh{}\PYZsh{}\PYZsh{}\PYZsh{}\PYZsh{}\PYZsh{}\PYZsh{}\PYZsh{}\PYZsh{}\PYZsh{}\PYZsh{}\PYZsh{}\PYZsh{}\PYZsh{}\PYZsh{}\PYZsh{}\PYZsh{}\PYZsh{}\PYZsh{}\PYZsh{}\PYZsh{}\PYZsh{}\PYZsh{}\PYZsh{}\PYZsh{}\PYZsh{}\PYZsh{}\PYZsh{}\PYZsh{}\PYZsh{}\PYZsh{}\PYZsh{}\PYZsh{}\PYZsh{}\PYZsh{}\PYZsh{}\PYZsh{}}
doSVA\PY{o}{\PYZlt{}\PYZhy{}}\PY{k+kr}{function}\PY{p}{(}
	Y\PY{p}{,}Intercept\PY{o}{=}\PY{k+kc}{TRUE}\PY{p}{,}X\PY{o}{=}\PY{k+kc}{NULL}\PY{p}{,}Z\PY{o}{=}\PY{k+kc}{NULL}\PY{p}{,}nosigsv\PY{o}{=}\PY{k+kc}{NULL}\PY{p}{,}verbose\PY{o}{=}\PY{k+kc}{FALSE}\PY{p}{,}seed\PY{o}{=}\PY{k+kc}{NULL}
\PY{p}{)}\PY{p}{\PYZob{}}
	n\PY{o}{\PYZlt{}\PYZhy{}}\PY{k+kp}{nrow}\PY{p}{(}Y\PY{p}{)}
	thisInt\PY{o}{\PYZlt{}\PYZhy{}}IXZ\PY{o}{\PYZlt{}\PYZhy{}}\PY{k+kc}{NULL}
	\PY{k+kr}{if}\PY{p}{(}Intercept\PY{p}{)} thisInt\PY{o}{\PYZlt{}\PYZhy{}}\PY{k+kt}{matrix}\PY{p}{(}\PY{l+m}{1}\PY{p}{,}nrow\PY{o}{=}n\PY{p}{,}ncol\PY{o}{=}\PY{l+m}{1}\PY{p}{,}dimnames\PY{o}{=}\PY{k+kt}{list}\PY{p}{(}\PY{k+kc}{NULL}\PY{p}{,}\PY{l+s}{\PYZdq{}}\PY{l+s}{Intcpt\PYZdq{}}\PY{p}{)}\PY{p}{)}
	\PY{k+kr}{if}\PY{p}{(}\PY{k+kp}{is.null}\PY{p}{(}thisInt\PY{p}{)} \PY{o}{\PYZam{}\PYZam{}} \PY{k+kp}{is.null}\PY{p}{(}X\PY{p}{)} \PY{o}{\PYZam{}\PYZam{}} \PY{k+kp}{is.null}\PY{p}{(}Z\PY{p}{)}\PY{p}{)}
	\PY{p}{\PYZob{}}
		\PY{k+kp}{cat}\PY{p}{(}\PY{l+s}{\PYZdq{}}\PY{l+s}{At least one of: Intercept, X and Z must be specified \PYZbs{}n\PYZdq{}}\PY{p}{)}
		\PY{k+kr}{return}\PY{p}{(}\PY{k+kc}{NULL}\PY{p}{)}
	\PY{p}{\PYZcb{}} \PY{k+kr}{else} IXZ\PY{o}{\PYZlt{}\PYZhy{}}\PY{k+kp}{cbind}\PY{p}{(}thisInt\PY{p}{,}X\PY{p}{,}Z\PY{p}{)}
	kparam\PY{o}{\PYZlt{}\PYZhy{}}\PY{k+kp}{ncol}\PY{p}{(}IXZ\PY{p}{)}
	colmarkers\PY{o}{\PYZlt{}\PYZhy{}}\PY{k+kp}{rep}\PY{p}{(}\PY{l+s}{\PYZdq{}}\PY{l+s}{\PYZdq{}}\PY{p}{,}kparam\PY{p}{)}
	indx\PY{o}{\PYZlt{}\PYZhy{}}\PY{l+m}{0}
	\PY{k+kr}{if}\PY{p}{(}\PY{o}{!}\PY{k+kp}{is.null}\PY{p}{(}thisInt\PY{p}{)}\PY{p}{)} colmarkers\PY{p}{[}indx\PY{o}{\PYZlt{}\PYZhy{}}indx\PY{l+m}{+1}\PY{p}{]}\PY{o}{\PYZlt{}\PYZhy{}}\PY{l+s}{\PYZdq{}}\PY{l+s}{I\PYZdq{}}
	\PY{k+kr}{if}\PY{p}{(}\PY{o}{!}\PY{k+kp}{is.null}\PY{p}{(}X\PY{p}{)}\PY{p}{)} colmarkers\PY{p}{[}\PY{p}{(}indx\PY{o}{\PYZlt{}\PYZhy{}}indx\PY{l+m}{+1}\PY{p}{)}\PY{o}{:}\PY{p}{(}indx\PY{o}{\PYZlt{}\PYZhy{}}indx\PY{o}{+}\PY{k+kp}{ncol}\PY{p}{(}X\PY{p}{)}\PY{l+m}{\PYZhy{}1}\PY{p}{)}\PY{p}{]}\PY{o}{\PYZlt{}\PYZhy{}}\PY{l+s}{\PYZdq{}}\PY{l+s}{X\PYZdq{}}
	\PY{k+kr}{if}\PY{p}{(}\PY{o}{!}\PY{k+kp}{is.null}\PY{p}{(}Z\PY{p}{)}\PY{p}{)} colmarkers\PY{p}{[}\PY{p}{(}indx\PY{o}{\PYZlt{}\PYZhy{}}indx\PY{l+m}{+1}\PY{p}{)}\PY{o}{:}\PY{p}{(}indx\PY{o}{\PYZlt{}\PYZhy{}}indx\PY{o}{+}\PY{k+kp}{ncol}\PY{p}{(}Z\PY{p}{)}\PY{l+m}{\PYZhy{}1}\PY{p}{)}\PY{p}{]}\PY{o}{\PYZlt{}\PYZhy{}}\PY{l+s}{\PYZdq{}}\PY{l+s}{Z\PYZdq{}}

	RIXZ\PY{o}{\PYZlt{}\PYZhy{}}multReg\PY{p}{(}Y\PY{p}{,}IXZ\PY{p}{,}createNAvals\PY{o}{=}\PY{k+kc}{TRUE}\PY{p}{,}seed\PY{o}{=}seed\PY{p}{)} 
	thissvd\PY{o}{\PYZlt{}\PYZhy{}}\PY{k+kp}{svd}\PY{p}{(}RIXZ\PY{o}{\PYZdl{}}RES\PY{p}{)}
	
	W\PY{o}{\PYZlt{}\PYZhy{}}H\PY{o}{\PYZlt{}\PYZhy{}}\PY{k+kc}{NULL}
	\PY{k+kr}{if}\PY{p}{(}\PY{k+kp}{is.null}\PY{p}{(}nosigsv\PY{p}{)}\PY{p}{)}\PY{p}{\PYZob{}}
		H\PY{o}{\PYZlt{}\PYZhy{}}getH\PY{p}{(}RIXZ\PY{o}{\PYZdl{}}RES\PY{p}{,}IXZ\PY{p}{,}nullsig\PY{o}{=}\PY{l+m}{0.1}\PY{p}{,}verbose\PY{o}{=}\PY{k+kc}{FALSE}\PY{p}{)}
		\PY{k+kr}{if}\PY{p}{(}H\PY{o}{\PYZlt{}}\PY{l+m}{1}\PY{p}{)} \PY{k+kp}{cat}\PY{p}{(}\PY{l+s}{\PYZdq{}}\PY{l+s}{No significant surrogate variables found \PYZbs{}n\PYZdq{}}\PY{p}{)}
	\PY{p}{\PYZcb{}}\PY{k+kp}{else}\PY{p}{\PYZob{}}
		H\PY{o}{\PYZlt{}\PYZhy{}}nosigsv
	\PY{p}{\PYZcb{}}
	\PY{k+kr}{if}\PY{p}{(}H\PY{o}{\PYZlt{}}\PY{l+m}{1}\PY{p}{)}\PY{p}{\PYZob{}}
		\PY{k+kp}{cat}\PY{p}{(}\PY{l+s}{\PYZdq{}}\PY{l+s}{No surrogate variables will be used \PYZbs{}n\PYZdq{}}\PY{p}{)}
	\PY{p}{\PYZcb{}}\PY{k+kp}{else}\PY{p}{\PYZob{}}
		\PY{k+kp}{cat}\PY{p}{(}\PY{l+s}{\PYZdq{}}\PY{l+s}{Using H=\PYZdq{}}\PY{p}{,}H\PY{p}{,}\PY{l+s}{\PYZdq{}}\PY{l+s}{ significant surrogate variables \PYZbs{}n\PYZdq{}}\PY{p}{,}sep\PY{o}{=}\PY{l+s}{\PYZdq{}}\PY{l+s}{\PYZdq{}}\PY{p}{)}
		W\PY{o}{\PYZlt{}\PYZhy{}}\PY{k+kp}{as.matrix}\PY{p}{(}thissvd\PY{o}{\PYZdl{}}u\PY{p}{[}\PY{p}{,}\PY{l+m}{1}\PY{o}{:}H\PY{p}{]}\PY{p}{)} 
		\PY{k+kp}{colnames}\PY{p}{(}W\PY{p}{)}\PY{o}{\PYZlt{}\PYZhy{}}\PY{k+kp}{paste}\PY{p}{(}\PY{l+s}{\PYZdq{}}\PY{l+s}{W\PYZdq{}}\PY{p}{,}\PY{l+m}{1}\PY{o}{:}H\PY{p}{,}sep\PY{o}{=}\PY{l+s}{\PYZdq{}}\PY{l+s}{\PYZdq{}}\PY{p}{)}
		colmarkers\PY{o}{\PYZlt{}\PYZhy{}}\PY{k+kt}{c}\PY{p}{(}colmarkers\PY{p}{,}\PY{k+kp}{rep}\PY{p}{(}\PY{l+s}{\PYZdq{}}\PY{l+s}{W\PYZdq{}}\PY{p}{,}H\PY{p}{)}\PY{p}{)}
	\PY{p}{\PYZcb{}}
	IXZW\PY{o}{\PYZlt{}\PYZhy{}}\PY{k+kp}{cbind}\PY{p}{(}IXZ\PY{p}{,}W\PY{p}{)}
	Rtilde\PY{o}{\PYZlt{}\PYZhy{}}multReg\PY{p}{(}Y\PY{p}{,}IXZW\PY{p}{)}
	removecols\PY{o}{\PYZlt{}\PYZhy{}}colmarkers \PY{o}{\PYZpc{}in\PYZpc{}} \PY{k+kt}{c}\PY{p}{(}\PY{l+s}{\PYZdq{}}\PY{l+s}{Z\PYZdq{}}\PY{p}{,}\PY{l+s}{\PYZdq{}}\PY{l+s}{W\PYZdq{}}\PY{p}{)}
	ZBetaWDelta\PY{o}{\PYZlt{}\PYZhy{}}\PY{l+m}{0}
	\PY{k+kr}{if}\PY{p}{(}\PY{k+kp}{sum}\PY{p}{(}removecols\PY{p}{)}\PY{p}{)} ZBetaWDelta\PY{o}{\PYZlt{}\PYZhy{}}\PY{k+kp}{as.matrix}\PY{p}{(}IXZW\PY{p}{[}\PY{p}{,}removecols\PY{p}{]}\PY{p}{)} \PY{o}{\PYZpc{}*\PYZpc{}} 
								\PY{k+kp}{as.matrix}\PY{p}{(}Rtilde\PY{o}{\PYZdl{}}BETA\PY{p}{[}removecols\PY{p}{,}\PY{p}{]}\PY{p}{)}
	Ytilde\PY{o}{\PYZlt{}\PYZhy{}}Y\PY{o}{\PYZhy{}}ZBetaWDelta
	\PY{k+kr}{if}\PY{p}{(}verbose\PY{p}{)} \PY{k+kr}{return}\PY{p}{(}\PY{k+kt}{list}\PY{p}{(}Ytilde\PY{o}{=}Ytilde\PY{p}{,}paramlabels\PY{o}{=}colmarkers\PY{p}{,}W\PY{o}{=}W\PY{p}{,}H\PY{o}{=}H\PY{p}{)}\PY{p}{)} 
	\PY{k+kr}{else} \PY{k+kr}{return}\PY{p}{(}Ytilde\PY{p}{)}
\PY{p}{\PYZcb{}}
\end{Verbatim}

	
	
	
	
	
\clearpage



\section{Pairwise fusion linear discriminant analysis} 

\codetab{
\codeentry{Create a PFDA object}{09\_create\_pfda\_obj.R}{create\_pfda\_obj()}
\codeentry{Predict class for new data and a PFDA object}{10\_pfda\_predict.R}{pfda\_predict()}
}


\clearpage

\subsection{Create a PFDA object}

	\begin{Verbatim}[commandchars=\\\{\},codes={\catcode`\$=3\catcode`\^=7\catcode`\_=8},gobble=0,numbers=left,fontfamily=fvm,fontshape=n,fontsize=\footnotesize,tabsize=2]
\PY{c+c1}{\PYZsh{}\PYZsh{}\PYZsh{}\PYZsh{}\PYZsh{}\PYZsh{}\PYZsh{} FUNCTION: createPFldaobj()}
\PY{c+c1}{\PYZsh{}\PYZsh{}\PYZsh{} estimate parameters of PF\PYZhy{}DA model, so that a discrim function created}

\PY{c+c1}{\PYZsh{}\PYZsh{}\PYZsh{}\PYZsh{}\PYZsh{}\PYZsh{}\PYZsh{} input:}
\PY{c+c1}{\PYZsh{}\PYZsh{}\PYZsh{} X: a n x p matrix, of n obs and p variables}
\PY{c+c1}{\PYZsh{}\PYZsh{}\PYZsh{} Xclass: a vector of length n of the classes (must be a factor variable)}
\PY{c+c1}{\PYZsh{}\PYZsh{}\PYZsh{} priors: a vector of length K (\PYZsh{}classes) with elements in (0,1)}

createPFldaobj\PY{o}{\PYZlt{}\PYZhy{}}\PY{k+kr}{function}\PY{p}{(}X\PY{p}{,}Xclass\PY{p}{,}lambdar\PY{o}{=}\PY{l+m}{1}\PY{p}{,}priors\PY{o}{=}\PY{k+kc}{NULL}\PY{p}{,}alph\PY{o}{=}\PY{k+kc}{NULL}\PY{p}{,}wts\PY{o}{=}\PY{k+kc}{NULL}\PY{p}{)}
\PY{p}{\PYZob{}}
  N\PY{o}{\PYZlt{}\PYZhy{}}\PY{k+kp}{length}\PY{p}{(}Xclass\PY{p}{)}
  P\PY{o}{\PYZlt{}\PYZhy{}}\PY{k+kp}{ncol}\PY{p}{(}X\PY{p}{)}
  nks\PY{o}{\PYZlt{}\PYZhy{}}\PY{k+kp}{table}\PY{p}{(}Xclass\PY{p}{)}
  classnames\PY{o}{\PYZlt{}\PYZhy{}}\PY{k+kp}{levels}\PY{p}{(}Xclass\PY{p}{)}
  K\PY{o}{\PYZlt{}\PYZhy{}}\PY{k+kp}{length}\PY{p}{(}classnames\PY{p}{)}
  
  \PY{c+c1}{\PYZsh{}\PYZsh{}\PYZsh{} if not supplied, make $\hat{\pi}_k$ data proportions}
  \PY{k+kr}{if}\PY{p}{(}\PY{k+kp}{is.null}\PY{p}{(}priors\PY{p}{)}\PY{p}{)} priors\PY{o}{\PYZlt{}\PYZhy{}}nks\PY{o}{/}N
  
  \PY{k+kr}{if}\PY{p}{(}\PY{k+kp}{length}\PY{p}{(}priors\PY{p}{)}\PY{o}{!=}K\PY{p}{)}\PY{p}{\PYZob{}}
    \PY{k+kp}{cat}\PY{p}{(}\PY{l+s}{\PYZdq{}}\PY{l+s}{The length of priors and the total number }
\PY{l+s}{        of groups must be equal \PYZbs{}n\PYZdq{}}\PY{p}{)}
    \PY{k+kr}{return}\PY{p}{(}\PY{k+kc}{NULL}\PY{p}{)}
  \PY{p}{\PYZcb{}}\PY{k+kr}{else} \PY{k+kr}{if}\PY{p}{(}\PY{k+kp}{is.null}\PY{p}{(}alph\PY{p}{)} \PY{o}{\PYZam{}} \PY{p}{(}N\PY{o}{\PYZlt{}}P\PY{p}{)}\PY{p}{)}\PY{p}{\PYZob{}}
    \PY{k+kp}{cat}\PY{p}{(}\PY{l+s}{\PYZdq{}}\PY{l+s}{Alpha is suggested for n\PYZlt{}p data \PYZbs{}n\PYZdq{}}\PY{p}{)}
  \PY{p}{\PYZcb{}}\PY{k+kr}{else} \PY{k+kr}{if}\PY{p}{(}N\PY{o}{!=}\PY{k+kp}{nrow}\PY{p}{(}X\PY{p}{)}\PY{p}{)}\PY{p}{\PYZob{}}
    \PY{k+kp}{cat}\PY{p}{(}\PY{l+s}{\PYZdq{}}\PY{l+s}{The length of Xclass and the number }
\PY{l+s}{        of rows in X must agree \PYZbs{}n\PYZdq{}}\PY{p}{)}
    \PY{k+kr}{return}\PY{p}{(}\PY{k+kc}{NULL}\PY{p}{)}
  \PY{p}{\PYZcb{}}\PY{k+kr}{else} \PY{k+kr}{if}\PY{p}{(}\PY{o}{!}\PY{k+kp}{all}\PY{p}{(}nks\PY{o}{\PYZgt{}}\PY{l+m}{1}\PY{p}{)}\PY{p}{)}\PY{p}{\PYZob{}}
    \PY{k+kp}{cat}\PY{p}{(}\PY{l+s}{\PYZdq{}}\PY{l+s}{There needs to be at least two obs in each }
\PY{l+s}{        group for variances to be computed \PYZbs{}n\PYZdq{}}\PY{p}{)}
    \PY{k+kr}{return}\PY{p}{(}\PY{k+kc}{NULL}\PY{p}{)}
  \PY{p}{\PYZcb{}}

  Xclassint\PY{o}{\PYZlt{}\PYZhy{}}\PY{k+kp}{as.integer}\PY{p}{(}Xclass\PY{p}{)}
  transMeans\PY{o}{\PYZlt{}\PYZhy{}}\PY{k+kp}{colMeans}\PY{p}{(}X\PY{p}{)}
  X\PY{o}{\PYZlt{}\PYZhy{}}X\PY{o}{\PYZhy{}}\PY{k+kt}{matrix}\PY{p}{(}\PY{k+kp}{rep}\PY{p}{(}transMeans\PY{p}{,}N\PY{p}{)}\PY{p}{,}nrow\PY{o}{=}N\PY{p}{,}byrow\PY{o}{=}\PY{k+kc}{TRUE}\PY{p}{)}

  \PY{c+c1}{\PYZsh{}\PYZsh{}\PYZsh{}\PYZsh{}\PYZsh{} create $\mu_k=\left[\mu_{k1},\hdots,\mu_{kp}\right]$ vectors}
  \PY{c+c1}{\PYZsh{}\PYZsh{}\PYZsh{}\PYZsh{}\PYZsh{} place on top of each other to get KxP matrix}
  MuMat\PY{o}{\PYZlt{}\PYZhy{}}\PY{k+kt}{matrix}\PY{p}{(}\PY{l+m}{0}\PY{p}{,}nrow\PY{o}{=}K\PY{p}{,}ncol\PY{o}{=}P\PY{p}{)}
  \PY{k+kr}{for}\PY{p}{(}k \PY{k+kr}{in} \PY{l+m}{1}\PY{o}{:}K\PY{p}{)} MuMat\PY{p}{[}k\PY{p}{,}\PY{p}{]}\PY{o}{\PYZlt{}\PYZhy{}}\PY{k+kp}{colMeans}\PY{p}{(}X\PY{p}{[}Xclassint\PY{o}{==}k\PY{p}{,}\PY{p}{]}\PY{p}{)}
  MuIter\PY{o}{\PYZlt{}\PYZhy{}}MuMat

  \PY{c+c1}{\PYZsh{}\PYZsh{}\PYZsh{}\PYZsh{}\PYZsh{} create $\Sigma$}
  Sigma\PY{o}{\PYZlt{}\PYZhy{}}\PY{k+kt}{matrix}\PY{p}{(}\PY{l+m}{0}\PY{p}{,}nrow\PY{o}{=}P\PY{p}{,}ncol\PY{o}{=}P\PY{p}{)}
  \PY{k+kr}{for}\PY{p}{(}k \PY{k+kr}{in} \PY{l+m}{1}\PY{o}{:}K\PY{p}{)} 
  \PY{p}{\PYZob{}}
    rowuse\PY{o}{\PYZlt{}\PYZhy{}}\PY{k+kp}{which}\PY{p}{(}Xclassint\PY{o}{==}k\PY{p}{)}
    Sigma\PY{o}{\PYZlt{}\PYZhy{}}Sigma\PY{o}{+}\PY{k+kp}{length}\PY{p}{(}rowuse\PY{p}{)}\PY{o}{*}cov.wt\PY{p}{(}X\PY{p}{[}rowuse\PY{p}{,}\PY{p}{]}\PY{p}{,}cor\PY{o}{=}\PY{k+kc}{FALSE}
                                      \PY{p}{,}center\PY{o}{=}\PY{k+kc}{TRUE}\PY{p}{,}method\PY{o}{=}\PY{l+s}{\PYZdq{}}\PY{l+s}{ML\PYZdq{}}\PY{p}{)}\PY{o}{\PYZdl{}}cov
  \PY{p}{\PYZcb{}}
  Sigma\PY{o}{\PYZlt{}\PYZhy{}}Sigma\PY{o}{/}N
  \PY{c+c1}{\PYZsh{}\PYZsh{}\PYZsh{}\PYZsh{}\PYZsh{} extract Diag elements}
  sigmasqs\PY{o}{\PYZlt{}\PYZhy{}}\PY{k+kp}{diag}\PY{p}{(}Sigma\PY{p}{)}
  
  \PY{k+kr}{if}\PY{p}{(}\PY{o}{!}\PY{p}{(}\PY{k+kp}{is.null}\PY{p}{(}alph\PY{p}{)} \PY{o}{|} \PY{k+kp}{is.null}\PY{p}{(}wts\PY{p}{)}\PY{p}{)}\PY{p}{)} sigmasqs\PY{o}{\PYZlt{}\PYZhy{}}sigmasqs\PY{o}{+}alph\PY{o}{*}wts
  \PY{k+kr}{else} \PY{k+kr}{if}\PY{p}{(}\PY{o}{!}\PY{k+kp}{is.null}\PY{p}{(}alph\PY{p}{)}\PY{p}{)} sigmasqs\PY{o}{\PYZlt{}\PYZhy{}}sigmasqs\PY{o}{+}\PY{k+kp}{rep}\PY{p}{(}alph\PY{p}{,}P\PY{p}{)}
  
  \PY{c+c1}{\PYZsh{}\PYZsh{}\PYZsh{}\PYZsh{} Now start iterative estimation of the $\ell_1$ penalised means}
  \PY{c+c1}{\PYZsh{}\PYZsh{}\PYZsh{}\PYZsh{} Note \PYZdq{}squig\PYZdq{} is used for the ML estimates}
  G\PY{o}{\PYZlt{}\PYZhy{}}\PY{k+kt}{matrix}\PY{p}{(}\PY{l+m}{0}\PY{p}{,}nrow\PY{o}{=}K\PY{p}{,}ncol\PY{o}{=}K\PY{p}{)}
  deltatol\PY{o}{\PYZlt{}\PYZhy{}}\PY{l+m}{1e\PYZhy{}10}
  deltaMu\PY{o}{\PYZlt{}\PYZhy{}}\PY{l+m}{1}
  maxIter\PY{o}{\PYZlt{}\PYZhy{}}\PY{l+m}{500}
  itcount\PY{o}{\PYZlt{}\PYZhy{}}\PY{l+m}{0}
  \PY{k+kr}{while}\PY{p}{(}deltaMu\PY{o}{\PYZgt{}}\PY{p}{(}\PY{l+m}{1e\PYZhy{}5}\PY{p}{)} \PY{o}{\PYZam{}\PYZam{}} itcount\PY{o}{\PYZlt{}}maxIter\PY{p}{)}
  \PY{p}{\PYZob{}}
    deltaMuNumer\PY{o}{\PYZlt{}\PYZhy{}}\PY{l+m}{0}
    deltaMuDenom\PY{o}{\PYZlt{}\PYZhy{}}\PY{l+m}{0}
    itcount\PY{o}{\PYZlt{}\PYZhy{}}itcount\PY{l+m}{+1}
    
    \PY{c+c1}{\PYZsh{}\PYZsh{}\PYZsh{} For each of the features}
    \PY{k+kr}{for}\PY{p}{(}j \PY{k+kr}{in} \PY{l+m}{1}\PY{o}{:}P\PY{p}{)} 
    \PY{p}{\PYZob{}}
      beta.t.j\PY{o}{\PYZlt{}\PYZhy{}}MuIter\PY{p}{[}\PY{p}{,}j\PY{p}{]}
      musqig.j\PY{o}{\PYZlt{}\PYZhy{}}MuMat\PY{p}{[}\PY{p}{,}j\PY{p}{]}
      sqigY\PY{o}{\PYZlt{}\PYZhy{}}X\PY{p}{[}\PY{p}{,}j\PY{p}{]}
      sqigX\PY{o}{\PYZlt{}\PYZhy{}}\PY{k+kt}{matrix}\PY{p}{(}\PY{l+m}{0}\PY{p}{,}nrow\PY{o}{=}N\PY{p}{,}ncol\PY{o}{=}K\PY{p}{)}
      G\PY{o}{\PYZlt{}\PYZhy{}}\PY{k+kt}{matrix}\PY{p}{(}\PY{l+m}{0}\PY{p}{,}nrow\PY{o}{=}K\PY{p}{,}ncol\PY{o}{=}K\PY{p}{)}
      
      \PY{c+c1}{\PYZsh{}\PYZsh{}\PYZsh{} $\sum_{k=1}^{K-1}\sum_{k_{dash}=k+1}^K$}
      \PY{k+kr}{for}\PY{p}{(}k \PY{k+kr}{in} \PY{l+m}{1}\PY{o}{:}\PY{p}{(}K\PY{l+m}{\PYZhy{}1}\PY{p}{)}\PY{p}{)}
      \PY{p}{\PYZob{}}
        \PY{k+kr}{for}\PY{p}{(}kdash \PY{k+kr}{in} \PY{p}{(}k\PY{l+m}{+1}\PY{p}{)}\PY{o}{:}K\PY{p}{)}
        \PY{p}{\PYZob{}}
          PFweight\PY{o}{\PYZlt{}\PYZhy{}}\PY{l+m}{1}\PY{o}{/}\PY{k+kp}{abs}\PY{p}{(}musqig.j\PY{p}{[}k\PY{p}{]}\PY{o}{\PYZhy{}}musqig.j\PY{p}{[}kdash\PY{p}{]}\PY{p}{)}
          \PY{c+c1}{\PYZsh{}\PYZsh{}\PYZsh{} assign updated iterations, or tol value if \PYZdq{}zero\PYZdq{}}
          muDiffIter\PY{o}{\PYZlt{}\PYZhy{}}\PY{k+kp}{max}\PY{p}{(}\PY{k+kp}{abs}\PY{p}{(}beta.t.j\PY{p}{[}k\PY{p}{]}\PY{o}{\PYZhy{}}beta.t.j\PY{p}{[}kdash\PY{p}{]}\PY{p}{)}\PY{p}{,}deltatol\PY{p}{)}
          G\PY{p}{[}k\PY{p}{,}kdash\PY{p}{]}\PY{o}{\PYZlt{}\PYZhy{}}G\PY{p}{[}kdash\PY{p}{,}k\PY{p}{]}\PY{o}{\PYZlt{}\PYZhy{}} \PY{o}{\PYZhy{}}PFweight\PY{o}{/}muDiffIter
        \PY{p}{\PYZcb{}}
      \PY{p}{\PYZcb{}}
      \PY{k+kr}{for}\PY{p}{(}k \PY{k+kr}{in} \PY{l+m}{1}\PY{o}{:}K\PY{p}{)} 
      \PY{p}{\PYZob{}}
        sqigX\PY{p}{[}\PY{k+kp}{which}\PY{p}{(}Xclassint\PY{o}{==}k\PY{p}{)}\PY{p}{,}k\PY{p}{]}\PY{o}{\PYZlt{}\PYZhy{}}\PY{l+m}{1}  
        \PY{c+c1}{\PYZsh{}\PYZsh{}\PYZsh{} note the diag elements of G can be calculated as the sum of the column}
        G\PY{p}{[}k\PY{p}{,}k\PY{p}{]}\PY{o}{\PYZlt{}\PYZhy{}} \PY{o}{\PYZhy{}}\PY{k+kp}{sum}\PY{p}{(}G\PY{p}{[}\PY{p}{,}k\PY{p}{]}\PY{p}{)}
      \PY{p}{\PYZcb{}}
      \PY{c+c1}{\PYZsh{}\PYZsh{}\PYZsh{}\PYZsh{} $\hat{M}=\left(B^TB+\lambda\sigma_j^2G\right)^{-1}B^TJ$}
      MuIter\PY{p}{[}\PY{p}{,}j\PY{p}{]}\PY{o}{\PYZlt{}\PYZhy{}}\PY{k+kp}{solve}\PY{p}{(}\PY{k+kp}{t}\PY{p}{(}sqigX\PY{p}{)}\PY{o}{\PYZpc{}*\PYZpc{}}sqigX\PY{o}{+}lambdar\PY{o}{*}sigmasqs\PY{p}{[}j\PY{p}{]}\PY{o}{*}G\PY{p}{)}\PY{o}{\PYZpc{}*\PYZpc{}}\PY{p}{(}\PY{k+kp}{t}\PY{p}{(}sqigX\PY{p}{)}\PY{o}{\PYZpc{}*\PYZpc{}}sqigY\PY{p}{)}
      deltaMuNumer\PY{o}{\PYZlt{}\PYZhy{}}deltaMuNumer\PY{o}{+}\PY{k+kp}{sum}\PY{p}{(}\PY{k+kp}{abs}\PY{p}{(}MuIter\PY{p}{[}\PY{p}{,}j\PY{p}{]}\PY{o}{\PYZhy{}}beta.t.j\PY{p}{)}\PY{p}{)}
      deltaMuDenom\PY{o}{\PYZlt{}\PYZhy{}}deltaMuDenom\PY{o}{+}\PY{k+kp}{sum}\PY{p}{(}\PY{k+kp}{abs}\PY{p}{(}beta.t.j\PY{p}{)}\PY{p}{)}
    \PY{p}{\PYZcb{}}
    \PY{c+c1}{\PYZsh{}\PYZsh{}\PYZsh{} our break loop value}
    deltaMu\PY{o}{\PYZlt{}\PYZhy{}}deltaMuNumer\PY{o}{/}deltaMuDenom
  
  \PY{p}{\PYZcb{}}
  \PY{k+kp}{cat}\PY{p}{(}\PY{l+s}{\PYZdq{}}\PY{l+s}{Iterations performed to aquire a solution:\PYZdq{}}\PY{p}{,}itcount
       \PY{p}{,}\PY{l+s}{\PYZdq{}}\PY{l+s}{| Final tol val:\PYZdq{}}\PY{p}{,}deltaMu\PY{p}{,}\PY{l+s}{\PYZdq{}}\PY{l+s}{ \PYZbs{}n\PYZdq{}}\PY{p}{)}
  
  MuIter\PY{p}{[}\PY{k+kp}{which}\PY{p}{(}MuIter\PY{o}{\PYZlt{}}deltatol\PY{p}{)}\PY{p}{]}\PY{o}{\PYZlt{}\PYZhy{}}\PY{l+m}{0}
  
  \PY{c+c1}{\PYZsh{}\PYZsh{}\PYZsh{} $\tfrac{1}{2}\sum_{j=1}^{p}\hat{\mu}_{kj}^2$ constant term}
  consts\PY{o}{\PYZlt{}\PYZhy{}}\PY{k+kp}{rep}\PY{p}{(}\PY{l+m}{0}\PY{p}{,}K\PY{p}{)}
  \PY{k+kr}{for}\PY{p}{(}k \PY{k+kr}{in} \PY{l+m}{1}\PY{o}{:}K\PY{p}{)} consts\PY{p}{[}k\PY{p}{]}\PY{o}{\PYZlt{}\PYZhy{}}\PY{l+m}{0.5}\PY{o}{*}\PY{k+kp}{sum}\PY{p}{(}MuIter\PY{p}{[}k\PY{p}{,}\PY{p}{]}\PY{o}{\PYZca{}}\PY{l+m}{2}\PY{o}{/}sigmasqs\PY{p}{)}
  \PY{c+c1}{\PYZsh{}\PYZsh{}\PYZsh{} $\hat{\mu}_{kj}^2/\sigma_p^2$ term}
  lins\PY{o}{\PYZlt{}\PYZhy{}}\PY{k+kt}{vector}\PY{p}{(}mode\PY{o}{=}\PY{l+s}{\PYZdq{}}\PY{l+s}{list\PYZdq{}}\PY{p}{,}length\PY{o}{=}K\PY{p}{)}
  \PY{k+kr}{for}\PY{p}{(}k \PY{k+kr}{in} \PY{l+m}{1}\PY{o}{:}K\PY{p}{)} lins\PY{p}{[[}k\PY{p}{]]}\PY{o}{\PYZlt{}\PYZhy{}}MuIter\PY{p}{[}k\PY{p}{,}\PY{p}{]}\PY{o}{/}sigmasqs
  
  \PY{c+c1}{\PYZsh{}\PYZsh{}\PYZsh{} return calculated information as list object}
  \PY{k+kr}{return}\PY{p}{(}\PY{k+kt}{list}\PY{p}{(}classes\PY{o}{=}classnames\PY{p}{,}consts\PY{o}{=}consts\PY{p}{,}lins\PY{o}{=}lins\PY{p}{,}prior\PY{o}{=}priors
              \PY{p}{,}meanadj\PY{o}{=}transMeans\PY{p}{,}MuIter\PY{o}{=}MuIter\PY{p}{,}InitMu\PY{o}{=}MuMat\PY{p}{)}\PY{p}{)}
\PY{p}{\PYZcb{}}
\end{Verbatim}


\clearpage

\subsection{Predict class for new data and a PFDA object}

	\begin{Verbatim}[commandchars=\\\{\},codes={\catcode`\$=3\catcode`\^=7\catcode`\_=8},gobble=0,numbers=left,fontfamily=fvm,fontshape=n,fontsize=\footnotesize,tabsize=2]
\PY{c+c1}{\PYZsh{}\PYZsh{}\PYZsh{}\PYZsh{}\PYZsh{}\PYZsh{}\PYZsh{} FUNCTION: pfda\PYZus{}predict()}
\PY{c+c1}{\PYZsh{}\PYZsh{}\PYZsh{} estimate probabilities and class of new inputs}

\PY{c+c1}{\PYZsh{}\PYZsh{}\PYZsh{}\PYZsh{}\PYZsh{}\PYZsh{}\PYZsh{} input:}
\PY{c+c1}{\PYZsh{}\PYZsh{}\PYZsh{} ldaobj: an object created by create\PYZus{}pfda\PYZus{}obj()}
\PY{c+c1}{\PYZsh{}\PYZsh{}\PYZsh{} Xnew: a n x p matrix, of n obs and p variables. }
\PY{c+c1}{\PYZsh{}\PYZsh{}\PYZsh{}         May also be a single numeric vector (n=1) of length p}
\PY{c+c1}{\PYZsh{}\PYZsh{}\PYZsh{} priors: a vector of length K (\PYZsh{}classes) with elements in (0,1)}

pfda\PYZus{}predict\PY{o}{\PYZlt{}\PYZhy{}}\PY{k+kr}{function}\PY{p}{(}ldaobj\PY{p}{,}Xnew\PY{p}{,}priors\PY{o}{=}\PY{k+kc}{NULL}\PY{p}{)}
\PY{p}{\PYZob{}}
  \PY{k+kr}{if}\PY{p}{(}\PY{k+kp}{is.vector}\PY{p}{(}Xnew\PY{p}{,}mode\PY{o}{=}\PY{l+s}{\PYZdq{}}\PY{l+s}{numeric\PYZdq{}}\PY{p}{)}\PY{p}{)}\PY{p}{\PYZob{}}
    Xnew\PY{o}{\PYZlt{}\PYZhy{}}\PY{k+kt}{matrix}\PY{p}{(}Xnew\PY{p}{,}nrow\PY{o}{=}\PY{l+m}{1}\PY{p}{)}
  \PY{p}{\PYZcb{}}\PY{k+kr}{else} \PY{k+kr}{if}\PY{p}{(}\PY{o}{!}\PY{k+kp}{is.matrix}\PY{p}{(}Xnew\PY{p}{)}\PY{p}{)}\PY{p}{\PYZob{}}
    \PY{k+kp}{cat}\PY{p}{(}\PY{l+s}{\PYZdq{}}\PY{l+s}{Xnew must be numeric and either a vector or matrix \PYZbs{}n \PYZdq{}}\PY{p}{)}
  \PY{p}{\PYZcb{}}

  Nnew\PY{o}{\PYZlt{}\PYZhy{}}\PY{k+kp}{nrow}\PY{p}{(}Xnew\PY{p}{)}
  \PY{c+c1}{\PYZsh{}\PYZsh{}\PYZsh{} centre each feature at 0, as done on the training data}
  Xnew\PY{o}{\PYZlt{}\PYZhy{}}Xnew\PY{o}{\PYZhy{}}\PY{k+kt}{matrix}\PY{p}{(}\PY{k+kp}{rep}\PY{p}{(}ldaobj\PY{o}{\PYZdl{}}meanadj\PY{p}{,}Nnew\PY{p}{)}\PY{p}{,}nrow\PY{o}{=}Nnew\PY{p}{,}byrow\PY{o}{=}\PY{k+kc}{TRUE}\PY{p}{)}
  Xclasses\PY{o}{\PYZlt{}\PYZhy{}}ldaobj\PY{o}{\PYZdl{}}classes
  K\PY{o}{\PYZlt{}\PYZhy{}}\PY{k+kp}{length}\PY{p}{(}Xclasses\PY{p}{)}
  \PY{k+kr}{if}\PY{p}{(}\PY{k+kp}{is.null}\PY{p}{(}priors\PY{p}{)}\PY{p}{)} priors\PY{o}{\PYZlt{}\PYZhy{}}ldaobj\PY{o}{\PYZdl{}}prior
  
  outposteriors\PY{o}{\PYZlt{}\PYZhy{}}\PY{k+kt}{matrix}\PY{p}{(}\PY{l+m}{0}\PY{p}{,}nrow\PY{o}{=}Nnew\PY{p}{,}ncol\PY{o}{=}K\PY{p}{)}
  \PY{k+kp}{colnames}\PY{p}{(}outposteriors\PY{p}{)}\PY{o}{\PYZlt{}\PYZhy{}}Xclasses
  
  \PY{c+c1}{\PYZsh{}\PYZsh{}\PYZsh{} discrim function: $\delta_k(x_{new})$}
  \PY{k+kr}{for}\PY{p}{(}i \PY{k+kr}{in} \PY{l+m}{1}\PY{o}{:}Nnew\PY{p}{)}\PY{p}{\PYZob{}}
    \PY{k+kr}{for}\PY{p}{(}k \PY{k+kr}{in} \PY{l+m}{1}\PY{o}{:}K\PY{p}{)}\PY{p}{\PYZob{}}
      outposteriors\PY{p}{[}i\PY{p}{,}k\PY{p}{]}\PY{o}{\PYZlt{}\PYZhy{}}\PY{k+kp}{log}\PY{p}{(}priors\PY{p}{[}k\PY{p}{]}\PY{p}{)} \PY{o}{\PYZhy{}} ldaobj\PY{o}{\PYZdl{}}consts\PY{p}{[}k\PY{p}{]} 
                                 \PY{o}{+} \PY{k+kp}{sum}\PY{p}{(}Xnew\PY{p}{[}i\PY{p}{,}\PY{p}{]}\PY{o}{*}ldaobj\PY{o}{\PYZdl{}}lins\PY{p}{[[}k\PY{p}{]]}\PY{p}{)}
    \PY{p}{\PYZcb{}}
  \PY{p}{\PYZcb{}}
  
  \PY{c+c1}{\PYZsh{}\PYZsh{}\PYZsh{} predicted class is arg max}
  \PY{c+c1}{\PYZsh{}\PYZsh{}\PYZsh{} $p(G_K|x_{new})=\tfrac{P\left(x_{new}|G_k\right)P\left(G_k\right)}{\sum_{i=1}^{K}P\left(x_{new}|G_i\right)P\left(G_i\right)}$}
  predclasses\PY{o}{\PYZlt{}\PYZhy{}}\PY{k+kp}{rep}\PY{p}{(}\PY{l+m}{0}\PY{p}{,}Nnew\PY{p}{)}
  \PY{k+kr}{for}\PY{p}{(}i \PY{k+kr}{in} \PY{l+m}{1}\PY{o}{:}Nnew\PY{p}{)}\PY{p}{\PYZob{}}
    outposteriors\PY{p}{[}i\PY{p}{,}\PY{p}{]}\PY{o}{\PYZlt{}\PYZhy{}}\PY{k+kp}{exp}\PY{p}{(}outposteriors\PY{p}{[}i\PY{p}{,}\PY{p}{]}\PY{p}{)}
    predclasses\PY{p}{[}i\PY{p}{]}\PY{o}{\PYZlt{}\PYZhy{}}\PY{k+kp}{which.max}\PY{p}{(}outposteriors\PY{p}{[}i\PY{p}{,}\PY{p}{]}\PY{p}{)}
    outposteriors\PY{p}{[}i\PY{p}{,}\PY{p}{]}\PY{o}{\PYZlt{}\PYZhy{}}outposteriors\PY{p}{[}i\PY{p}{,}\PY{p}{]}\PY{o}{/}\PY{k+kp}{sum}\PY{p}{(}outposteriors\PY{p}{[}i\PY{p}{,}\PY{p}{]}\PY{p}{)}
  \PY{p}{\PYZcb{}}
  outpred\PY{o}{\PYZlt{}\PYZhy{}}\PY{k+kp}{factor}\PY{p}{(}Xclasses\PY{p}{[}predclasses\PY{p}{]}\PY{p}{)}
  
  \PY{k+kr}{return}\PY{p}{(}\PY{k+kt}{list}\PY{p}{(}pred\PY{o}{=}outpred\PY{p}{,}posteriors\PY{o}{=}outposteriors\PY{p}{)}\PY{p}{)}
\PY{p}{\PYZcb{}}
\end{Verbatim}


\clearpage



\section{Pareto Fronts for variable ranking} 


\codetab{
\codeentry{Calculate dominating features}{11\_dom\_feat.c}{domfeat()}
\codeentry{Pareto Front wrapper functions}{12\_pareto\_fronts.R}{paretoRanking()}
}

\clearpage

\subsection{Calculate dominating features} 

Below is the core of the Pareto Front code, finding features that are the dominated as per the definition. Written in {\tt C} to be compiled to a {\tt .so} file (or \texttt{.dll} on Windows operating systems) that in turn can be loaded into {\tt R}.

	\begin{Verbatim}[commandchars=\\\{\},codes={\catcode`\$=3\catcode`\^=7\catcode`\_=8},gobble=0,numbers=left,fontfamily=fvm,fontshape=n,fontsize=\footnotesize,tabsize=2]
\PY{c+cp}{\PYZsh{}}\PY{c+cp}{include \PYZlt{}R.h\PYZgt{} }

\PY{k+kt}{void} \PY{n+nf}{domfeat}\PY{p}{(}\PY{k+kt}{int} \PY{o}{*}\PY{n}{n}\PY{p}{,} \PY{k+kt}{double} \PY{o}{*}\PY{n}{obja}\PY{p}{,} \PY{k+kt}{double} \PY{o}{*}\PY{n}{objb}\PY{p}{,} \PY{k+kt}{int} \PY{o}{*}\PY{n}{domvec}\PY{p}{)}
\PY{p}{\PYZob{}}
	\PY{k+kt}{int} \PY{n}{i}\PY{p}{,} \PY{n}{j}\PY{p}{,} \PY{n}{nonDomI}\PY{p}{;}
	\PY{k}{for} \PY{p}{(}\PY{n}{i}\PY{o}{=}\PY{l+m+mi}{0}\PY{p}{;} \PY{n}{i}\PY{o}{\PYZlt{}}\PY{o}{*}\PY{n}{n}\PY{p}{;} \PY{n}{i}\PY{o}{+}\PY{o}{+}\PY{p}{)}
	\PY{p}{\PYZob{}}
		\PY{n}{j}\PY{o}{=}\PY{l+m+mi}{0}\PY{p}{;}
		\PY{n}{nonDomI}\PY{o}{=}\PY{l+m+mi}{1}\PY{p}{;}
		\PY{k}{while}\PY{p}{(}\PY{n}{nonDomI} \PY{o}{\PYZam{}}\PY{o}{\PYZam{}} \PY{n}{j}\PY{o}{\PYZlt{}}\PY{o}{*}\PY{n}{n}\PY{p}{)}
		\PY{p}{\PYZob{}}
			\PY{k}{if}\PY{p}{(}\PY{p}{(}\PY{n}{obja}\PY{p}{[}\PY{n}{i}\PY{p}{]}\PY{o}{\PYZlt{}}\PY{n}{obja}\PY{p}{[}\PY{n}{j}\PY{p}{]}\PY{p}{)} \PY{o}{\PYZam{}}\PY{o}{\PYZam{}} \PY{p}{(}\PY{n}{objb}\PY{p}{[}\PY{n}{i}\PY{p}{]}\PY{o}{\PYZlt{}}\PY{o}{=}\PY{n}{objb}\PY{p}{[}\PY{n}{j}\PY{p}{]}\PY{p}{)}\PY{p}{)} \PY{n}{nonDomI}\PY{o}{=}\PY{l+m+mi}{0}\PY{p}{;}
			\PY{k}{else} \PY{k}{if}\PY{p}{(}\PY{p}{(}\PY{n}{objb}\PY{p}{[}\PY{n}{i}\PY{p}{]}\PY{o}{\PYZlt{}}\PY{n}{objb}\PY{p}{[}\PY{n}{j}\PY{p}{]}\PY{p}{)} \PY{o}{\PYZam{}}\PY{o}{\PYZam{}} \PY{p}{(}\PY{n}{obja}\PY{p}{[}\PY{n}{i}\PY{p}{]}\PY{o}{\PYZlt{}}\PY{o}{=}\PY{n}{obja}\PY{p}{[}\PY{n}{j}\PY{p}{]}\PY{p}{)}\PY{p}{)} \PY{n}{nonDomI}\PY{o}{=}\PY{l+m+mi}{0}\PY{p}{;}
			\PY{n}{j}\PY{o}{+}\PY{o}{=}\PY{l+m+mi}{1}\PY{p}{;}
		\PY{p}{\PYZcb{}}
		\PY{n}{domvec}\PY{p}{[}\PY{n}{i}\PY{p}{]}\PY{o}{=}\PY{o}{\PYZhy{}}\PY{n}{nonDomI}\PY{o}{+}\PY{l+m+mi}{1}\PY{p}{;}
	\PY{p}{\PYZcb{}}
\PY{p}{\PYZcb{}}
\end{Verbatim}


\clearpage

\subsection{Pareto Front wrapper functions} 

	\begin{Verbatim}[commandchars=\\\{\},codes={\catcode`\$=3\catcode`\^=7\catcode`\_=8},gobble=0,numbers=left,fontfamily=fvm,fontshape=n,fontsize=\footnotesize,tabsize=2]
\PY{k+kn}{library}\PY{p}{(}animation\PY{p}{)}
\PY{k+kp}{dyn.load}\PY{p}{(}\PY{l+s}{\PYZdq{}}\PY{l+s}{domfeat.so\PYZdq{}}\PY{p}{)}

\PY{c+c1}{\PYZsh{}\PYZsh{}\PYZsh{}\PYZsh{}\PYZsh{}\PYZsh{}\PYZsh{}\PYZsh{}\PYZsh{}\PYZsh{}\PYZsh{}\PYZsh{}\PYZsh{}\PYZsh{}\PYZsh{}\PYZsh{}\PYZsh{}\PYZsh{}\PYZsh{}\PYZsh{}\PYZsh{}\PYZsh{}\PYZsh{}\PYZsh{}\PYZsh{}\PYZsh{}\PYZsh{}\PYZsh{}\PYZsh{}\PYZsh{}\PYZsh{}\PYZsh{}\PYZsh{}\PYZsh{}\PYZsh{}\PYZsh{}\PYZsh{}\PYZsh{}\PYZsh{}\PYZsh{}\PYZsh{}\PYZsh{}}
\PY{c+c1}{\PYZsh{}\PYZsh{}\PYZsh{}\PYZsh{}\PYZsh{}\PYZsh{}\PYZsh{}\PYZsh{}\PYZsh{}\PYZsh{}\PYZsh{}\PYZsh{}\PYZsh{}\PYZsh{}\PYZsh{}\PYZsh{}\PYZsh{}\PYZsh{}\PYZsh{}\PYZsh{}\PYZsh{}\PYZsh{}\PYZsh{}\PYZsh{}\PYZsh{}\PYZsh{}\PYZsh{}\PYZsh{}\PYZsh{}\PYZsh{}\PYZsh{}\PYZsh{}\PYZsh{}\PYZsh{}\PYZsh{}\PYZsh{}\PYZsh{}\PYZsh{}\PYZsh{}\PYZsh{}\PYZsh{}\PYZsh{}}
\PY{c+c1}{\PYZsh{}}
\PY{c+c1}{\PYZsh{} Pairwise case of Pareto Fronts}
\PY{c+c1}{\PYZsh{}     obj is the $2 \times n$ matrix of the two vectors of length $n$ for  }
\PY{c+c1}{\PYZsh{}			the features/observations of the 2 criteria/objective functions}
\PY{c+c1}{\PYZsh{}     istomin is a boolean vector of whether the criteria obj$_1$, obj$_2$  }
\PY{c+c1}{\PYZsh{}			are to be minimised (=TRUE), respectively}
\PY{c+c1}{\PYZsh{}}
\PY{c+c1}{\PYZsh{}\PYZsh{}\PYZsh{}\PYZsh{}\PYZsh{}\PYZsh{}\PYZsh{}\PYZsh{}\PYZsh{}\PYZsh{}\PYZsh{}\PYZsh{}\PYZsh{}\PYZsh{}\PYZsh{}\PYZsh{}\PYZsh{}\PYZsh{}\PYZsh{}\PYZsh{}\PYZsh{}\PYZsh{}\PYZsh{}\PYZsh{}\PYZsh{}\PYZsh{}\PYZsh{}\PYZsh{}\PYZsh{}\PYZsh{}\PYZsh{}\PYZsh{}\PYZsh{}\PYZsh{}\PYZsh{}\PYZsh{}\PYZsh{}\PYZsh{}\PYZsh{}\PYZsh{}\PYZsh{}\PYZsh{}}
\PY{c+c1}{\PYZsh{}\PYZsh{}\PYZsh{}\PYZsh{}\PYZsh{}\PYZsh{}\PYZsh{}\PYZsh{}\PYZsh{}\PYZsh{}\PYZsh{}\PYZsh{}\PYZsh{}\PYZsh{}\PYZsh{}\PYZsh{}\PYZsh{}\PYZsh{}\PYZsh{}\PYZsh{}\PYZsh{}\PYZsh{}\PYZsh{}\PYZsh{}\PYZsh{}\PYZsh{}\PYZsh{}\PYZsh{}\PYZsh{}\PYZsh{}\PYZsh{}\PYZsh{}\PYZsh{}\PYZsh{}\PYZsh{}\PYZsh{}\PYZsh{}\PYZsh{}\PYZsh{}\PYZsh{}\PYZsh{}\PYZsh{}}

\PY{c+c1}{\PYZsh{}}
\PY{c+c1}{\PYZsh{} This function returns a vector of \PYZsq{}dominated\PYZsq{} observations (Boolean, }
\PY{c+c1}{\PYZsh{} length $n$ vector) FALSE=Pareto front, TRUE=dominated observation}
\PY{c+c1}{\PYZsh{}}

domFeaturesPW\PY{o}{\PYZlt{}\PYZhy{}}\PY{k+kr}{function}\PY{p}{(}obj\PY{p}{,}istomin\PY{p}{)}
\PY{p}{\PYZob{}}
	\PY{c+c1}{\PYZsh{}if to be minimised then just make negative and maximise}
	obj\PY{p}{[}\PY{p}{,}istomin\PY{p}{]}\PY{o}{\PYZlt{}\PYZhy{}} \PY{o}{\PYZhy{}}obj\PY{p}{[}\PY{p}{,}istomin\PY{p}{]} 
	n\PY{o}{\PYZlt{}\PYZhy{}}\PY{k+kp}{as.integer}\PY{p}{(}\PY{k+kp}{nrow}\PY{p}{(}obj\PY{p}{)}\PY{p}{)}
	obj1\PY{o}{\PYZlt{}\PYZhy{}}\PY{k+kp}{as.double}\PY{p}{(}obj\PY{p}{[}\PY{p}{,}\PY{l+m}{1}\PY{p}{]}\PY{p}{)}
	obj2\PY{o}{\PYZlt{}\PYZhy{}}\PY{k+kp}{as.double}\PY{p}{(}obj\PY{p}{[}\PY{p}{,}\PY{l+m}{2}\PY{p}{]}\PY{p}{)}
	domvec\PY{o}{\PYZlt{}\PYZhy{}}\PY{k+kp}{as.integer}\PY{p}{(}\PY{k+kp}{rep}\PY{p}{(}\PY{l+m}{0}\PY{p}{,}n\PY{p}{)}\PY{p}{)} \PY{c+c1}{\PYZsh{}output vector}
	
	\PY{k+kr}{return}\PY{p}{(}\PY{k+kp}{as.logical}\PY{p}{(}\PY{l+m}{.}C\PY{p}{(}\PY{l+s}{\PYZdq{}}\PY{l+s}{domfeat\PYZdq{}}\PY{p}{,}n\PY{p}{,}obj1\PY{p}{,}obj2\PY{p}{,}domvec\PY{p}{)}\PY{p}{[[}\PY{l+m}{4}\PY{p}{]]}\PY{p}{)}\PY{p}{)}
\PY{p}{\PYZcb{}}
	
\PY{c+c1}{\PYZsh{}\PYZsh{}\PYZsh{}\PYZsh{}\PYZsh{}\PYZsh{}\PYZsh{}\PYZsh{}\PYZsh{}\PYZsh{}\PYZsh{}\PYZsh{}\PYZsh{}\PYZsh{}\PYZsh{}\PYZsh{}\PYZsh{}\PYZsh{}\PYZsh{}\PYZsh{}\PYZsh{}\PYZsh{}\PYZsh{}\PYZsh{}\PYZsh{}\PYZsh{}\PYZsh{}\PYZsh{}\PYZsh{}\PYZsh{}\PYZsh{}\PYZsh{}\PYZsh{}\PYZsh{}\PYZsh{}\PYZsh{}\PYZsh{}\PYZsh{}\PYZsh{}\PYZsh{}\PYZsh{}\PYZsh{}}
\PY{c+c1}{\PYZsh{}\PYZsh{}\PYZsh{}\PYZsh{}\PYZsh{}\PYZsh{}\PYZsh{}\PYZsh{}\PYZsh{}\PYZsh{}\PYZsh{}\PYZsh{}\PYZsh{}\PYZsh{}\PYZsh{}\PYZsh{}\PYZsh{}\PYZsh{}\PYZsh{}\PYZsh{}\PYZsh{}\PYZsh{}\PYZsh{}\PYZsh{}\PYZsh{}\PYZsh{}\PYZsh{}\PYZsh{}\PYZsh{}\PYZsh{}\PYZsh{}\PYZsh{}\PYZsh{}\PYZsh{}\PYZsh{}\PYZsh{}\PYZsh{}\PYZsh{}\PYZsh{}\PYZsh{}\PYZsh{}\PYZsh{}}
\PY{c+c1}{\PYZsh{}}
\PY{c+c1}{\PYZsh{} General case}
\PY{c+c1}{\PYZsh{}     objmatrix is $n \times m$ matrix. $n$ features/observations and }
\PY{c+c1}{\PYZsh{}     $m$ criteria/objective functions istominvec is a boolean vec}
\PY{c+c1}{\PYZsh{}     of length $m$ to say whether the criteria are to be minimised}
\PY{c+c1}{\PYZsh{}}
\PY{c+c1}{\PYZsh{}\PYZsh{}\PYZsh{}\PYZsh{}\PYZsh{}\PYZsh{}\PYZsh{}\PYZsh{}\PYZsh{}\PYZsh{}\PYZsh{}\PYZsh{}\PYZsh{}\PYZsh{}\PYZsh{}\PYZsh{}\PYZsh{}\PYZsh{}\PYZsh{}\PYZsh{}\PYZsh{}\PYZsh{}\PYZsh{}\PYZsh{}\PYZsh{}\PYZsh{}\PYZsh{}\PYZsh{}\PYZsh{}\PYZsh{}\PYZsh{}\PYZsh{}\PYZsh{}\PYZsh{}\PYZsh{}\PYZsh{}\PYZsh{}\PYZsh{}\PYZsh{}\PYZsh{}\PYZsh{}\PYZsh{}}
\PY{c+c1}{\PYZsh{}\PYZsh{}\PYZsh{}\PYZsh{}\PYZsh{}\PYZsh{}\PYZsh{}\PYZsh{}\PYZsh{}\PYZsh{}\PYZsh{}\PYZsh{}\PYZsh{}\PYZsh{}\PYZsh{}\PYZsh{}\PYZsh{}\PYZsh{}\PYZsh{}\PYZsh{}\PYZsh{}\PYZsh{}\PYZsh{}\PYZsh{}\PYZsh{}\PYZsh{}\PYZsh{}\PYZsh{}\PYZsh{}\PYZsh{}\PYZsh{}\PYZsh{}\PYZsh{}\PYZsh{}\PYZsh{}\PYZsh{}\PYZsh{}\PYZsh{}\PYZsh{}\PYZsh{}\PYZsh{}\PYZsh{}}

\PY{c+c1}{\PYZsh{}}
\PY{c+c1}{\PYZsh{} This function returns a vector of \PYZsq{}dominated\PYZsq{} observations (Boolean,}
\PY{c+c1}{\PYZsh{} length $n$ vector) FALSE=Pareto front, TRUE=dominated observation}
\PY{c+c1}{\PYZsh{} same as pairwise but the input can take more than two objective functions}
\PY{c+c1}{\PYZsh{}}

domFeatures\PY{o}{\PYZlt{}\PYZhy{}}\PY{k+kr}{function}\PY{p}{(}objmatrix\PY{p}{,}istominvec\PY{p}{)}
\PY{p}{\PYZob{}}
	n\PY{o}{\PYZlt{}\PYZhy{}}\PY{k+kp}{as.integer}\PY{p}{(}\PY{k+kp}{nrow}\PY{p}{(}objmatrix\PY{p}{)}\PY{p}{)}
	m\PY{o}{\PYZlt{}\PYZhy{}}\PY{k+kp}{ncol}\PY{p}{(}objmatrix\PY{p}{)}
	objmatrix\PY{p}{[}\PY{p}{,}istominvec\PY{p}{]}\PY{o}{\PYZlt{}\PYZhy{}} \PY{o}{\PYZhy{}}objmatrix\PY{p}{[}\PY{p}{,}istominvec\PY{p}{]}
	vecdomvec\PY{o}{\PYZlt{}\PYZhy{}}\PY{k+kp}{rep}\PY{p}{(}\PY{l+m}{1}\PY{p}{,}n\PY{p}{)}
	indxs\PY{o}{\PYZlt{}\PYZhy{}}combn\PY{p}{(}m\PY{p}{,}\PY{l+m}{2}\PY{p}{)}
	nm\PY{o}{\PYZlt{}\PYZhy{}}\PY{k+kp}{ncol}\PY{p}{(}indxs\PY{p}{)}
	i\PY{o}{\PYZlt{}\PYZhy{}}\PY{l+m}{0}
	\PY{k+kr}{while}\PY{p}{(}i\PY{o}{\PYZlt{}}nm\PY{p}{)}
	\PY{p}{\PYZob{}}
		i\PY{o}{\PYZlt{}\PYZhy{}}i\PY{l+m}{+1}
		\PY{c+c1}{\PYZsh{} call pairwise function, take the intersection of previous }
		\PY{c+c1}{\PYZsh{} dominated observations remembering the intersection(s)}
		\PY{c+c1}{\PYZsh{} of dominated in the same as unions(s) of Pareto fronts}
		vecdomvec\PY{o}{\PYZlt{}\PYZhy{}}vecdomvec \PY{o}{*} \PY{l+m}{.}C\PY{p}{(}\PY{l+s}{\PYZdq{}}\PY{l+s}{domfeat\PYZdq{}}
							\PY{p}{,}n
							\PY{p}{,}\PY{k+kp}{as.double}\PY{p}{(}objmatrix\PY{p}{[}\PY{p}{,}indxs\PY{p}{[}\PY{l+m}{1}\PY{p}{,}i\PY{p}{]]}\PY{p}{)}
							\PY{p}{,}\PY{k+kp}{as.double}\PY{p}{(}objmatrix\PY{p}{[}\PY{p}{,}indxs\PY{p}{[}\PY{l+m}{2}\PY{p}{,}i\PY{p}{]]}\PY{p}{)}
							\PY{p}{,}\PY{k+kp}{as.integer}\PY{p}{(}\PY{k+kp}{rep}\PY{p}{(}\PY{l+m}{0}\PY{p}{,}n\PY{p}{)}\PY{p}{)}\PY{p}{)}\PY{p}{[[}\PY{l+m}{4}\PY{p}{]]}
	\PY{p}{\PYZcb{}}
	\PY{k+kr}{return}\PY{p}{(}\PY{k+kp}{as.logical}\PY{p}{(}vecdomvec\PY{p}{)}\PY{p}{)}
\PY{p}{\PYZcb{}}

\PY{c+c1}{\PYZsh{}\PYZsh{}\PYZsh{}\PYZsh{}\PYZsh{}\PYZsh{}\PYZsh{}\PYZsh{}\PYZsh{}\PYZsh{}\PYZsh{}\PYZsh{}\PYZsh{}\PYZsh{}\PYZsh{}\PYZsh{}\PYZsh{}\PYZsh{}\PYZsh{}\PYZsh{}\PYZsh{}\PYZsh{}\PYZsh{}\PYZsh{}\PYZsh{}\PYZsh{}\PYZsh{}\PYZsh{}\PYZsh{}\PYZsh{}\PYZsh{}\PYZsh{}\PYZsh{}\PYZsh{}\PYZsh{}\PYZsh{}\PYZsh{}\PYZsh{}\PYZsh{}\PYZsh{}\PYZsh{}\PYZsh{}}
\PY{c+c1}{\PYZsh{}\PYZsh{}\PYZsh{}\PYZsh{}\PYZsh{}\PYZsh{}\PYZsh{}\PYZsh{}\PYZsh{}\PYZsh{}\PYZsh{}\PYZsh{}\PYZsh{}\PYZsh{}\PYZsh{}\PYZsh{}\PYZsh{}\PYZsh{}\PYZsh{}\PYZsh{}\PYZsh{}\PYZsh{}\PYZsh{}\PYZsh{}\PYZsh{}\PYZsh{}\PYZsh{}\PYZsh{}\PYZsh{}\PYZsh{}\PYZsh{}\PYZsh{}\PYZsh{}\PYZsh{}\PYZsh{}\PYZsh{}\PYZsh{}\PYZsh{}\PYZsh{}\PYZsh{}\PYZsh{}\PYZsh{}}
\PY{c+c1}{\PYZsh{}}
\PY{c+c1}{\PYZsh{} Sucessive Pareto Fronts}
\PY{c+c1}{\PYZsh{}     noFronts is the \PYZsh{} of pareto fronts required}
\PY{c+c1}{\PYZsh{}}
\PY{c+c1}{\PYZsh{}     fn returns a vector of length $n$}
\PY{c+c1}{\PYZsh{}           each element is labelled the pareto front \PYZsh{},  }
\PY{c+c1}{\PYZsh{}           0 is dominated even after noFronts found}
\PY{c+c1}{\PYZsh{}}
\PY{c+c1}{\PYZsh{}\PYZsh{}\PYZsh{}\PYZsh{}\PYZsh{}\PYZsh{}\PYZsh{}\PYZsh{}\PYZsh{}\PYZsh{}\PYZsh{}\PYZsh{}\PYZsh{}\PYZsh{}\PYZsh{}\PYZsh{}\PYZsh{}\PYZsh{}\PYZsh{}\PYZsh{}\PYZsh{}\PYZsh{}\PYZsh{}\PYZsh{}\PYZsh{}\PYZsh{}\PYZsh{}\PYZsh{}\PYZsh{}\PYZsh{}\PYZsh{}\PYZsh{}\PYZsh{}\PYZsh{}\PYZsh{}\PYZsh{}\PYZsh{}\PYZsh{}\PYZsh{}\PYZsh{}\PYZsh{}\PYZsh{}}
\PY{c+c1}{\PYZsh{}\PYZsh{}\PYZsh{}\PYZsh{}\PYZsh{}\PYZsh{}\PYZsh{}\PYZsh{}\PYZsh{}\PYZsh{}\PYZsh{}\PYZsh{}\PYZsh{}\PYZsh{}\PYZsh{}\PYZsh{}\PYZsh{}\PYZsh{}\PYZsh{}\PYZsh{}\PYZsh{}\PYZsh{}\PYZsh{}\PYZsh{}\PYZsh{}\PYZsh{}\PYZsh{}\PYZsh{}\PYZsh{}\PYZsh{}\PYZsh{}\PYZsh{}\PYZsh{}\PYZsh{}\PYZsh{}\PYZsh{}\PYZsh{}\PYZsh{}\PYZsh{}\PYZsh{}\PYZsh{}\PYZsh{}}

paretoFronts\PY{o}{\PYZlt{}\PYZhy{}}\PY{k+kr}{function}\PY{p}{(}noFronts\PY{p}{,}objmatrix\PY{p}{,}istominvec\PY{p}{)}
\PY{p}{\PYZob{}}
	objmatrix\PY{p}{[}\PY{p}{,}istominvec\PY{p}{]}\PY{o}{\PYZlt{}\PYZhy{}} \PY{o}{\PYZhy{}}objmatrix\PY{p}{[}\PY{p}{,}istominvec\PY{p}{]}
	n\PY{o}{\PYZlt{}\PYZhy{}}\PY{k+kp}{as.integer}\PY{p}{(}\PY{k+kp}{nrow}\PY{p}{(}objmatrix\PY{p}{)}\PY{p}{)}
	m\PY{o}{\PYZlt{}\PYZhy{}}\PY{k+kp}{ncol}\PY{p}{(}objmatrix\PY{p}{)}
	pfvec\PY{o}{\PYZlt{}\PYZhy{}}\PY{k+kp}{rep}\PY{p}{(}\PY{l+m}{0}\PY{p}{,}n\PY{p}{)} \PY{c+c1}{\PYZsh{}output vector}
	\PY{c+c1}{\PYZsh{}once a front is found we need to set the correponding values to $\infty$ or }
	\PY{c+c1}{\PYZsh{} $-\infty$ so they won\PYZsq{}t be chosen again}
	\PY{c+c1}{\PYZsh{}try: as.numeric(c(TRUE,FALSE,TRUE))*2\PYZhy{}1 to see what the next line is doing}
	\PY{c+c1}{\PYZsh{}if Min then set 1, ifMax then set \PYZhy{}1 (the sign of the Inf if we find front }
	\PY{c+c1}{\PYZsh{}	 and have to put to a value)}
	ourInfs\PY{o}{\PYZlt{}\PYZhy{}}\PY{k+kp}{min}\PY{p}{(}objmatrix\PY{p}{)}\PY{l+m}{\PYZhy{}1}
	allFrontsFound\PY{o}{\PYZlt{}\PYZhy{}}\PY{k+kc}{FALSE}
	
	i\PY{o}{\PYZlt{}\PYZhy{}}\PY{l+m}{0}
	\PY{k+kr}{while}\PY{p}{(}i\PY{o}{\PYZlt{}}noFronts \PY{o}{\PYZam{}\PYZam{}} \PY{o}{!}allFrontsFound\PY{p}{)} \PY{c+c1}{\PYZsh{}go thru all fronts required}
	\PY{p}{\PYZob{}}
		i\PY{o}{\PYZlt{}\PYZhy{}}i\PY{l+m}{+1}
		df\PY{o}{\PYZlt{}\PYZhy{}}domFeatures\PY{p}{(}objmatrix\PY{p}{,}\PY{k+kp}{rep}\PY{p}{(}\PY{k+kc}{FALSE}\PY{p}{,}m\PY{p}{)}\PY{p}{)} \PY{c+c1}{\PYZsh{}general m obj vectors function}
		\PY{c+c1}{\PYZsh{} pf.i are the indexs of the output vector that need to be updated }
		\PY{c+c1}{\PYZsh{} with the pareto front number}
		pf.i\PY{o}{\PYZlt{}\PYZhy{}}\PY{p}{(}\PY{o}{!}df\PY{p}{)} \PY{o}{\PYZam{}} \PY{p}{(}pfvec\PY{o}{\PYZlt{}}\PY{l+m}{1}\PY{p}{)} 
		pfvec\PY{p}{[}pf.i\PY{p}{]}\PY{o}{\PYZlt{}\PYZhy{}}i
		\PY{c+c1}{\PYZsh{} re\PYZhy{}assign values were pareto front found}
		objmatrix\PY{p}{[}pf.i\PY{p}{,}\PY{p}{]}\PY{o}{\PYZlt{}\PYZhy{}}ourInfs
		\PY{k+kr}{if}\PY{p}{(}\PY{k+kp}{all}\PY{p}{(}pfvec\PY{o}{\PYZgt{}}\PY{l+m}{0}\PY{p}{)}\PY{p}{)} allFrontsFound\PY{o}{\PYZlt{}\PYZhy{}}\PY{k+kc}{TRUE}
	\PY{p}{\PYZcb{}}
	\PY{k+kr}{return}\PY{p}{(}pfvec\PY{p}{)}	
\PY{p}{\PYZcb{}}

\PY{c+c1}{\PYZsh{}\PYZsh{}\PYZsh{}\PYZsh{}\PYZsh{}\PYZsh{}\PYZsh{}\PYZsh{}\PYZsh{}\PYZsh{}\PYZsh{}\PYZsh{}\PYZsh{}\PYZsh{}\PYZsh{}\PYZsh{}\PYZsh{}\PYZsh{}\PYZsh{}\PYZsh{}\PYZsh{}\PYZsh{}\PYZsh{}\PYZsh{}\PYZsh{}\PYZsh{}\PYZsh{}\PYZsh{}\PYZsh{}\PYZsh{}\PYZsh{}\PYZsh{}\PYZsh{}\PYZsh{}\PYZsh{}\PYZsh{}\PYZsh{}\PYZsh{}\PYZsh{}\PYZsh{}\PYZsh{}\PYZsh{}}
\PY{c+c1}{\PYZsh{}\PYZsh{}\PYZsh{}\PYZsh{}\PYZsh{}\PYZsh{}\PYZsh{}\PYZsh{}\PYZsh{}\PYZsh{}\PYZsh{}\PYZsh{}\PYZsh{}\PYZsh{}\PYZsh{}\PYZsh{}\PYZsh{}\PYZsh{}\PYZsh{}\PYZsh{}\PYZsh{}\PYZsh{}\PYZsh{}\PYZsh{}\PYZsh{}\PYZsh{}\PYZsh{}\PYZsh{}\PYZsh{}\PYZsh{}\PYZsh{}\PYZsh{}\PYZsh{}\PYZsh{}\PYZsh{}\PYZsh{}\PYZsh{}\PYZsh{}\PYZsh{}\PYZsh{}\PYZsh{}\PYZsh{}}
\PY{c+c1}{\PYZsh{}}
\PY{c+c1}{\PYZsh{} Leave\PYZhy{}one\PYZhy{}out/k\PYZhy{}fold feature ranking}
\PY{c+c1}{\PYZsh{}}
\PY{c+c1}{\PYZsh{} returns a vector of length $n$ with values $\in (0,1]$ for feature importance		}
\PY{c+c1}{\PYZsh{}}
\PY{c+c1}{\PYZsh{}\PYZsh{}\PYZsh{}\PYZsh{}\PYZsh{}\PYZsh{}\PYZsh{}\PYZsh{}\PYZsh{}\PYZsh{}\PYZsh{}\PYZsh{}\PYZsh{}\PYZsh{}\PYZsh{}\PYZsh{}\PYZsh{}\PYZsh{}\PYZsh{}\PYZsh{}\PYZsh{}\PYZsh{}\PYZsh{}\PYZsh{}\PYZsh{}\PYZsh{}\PYZsh{}\PYZsh{}\PYZsh{}\PYZsh{}\PYZsh{}\PYZsh{}\PYZsh{}\PYZsh{}\PYZsh{}\PYZsh{}\PYZsh{}\PYZsh{}\PYZsh{}\PYZsh{}\PYZsh{}\PYZsh{}}
\PY{c+c1}{\PYZsh{}\PYZsh{}\PYZsh{}\PYZsh{}\PYZsh{}\PYZsh{}\PYZsh{}\PYZsh{}\PYZsh{}\PYZsh{}\PYZsh{}\PYZsh{}\PYZsh{}\PYZsh{}\PYZsh{}\PYZsh{}\PYZsh{}\PYZsh{}\PYZsh{}\PYZsh{}\PYZsh{}\PYZsh{}\PYZsh{}\PYZsh{}\PYZsh{}\PYZsh{}\PYZsh{}\PYZsh{}\PYZsh{}\PYZsh{}\PYZsh{}\PYZsh{}\PYZsh{}\PYZsh{}\PYZsh{}\PYZsh{}\PYZsh{}\PYZsh{}\PYZsh{}\PYZsh{}\PYZsh{}\PYZsh{}}

\PY{c+c1}{\PYZsh{}}
\PY{c+c1}{\PYZsh{} Same inputs of previous functions, with folds (aka $k$\PYZhy{}fold cross  }
\PY{c+c1}{\PYZsh{} validation) and reps is the number of times we re\PYZhy{}do the cross }
\PY{c+c1}{\PYZsh{} validation fold=1 or the length of the input (i.e. n) creates }
\PY{c+c1}{\PYZsh{} leave\PYZhy{}one\PYZhy{}out cross validation}

paretoRanking\PY{o}{\PYZlt{}\PYZhy{}}\PY{k+kr}{function}\PY{p}{(}objmatrix\PY{p}{,}istominvec\PY{p}{,}noFronts\PY{o}{=}\PY{l+m}{20}\PY{p}{,}folds\PY{o}{=}\PY{l+m}{1}\PY{p}{,}reps\PY{o}{=}\PY{l+m}{5}\PY{p}{)}
\PY{p}{\PYZob{}}
	objmatrix\PY{p}{[}\PY{p}{,}istominvec\PY{p}{]}\PY{o}{\PYZlt{}\PYZhy{}} \PY{o}{\PYZhy{}}objmatrix\PY{p}{[}\PY{p}{,}istominvec\PY{p}{]}
	m\PY{o}{\PYZlt{}\PYZhy{}}\PY{k+kp}{ncol}\PY{p}{(}objmatrix\PY{p}{)}
	n\PY{o}{\PYZlt{}\PYZhy{}}\PY{k+kp}{nrow}\PY{p}{(}objmatrix\PY{p}{)}
	pfmetric\PY{o}{\PYZlt{}\PYZhy{}}\PY{k+kp}{rep}\PY{p}{(}\PY{l+m}{0}\PY{p}{,}n\PY{p}{)} \PY{c+c1}{\PYZsh{}output vector}
	nfolds\PY{o}{\PYZlt{}\PYZhy{}}n
	\PY{k+kr}{if}\PY{p}{(}folds\PY{o}{\PYZgt{}}\PY{l+m}{1}\PY{p}{)} nfolds\PY{o}{\PYZlt{}\PYZhy{}}folds
	\PY{k+kr}{if}\PY{p}{(}nfolds\PY{o}{==}n\PY{p}{)} reps\PY{o}{\PYZlt{}\PYZhy{}}\PY{l+m}{1}
	
	blocks\PY{o}{\PYZlt{}\PYZhy{}}kfcv\PY{p}{(}nfolds\PY{p}{,}n\PY{p}{)}
	block.nos\PY{o}{\PYZlt{}\PYZhy{}}\PY{k+kp}{rep}\PY{p}{(}\PY{l+m}{1}\PY{o}{:}nfolds\PY{p}{,}blocks\PY{p}{)}
	
	\PY{k+kr}{for}\PY{p}{(}r \PY{k+kr}{in} \PY{l+m}{1}\PY{o}{:}reps\PY{p}{)}
	\PY{p}{\PYZob{}}
		indxs\PY{o}{\PYZlt{}\PYZhy{}}\PY{k+kp}{sample}\PY{p}{(}\PY{l+m}{1}\PY{o}{:}n\PY{p}{)} \PY{c+c1}{\PYZsh{}fresh randomisation each repetition}
		k.f.mat\PY{o}{\PYZlt{}\PYZhy{}}\PY{k+kp}{cbind}\PY{p}{(}indxs\PY{p}{,}block.nos\PY{p}{)} \PY{c+c1}{\PYZsh{} create the fold \PYZsq{}blocks\PYZsq{} of data}
		\PY{k+kr}{for}\PY{p}{(}i \PY{k+kr}{in} \PY{l+m}{1}\PY{o}{:}nfolds\PY{p}{)}
		\PY{p}{\PYZob{}}
			rows\PY{o}{\PYZlt{}\PYZhy{}}k.f.mat\PY{p}{[}k.f.mat\PY{p}{[}\PY{p}{,}\PY{l+m}{2}\PY{p}{]}\PY{o}{==}i\PY{p}{,}\PY{l+m}{1}\PY{p}{]} \PY{c+c1}{\PYZsh{} find the ith fold to leave out}
			\PY{c+c1}{\PYZsh{} call general function with ith fold removed}
			calcfronts\PY{o}{\PYZlt{}\PYZhy{}}paretoFronts\PY{p}{(}noFronts\PY{p}{,}objmatrix\PY{p}{[}\PY{o}{\PYZhy{}}rows\PY{p}{,}\PY{p}{]}\PY{p}{,}\PY{k+kp}{rep}\PY{p}{(}\PY{k+kc}{FALSE}\PY{p}{,}m\PY{p}{)}\PY{p}{)}
			\PY{c+c1}{\PYZsh{} which are non\PYZhy{}dominated}
			whichnondom\PY{o}{\PYZlt{}\PYZhy{}}calcfronts\PY{o}{\PYZgt{}}\PY{l+m}{0}
			\PY{c+c1}{\PYZsh{} if you are ont the first front you get 1, second=1/2, third=1/3,}
			\PY{c+c1}{\PYZsh{} ..., jth=1/j, else 0}
			pfmetric\PY{p}{[}\PY{o}{\PYZhy{}}rows\PY{p}{]}\PY{p}{[}whichnondom\PY{p}{]}\PY{o}{\PYZlt{}\PYZhy{}}
					pfmetric\PY{p}{[}\PY{o}{\PYZhy{}}rows\PY{p}{]}\PY{p}{[}whichnondom\PY{p}{]}\PY{l+m}{+1}\PY{o}{/}calcfronts\PY{p}{[}whichnondom\PY{p}{]}
		\PY{p}{\PYZcb{}}
	\PY{p}{\PYZcb{}}
	\PY{c+c1}{\PYZsh{}now divide by maximum posible value i.e. (nfolds\PYZhy{}1)*reps so output in [0,1]}
	pfmetric\PY{o}{\PYZlt{}\PYZhy{}}pfmetric\PY{o}{/}\PY{p}{(}\PY{p}{(}nfolds\PY{l+m}{\PYZhy{}1}\PY{p}{)}\PY{o}{*}reps\PY{p}{)} 
	\PY{k+kr}{return}\PY{p}{(}pfmetric\PY{p}{)}
\PY{p}{\PYZcb{}}

\PY{c+c1}{\PYZsh{}\PYZsh{}\PYZsh{}\PYZsh{}\PYZsh{}\PYZsh{}\PYZsh{}\PYZsh{}\PYZsh{}\PYZsh{}\PYZsh{}\PYZsh{}\PYZsh{}\PYZsh{}\PYZsh{}\PYZsh{}\PYZsh{}\PYZsh{}\PYZsh{}\PYZsh{}\PYZsh{}\PYZsh{}\PYZsh{}\PYZsh{}\PYZsh{}\PYZsh{}\PYZsh{}\PYZsh{}\PYZsh{}\PYZsh{}\PYZsh{}\PYZsh{}\PYZsh{}\PYZsh{}\PYZsh{}\PYZsh{}\PYZsh{}\PYZsh{}\PYZsh{}\PYZsh{}\PYZsh{}\PYZsh{}\PYZsh{}\PYZsh{}\PYZsh{}\PYZsh{}\PYZsh{}\PYZsh{}\PYZsh{}\PYZsh{}\PYZsh{}\PYZsh{}\PYZsh{}\PYZsh{}\PYZsh{}\PYZsh{}\PYZsh{}\PYZsh{}\PYZsh{}\PYZsh{}\PYZsh{}\PYZsh{}\PYZsh{}\PYZsh{}\PYZsh{}\PYZsh{}\PYZsh{}\PYZsh{}\PYZsh{}\PYZsh{}\PYZsh{}\PYZsh{}\PYZsh{}\PYZsh{}\PYZsh{}}
\PY{c+c1}{\PYZsh{} Below are three metrics that can possibly describe the value of      }
\PY{c+c1}{\PYZsh{}   variables/fetures to discriminate betwwen classes                                                             \PYZsh{}\PYZsh{}\PYZsh{}\PYZsh{}}
\PY{c+c1}{\PYZsh{}\PYZsh{}\PYZsh{}\PYZsh{}\PYZsh{}\PYZsh{}\PYZsh{}\PYZsh{}\PYZsh{}\PYZsh{}\PYZsh{}\PYZsh{}\PYZsh{}\PYZsh{}\PYZsh{}\PYZsh{}\PYZsh{}\PYZsh{}\PYZsh{}\PYZsh{}\PYZsh{}\PYZsh{}\PYZsh{}\PYZsh{}\PYZsh{}\PYZsh{}\PYZsh{}\PYZsh{}\PYZsh{}\PYZsh{}\PYZsh{}\PYZsh{}\PYZsh{}\PYZsh{}\PYZsh{}\PYZsh{}\PYZsh{}\PYZsh{}\PYZsh{}\PYZsh{}\PYZsh{}\PYZsh{}\PYZsh{}\PYZsh{}\PYZsh{}\PYZsh{}\PYZsh{}\PYZsh{}\PYZsh{}\PYZsh{}\PYZsh{}\PYZsh{}\PYZsh{}\PYZsh{}\PYZsh{}\PYZsh{}\PYZsh{}\PYZsh{}\PYZsh{}\PYZsh{}\PYZsh{}\PYZsh{}\PYZsh{}\PYZsh{}\PYZsh{}\PYZsh{}\PYZsh{}\PYZsh{}\PYZsh{}\PYZsh{}\PYZsh{}\PYZsh{}\PYZsh{}\PYZsh{}\PYZsh{}}

\PY{c+c1}{\PYZsh{} minIntraClassVar(): find the minimum WITHIN class variance of the K groups}
\PY{c+c1}{\PYZsh{} interClassVar(): find the variance of means/centroids of the K groups}
\PY{c+c1}{\PYZsh{} maxInterClassDist(): possibly correlated with interClassVar, find the dist }
\PY{c+c1}{\PYZsh{}							MAXIMUM between the K group\PYZsq{}s centroids/means}
                         
\PY{c+c1}{\PYZsh{} The rationale of the last metric is that a variable/feature that only }
\PY{c+c1}{\PYZsh{} seperates two of the K classes  is undervalued by the Fisher score because }
\PY{c+c1}{\PYZsh{} it may not separate the K\PYZhy{}2 classes remaining well.	}
\PY{c+c1}{\PYZsh{} ... And a separation of two classes (in conjunction with other variables) }
\PY{c+c1}{\PYZsh{}      is important information for the discriminant model}

\PY{c+c1}{\PYZsh{}\PYZsh{}\PYZsh{}\PYZsh{}\PYZsh{}\PYZsh{}\PYZsh{}\PYZsh{}\PYZsh{}\PYZsh{}\PYZsh{}\PYZsh{}\PYZsh{}\PYZsh{}\PYZsh{}\PYZsh{}\PYZsh{}\PYZsh{}\PYZsh{}\PYZsh{}\PYZsh{}\PYZsh{}\PYZsh{}\PYZsh{}\PYZsh{}\PYZsh{}\PYZsh{}\PYZsh{}\PYZsh{}\PYZsh{}\PYZsh{}\PYZsh{}\PYZsh{}\PYZsh{}\PYZsh{}\PYZsh{}\PYZsh{}\PYZsh{}\PYZsh{}\PYZsh{}\PYZsh{}\PYZsh{}\PYZsh{}\PYZsh{}\PYZsh{}\PYZsh{}\PYZsh{}\PYZsh{}\PYZsh{}\PYZsh{}\PYZsh{}\PYZsh{}\PYZsh{}\PYZsh{}\PYZsh{}\PYZsh{}\PYZsh{}\PYZsh{}\PYZsh{}\PYZsh{}\PYZsh{}\PYZsh{}\PYZsh{}\PYZsh{}\PYZsh{}\PYZsh{}\PYZsh{}\PYZsh{}\PYZsh{}\PYZsh{}\PYZsh{}\PYZsh{}\PYZsh{}\PYZsh{}\PYZsh{}}
\PY{c+c1}{\PYZsh{}\PYZsh{}\PYZsh{}\PYZsh{} ds: a data.frame or matrix (numeric values only/factors will be dealt }
\PY{c+c1}{\PYZsh{}\PYZsh{}\PYZsh{}\PYZsh{}	with as integers)  class vec: a vector correspong to the class of }
\PY{c+c1}{\PYZsh{}\PYZsh{}\PYZsh{}\PYZsh{}    the rows of ds           }
\PY{c+c1}{\PYZsh{}\PYZsh{}\PYZsh{}\PYZsh{}\PYZsh{}\PYZsh{}\PYZsh{}\PYZsh{}\PYZsh{}\PYZsh{}\PYZsh{}\PYZsh{}\PYZsh{}\PYZsh{}\PYZsh{}\PYZsh{}\PYZsh{}\PYZsh{}\PYZsh{}\PYZsh{}\PYZsh{}\PYZsh{}\PYZsh{}\PYZsh{}\PYZsh{}\PYZsh{}\PYZsh{}\PYZsh{}\PYZsh{}\PYZsh{}\PYZsh{}\PYZsh{}\PYZsh{}\PYZsh{}\PYZsh{}\PYZsh{}\PYZsh{}\PYZsh{}\PYZsh{}\PYZsh{}\PYZsh{}\PYZsh{}\PYZsh{}\PYZsh{}\PYZsh{}\PYZsh{}\PYZsh{}\PYZsh{}\PYZsh{}\PYZsh{}\PYZsh{}\PYZsh{}\PYZsh{}\PYZsh{}\PYZsh{}\PYZsh{}\PYZsh{}\PYZsh{}\PYZsh{}\PYZsh{}\PYZsh{}\PYZsh{}\PYZsh{}\PYZsh{}\PYZsh{}\PYZsh{}\PYZsh{}\PYZsh{}\PYZsh{}\PYZsh{}\PYZsh{}\PYZsh{}\PYZsh{}\PYZsh{}\PYZsh{}}

minIntraClassVar\PY{o}{\PYZlt{}\PYZhy{}}\PY{k+kr}{function}\PY{p}{(}ds\PY{p}{,}class.vec\PY{p}{)}\PY{p}{\PYZob{}}

	dsfs\PY{o}{\PYZlt{}\PYZhy{}}ds
	\PY{k+kr}{if}\PY{p}{(}\PY{o}{!}\PY{k+kp}{is.matrix}\PY{p}{(}dsfs\PY{p}{)}\PY{p}{)} dsfs\PY{o}{\PYZlt{}\PYZhy{}}\PY{k+kp}{data.matrix}\PY{p}{(}dsfs\PY{p}{)}  
	
	p\PY{o}{\PYZlt{}\PYZhy{}}\PY{k+kp}{ncol}\PY{p}{(}dsfs\PY{p}{)}
	K\PY{o}{\PYZlt{}\PYZhy{}}\PY{k+kp}{length}\PY{p}{(}\PY{k+kp}{levels}\PY{p}{(}class.vec\PY{p}{)}\PY{p}{)}
	n.all\PY{o}{\PYZlt{}\PYZhy{}}\PY{k+kp}{length}\PY{p}{(}class.vec\PY{p}{)}
	n.i\PY{o}{\PYZlt{}\PYZhy{}}\PY{l+m}{0}
	
	intraClassVar\PY{o}{\PYZlt{}\PYZhy{}}\PY{k+kc}{Inf}
	mean.j\PY{o}{\PYZlt{}\PYZhy{}}\PY{k+kp}{colMeans}\PY{p}{(}dsfs\PY{p}{)}
	
	\PY{k+kr}{for}\PY{p}{(}i \PY{k+kr}{in} \PY{l+m}{1}\PY{o}{:}K\PY{p}{)}\PY{p}{\PYZob{}}
		true.vec\PY{o}{\PYZlt{}\PYZhy{}}\PY{p}{(}\PY{k+kp}{as.integer}\PY{p}{(}class.vec\PY{p}{)}\PY{o}{==}i\PY{p}{)}
		n.i\PY{o}{\PYZlt{}\PYZhy{}}\PY{k+kp}{length}\PY{p}{(}\PY{k+kp}{which}\PY{p}{(}true.vec\PY{p}{)}\PY{p}{)}
		mean.class\PY{o}{\PYZlt{}\PYZhy{}}\PY{k+kp}{colMeans}\PY{p}{(}\PY{k+kp}{as.matrix}\PY{p}{(}dsfs\PY{p}{[}true.vec\PY{p}{,}\PY{p}{]}\PY{p}{,}ncol\PY{o}{=}p\PY{p}{)}\PY{p}{)}
		var.class\PY{o}{\PYZlt{}\PYZhy{}}\PY{k+kp}{colSums}\PY{p}{(}\PY{k+kp}{as.matrix}\PY{p}{(}\PY{p}{(}dsfs\PY{p}{[}true.vec\PY{p}{,}\PY{p}{]}\PY{o}{\PYZhy{}}\PY{k+kp}{rep}\PY{p}{(}mean.class\PY{p}{,}each\PY{o}{=}n.i\PY{p}{)}\PY{p}{)}\PY{o}{\PYZca{}}\PY{l+m}{2}\PY{p}{,}ncol\PY{o}{=}p\PY{p}{)}\PY{p}{)}
		intraClassVar\PY{o}{\PYZlt{}\PYZhy{}}\PY{k+kp}{pmin}\PY{p}{(}intraClassVar\PY{p}{,}var.class\PY{o}{/}\PY{p}{(}n.i\PY{l+m}{\PYZhy{}1}\PY{p}{)}\PY{p}{)}
	\PY{p}{\PYZcb{}}
	\PY{k+kr}{return}\PY{p}{(}intraClassVar\PY{p}{)}
\PY{p}{\PYZcb{}}

interClassVar\PY{o}{\PYZlt{}\PYZhy{}}\PY{k+kr}{function}\PY{p}{(}ds\PY{p}{,}class.vec\PY{p}{)}\PY{p}{\PYZob{}}

	dsfs\PY{o}{\PYZlt{}\PYZhy{}}ds
	\PY{k+kr}{if}\PY{p}{(}\PY{o}{!}\PY{k+kp}{is.matrix}\PY{p}{(}dsfs\PY{p}{)}\PY{p}{)} dsfs\PY{o}{\PYZlt{}\PYZhy{}}\PY{k+kp}{data.matrix}\PY{p}{(}dsfs\PY{p}{)}  
	
	p\PY{o}{\PYZlt{}\PYZhy{}}\PY{k+kp}{ncol}\PY{p}{(}dsfs\PY{p}{)}
	K\PY{o}{\PYZlt{}\PYZhy{}}\PY{k+kp}{length}\PY{p}{(}\PY{k+kp}{levels}\PY{p}{(}class.vec\PY{p}{)}\PY{p}{)}
	
	interClassVar\PY{o}{\PYZlt{}\PYZhy{}}\PY{l+m}{0}
	mean.j\PY{o}{\PYZlt{}\PYZhy{}}\PY{k+kp}{colMeans}\PY{p}{(}dsfs\PY{p}{)}
	
	\PY{k+kr}{for}\PY{p}{(}i \PY{k+kr}{in} \PY{l+m}{1}\PY{o}{:}K\PY{p}{)}\PY{p}{\PYZob{}}
		true.vec\PY{o}{\PYZlt{}\PYZhy{}}\PY{p}{(}\PY{k+kp}{as.integer}\PY{p}{(}class.vec\PY{p}{)}\PY{o}{==}i\PY{p}{)}
		n.i\PY{o}{\PYZlt{}\PYZhy{}}\PY{k+kp}{length}\PY{p}{(}\PY{k+kp}{which}\PY{p}{(}true.vec\PY{p}{)}\PY{p}{)}
		mean.class\PY{o}{\PYZlt{}\PYZhy{}}\PY{k+kp}{colMeans}\PY{p}{(}\PY{k+kp}{as.matrix}\PY{p}{(}dsfs\PY{p}{[}true.vec\PY{p}{,}\PY{p}{]}\PY{p}{,}ncol\PY{o}{=}p\PY{p}{)}\PY{p}{)}
		var.class\PY{o}{\PYZlt{}\PYZhy{}}\PY{p}{(}\PY{p}{(}mean.j\PY{o}{\PYZhy{}}mean.class\PY{p}{)}\PY{o}{\PYZca{}}\PY{l+m}{2}\PY{p}{)}
		interClassVar\PY{o}{\PYZlt{}\PYZhy{}}interClassVar\PY{o}{+}var.class
	\PY{p}{\PYZcb{}}
	\PY{k+kr}{return}\PY{p}{(}interClassVar\PY{o}{/}\PY{p}{(}K\PY{l+m}{\PYZhy{}1}\PY{p}{)}\PY{p}{)}
\PY{p}{\PYZcb{}}

maxInterClassDist\PY{o}{\PYZlt{}\PYZhy{}}\PY{k+kr}{function}\PY{p}{(}ds\PY{p}{,}class.vec\PY{p}{)}\PY{p}{\PYZob{}}

	dsfs\PY{o}{\PYZlt{}\PYZhy{}}ds
	\PY{k+kr}{if}\PY{p}{(}\PY{o}{!}\PY{k+kp}{is.matrix}\PY{p}{(}dsfs\PY{p}{)}\PY{p}{)} dsfs\PY{o}{\PYZlt{}\PYZhy{}}\PY{k+kp}{data.matrix}\PY{p}{(}dsfs\PY{p}{)}  
	
	p\PY{o}{\PYZlt{}\PYZhy{}}\PY{k+kp}{ncol}\PY{p}{(}dsfs\PY{p}{)}
	K\PY{o}{\PYZlt{}\PYZhy{}}\PY{k+kp}{length}\PY{p}{(}\PY{k+kp}{levels}\PY{p}{(}class.vec\PY{p}{)}\PY{p}{)}
	
	interClassDist\PY{o}{\PYZlt{}\PYZhy{}}\PY{o}{\PYZhy{}}\PY{k+kc}{Inf}
	mean.j\PY{o}{\PYZlt{}\PYZhy{}}\PY{k+kp}{colMeans}\PY{p}{(}dsfs\PY{p}{)}
	
	\PY{k+kr}{for}\PY{p}{(}i \PY{k+kr}{in} \PY{l+m}{1}\PY{o}{:}K\PY{p}{)}\PY{p}{\PYZob{}}
		true.vec\PY{o}{\PYZlt{}\PYZhy{}}\PY{p}{(}\PY{k+kp}{as.integer}\PY{p}{(}class.vec\PY{p}{)}\PY{o}{==}i\PY{p}{)}
		mean.class\PY{o}{\PYZlt{}\PYZhy{}}\PY{k+kp}{colMeans}\PY{p}{(}\PY{k+kp}{as.matrix}\PY{p}{(}dsfs\PY{p}{[}true.vec\PY{p}{,}\PY{p}{]}\PY{p}{,}ncol\PY{o}{=}p\PY{p}{)}\PY{p}{)}
		interClassDist\PY{o}{\PYZlt{}\PYZhy{}}\PY{k+kp}{pmax}\PY{p}{(}interClassDist\PY{p}{,}\PY{k+kp}{abs}\PY{p}{(}mean.j\PY{o}{\PYZhy{}}mean.class\PY{p}{)}\PY{p}{)}
	\PY{p}{\PYZcb{}}
	\PY{k+kr}{return}\PY{p}{(}interClassDist\PY{p}{)}
\PY{p}{\PYZcb{}}
\end{Verbatim}



\end{appendix}






%\addtocontents{toc}{\setcounter{tocdepth}{2}}

%%%%%%%%%%%%%%%%%%%%%%%%%%%%%%%%%
%%%%%%%%%%%%%%%%%%%%%%%%%%%%%%%%%
%%%%%%%%%      Bibliography        %%%%%%%%%%
%%%%%%%%%%%%%%%%%%%%%%%%%%%%%%%%%
%%%%%%%%%%%%%%%%%%%%%%%%%%%%%%%%%







\end{document}  


